% \iffalse meta-comment
%
% File: makerobust.dtx
% Version: 2006/03/18 v1.0
% Info: Make existing macro robust
%
% Copyright (C) 2006 by
%    Heiko Oberdiek <heiko.oberdiek at googlemail.com>
%
% This work may be distributed and/or modified under the
% conditions of the LaTeX Project Public License, either
% version 1.3c of this license or (at your option) any later
% version. This version of this license is in
%    http://www.latex-project.org/lppl/lppl-1-3c.txt
% and the latest version of this license is in
%    http://www.latex-project.org/lppl.txt
% and version 1.3 or later is part of all distributions of
% LaTeX version 2005/12/01 or later.
%
% This work has the LPPL maintenance status "maintained".
%
% This Current Maintainer of this work is Heiko Oberdiek.
%
% This work consists of the main source file makerobust.dtx
% and the derived files
%    makerobust.sty, makerobust.pdf, makerobust.ins, makerobust.drv,
%    makerobust-example.tex.
%
% Distribution:
%    CTAN:macros/latex/contrib/oberdiek/makerobust.dtx
%    CTAN:macros/latex/contrib/oberdiek/makerobust.pdf
%
% Unpacking:
%    (a) If makerobust.ins is present:
%           tex makerobust.ins
%    (b) Without makerobust.ins:
%           tex makerobust.dtx
%    (c) If you insist on using LaTeX
%           latex \let\install=y% \iffalse meta-comment
%
% File: makerobust.dtx
% Version: 2006/03/18 v1.0
% Info: Make existing macro robust
%
% Copyright (C) 2006 by
%    Heiko Oberdiek <heiko.oberdiek at googlemail.com>
%
% This work may be distributed and/or modified under the
% conditions of the LaTeX Project Public License, either
% version 1.3c of this license or (at your option) any later
% version. This version of this license is in
%    http://www.latex-project.org/lppl/lppl-1-3c.txt
% and the latest version of this license is in
%    http://www.latex-project.org/lppl.txt
% and version 1.3 or later is part of all distributions of
% LaTeX version 2005/12/01 or later.
%
% This work has the LPPL maintenance status "maintained".
%
% This Current Maintainer of this work is Heiko Oberdiek.
%
% This work consists of the main source file makerobust.dtx
% and the derived files
%    makerobust.sty, makerobust.pdf, makerobust.ins, makerobust.drv,
%    makerobust-example.tex.
%
% Distribution:
%    CTAN:macros/latex/contrib/oberdiek/makerobust.dtx
%    CTAN:macros/latex/contrib/oberdiek/makerobust.pdf
%
% Unpacking:
%    (a) If makerobust.ins is present:
%           tex makerobust.ins
%    (b) Without makerobust.ins:
%           tex makerobust.dtx
%    (c) If you insist on using LaTeX
%           latex \let\install=y% \iffalse meta-comment
%
% File: makerobust.dtx
% Version: 2006/03/18 v1.0
% Info: Make existing macro robust
%
% Copyright (C) 2006 by
%    Heiko Oberdiek <heiko.oberdiek at googlemail.com>
%
% This work may be distributed and/or modified under the
% conditions of the LaTeX Project Public License, either
% version 1.3c of this license or (at your option) any later
% version. This version of this license is in
%    http://www.latex-project.org/lppl/lppl-1-3c.txt
% and the latest version of this license is in
%    http://www.latex-project.org/lppl.txt
% and version 1.3 or later is part of all distributions of
% LaTeX version 2005/12/01 or later.
%
% This work has the LPPL maintenance status "maintained".
%
% This Current Maintainer of this work is Heiko Oberdiek.
%
% This work consists of the main source file makerobust.dtx
% and the derived files
%    makerobust.sty, makerobust.pdf, makerobust.ins, makerobust.drv,
%    makerobust-example.tex.
%
% Distribution:
%    CTAN:macros/latex/contrib/oberdiek/makerobust.dtx
%    CTAN:macros/latex/contrib/oberdiek/makerobust.pdf
%
% Unpacking:
%    (a) If makerobust.ins is present:
%           tex makerobust.ins
%    (b) Without makerobust.ins:
%           tex makerobust.dtx
%    (c) If you insist on using LaTeX
%           latex \let\install=y% \iffalse meta-comment
%
% File: makerobust.dtx
% Version: 2006/03/18 v1.0
% Info: Make existing macro robust
%
% Copyright (C) 2006 by
%    Heiko Oberdiek <heiko.oberdiek at googlemail.com>
%
% This work may be distributed and/or modified under the
% conditions of the LaTeX Project Public License, either
% version 1.3c of this license or (at your option) any later
% version. This version of this license is in
%    http://www.latex-project.org/lppl/lppl-1-3c.txt
% and the latest version of this license is in
%    http://www.latex-project.org/lppl.txt
% and version 1.3 or later is part of all distributions of
% LaTeX version 2005/12/01 or later.
%
% This work has the LPPL maintenance status "maintained".
%
% This Current Maintainer of this work is Heiko Oberdiek.
%
% This work consists of the main source file makerobust.dtx
% and the derived files
%    makerobust.sty, makerobust.pdf, makerobust.ins, makerobust.drv,
%    makerobust-example.tex.
%
% Distribution:
%    CTAN:macros/latex/contrib/oberdiek/makerobust.dtx
%    CTAN:macros/latex/contrib/oberdiek/makerobust.pdf
%
% Unpacking:
%    (a) If makerobust.ins is present:
%           tex makerobust.ins
%    (b) Without makerobust.ins:
%           tex makerobust.dtx
%    (c) If you insist on using LaTeX
%           latex \let\install=y\input{makerobust.dtx}
%        (quote the arguments according to the demands of your shell)
%
% Documentation:
%    (a) If makerobust.drv is present:
%           latex makerobust.drv
%    (b) Without makerobust.drv:
%           latex makerobust.dtx; ...
%    The class ltxdoc loads the configuration file ltxdoc.cfg
%    if available. Here you can specify further options, e.g.
%    use A4 as paper format:
%       \PassOptionsToClass{a4paper}{article}
%
%    Programm calls to get the documentation (example):
%       pdflatex makerobust.dtx
%       makeindex -s gind.ist makerobust.idx
%       pdflatex makerobust.dtx
%       makeindex -s gind.ist makerobust.idx
%       pdflatex makerobust.dtx
%
% Installation:
%    TDS:tex/latex/oberdiek/makerobust.sty
%    TDS:doc/latex/oberdiek/makerobust.pdf
%    TDS:doc/latex/oberdiek/makerobust-example.tex
%    TDS:source/latex/oberdiek/makerobust.dtx
%
%<*ignore>
\begingroup
  \catcode123=1 %
  \catcode125=2 %
  \def\x{LaTeX2e}%
\expandafter\endgroup
\ifcase 0\ifx\install y1\fi\expandafter
         \ifx\csname processbatchFile\endcsname\relax\else1\fi
         \ifx\fmtname\x\else 1\fi\relax
\else\csname fi\endcsname
%</ignore>
%<*install>
\input docstrip.tex
\Msg{************************************************************************}
\Msg{* Installation}
\Msg{* Package: makerobust 2006/03/18 v1.0 Make existing macro robust (HO)}
\Msg{************************************************************************}

\keepsilent
\askforoverwritefalse

\let\MetaPrefix\relax
\preamble

This is a generated file.

Project: makerobust
Version: 2006/03/18 v1.0

Copyright (C) 2006 by
   Heiko Oberdiek <heiko.oberdiek at googlemail.com>

This work may be distributed and/or modified under the
conditions of the LaTeX Project Public License, either
version 1.3c of this license or (at your option) any later
version. This version of this license is in
   http://www.latex-project.org/lppl/lppl-1-3c.txt
and the latest version of this license is in
   http://www.latex-project.org/lppl.txt
and version 1.3 or later is part of all distributions of
LaTeX version 2005/12/01 or later.

This work has the LPPL maintenance status "maintained".

This Current Maintainer of this work is Heiko Oberdiek.

This work consists of the main source file makerobust.dtx
and the derived files
   makerobust.sty, makerobust.pdf, makerobust.ins, makerobust.drv,
   makerobust-example.tex.

\endpreamble
\let\MetaPrefix\DoubleperCent

\generate{%
  \file{makerobust.ins}{\from{makerobust.dtx}{install}}%
  \file{makerobust.drv}{\from{makerobust.dtx}{driver}}%
  \usedir{tex/latex/oberdiek}%
  \file{makerobust.sty}{\from{makerobust.dtx}{package}}%
  \usedir{doc/latex/oberdiek}%
  \file{makerobust-example.tex}{\from{makerobust.dtx}{example}}%
  \nopreamble
  \nopostamble
  \usedir{source/latex/oberdiek/catalogue}%
  \file{makerobust.xml}{\from{makerobust.dtx}{catalogue}}%
}

\catcode32=13\relax% active space
\let =\space%
\Msg{************************************************************************}
\Msg{*}
\Msg{* To finish the installation you have to move the following}
\Msg{* file into a directory searched by TeX:}
\Msg{*}
\Msg{*     makerobust.sty}
\Msg{*}
\Msg{* To produce the documentation run the file `makerobust.drv'}
\Msg{* through LaTeX.}
\Msg{*}
\Msg{* Happy TeXing!}
\Msg{*}
\Msg{************************************************************************}

\endbatchfile
%</install>
%<*ignore>
\fi
%</ignore>
%<*driver>
\NeedsTeXFormat{LaTeX2e}
\ProvidesFile{makerobust.drv}%
  [2006/03/18 v1.0 Make existing macro robust (HO)]%
\documentclass{ltxdoc}
\usepackage{holtxdoc}[2011/11/22]
\begin{document}
  \DocInput{makerobust.dtx}%
\end{document}
%</driver>
% \fi
%
% \CheckSum{59}
%
% \CharacterTable
%  {Upper-case    \A\B\C\D\E\F\G\H\I\J\K\L\M\N\O\P\Q\R\S\T\U\V\W\X\Y\Z
%   Lower-case    \a\b\c\d\e\f\g\h\i\j\k\l\m\n\o\p\q\r\s\t\u\v\w\x\y\z
%   Digits        \0\1\2\3\4\5\6\7\8\9
%   Exclamation   \!     Double quote  \"     Hash (number) \#
%   Dollar        \$     Percent       \%     Ampersand     \&
%   Acute accent  \'     Left paren    \(     Right paren   \)
%   Asterisk      \*     Plus          \+     Comma         \,
%   Minus         \-     Point         \.     Solidus       \/
%   Colon         \:     Semicolon     \;     Less than     \<
%   Equals        \=     Greater than  \>     Question mark \?
%   Commercial at \@     Left bracket  \[     Backslash     \\
%   Right bracket \]     Circumflex    \^     Underscore    \_
%   Grave accent  \`     Left brace    \{     Vertical bar  \|
%   Right brace   \}     Tilde         \~}
%
% \GetFileInfo{makerobust.drv}
%
% \title{The \xpackage{makerobust} package}
% \date{2006/03/18 v1.0}
% \author{Heiko Oberdiek\\\xemail{heiko.oberdiek at googlemail.com}}
%
% \maketitle
%
% \begin{abstract}
% Package \xpackage{makerobust} provides \cs{MakeRobustCommand}
% that converts an existing macro to a robust one.
% \end{abstract}
%
% \tableofcontents
%
% \section{User interface}
%
% \LaTeX\ offers \cs{DeclareRobustCommand} to define a robust macro
% that does not break if it is used in moving arguments.
% Sometimes a macro is already defined, but not robust. For
% example, \cs{(} and \cs{)} are not robust, inside \cs{section}
% the user must use \cs{protect} explicitly. This could be
% avoided by making \cs{(} and \cs{)} robust.
%
% \begin{declcs}{MakeRobustCommand}\M{cmd}
% \end{declcs}
% \cs{MakeRobustCommand} redefines the macro \meta{cmd}
% by using \cs{DeclareRobustCommand} and the existing definition
% of the macro \meta{cmd}.
% \begin{itemize}
% \item It is an error if \meta{cmd} is undefined. If you want to
%   define a robust command, then you can use \cs{DeclareRobustCommand}
%   directly.
% \item If the macro has previously been
%   defined by \cs{DeclareRobustCommand} then the redefinition of
%   \cs{MakeRobustCommand} is omitted, because the macro is already robust.
%   Only an information entry is written to the \xfile{.log} file.
%   Thus you do not get a warning or an error if the macro is already
%   robust because of an updated LaTeX or package that defines the macro.
% \item Two macros are defined for a macro, defined
%   by \cs{DeclareRobustCommand}. Example:
%   \begin{quote}
%   |\DeclareRobustCommand{\foobar}{definition text}|
%   \end{quote}
%   Then the macro ``\cs{foobar}'' contains the protection code
%   and, depending on the protection mode,
%   calls the internal macro ``\cs{foobar }''. Notice the space
%   at the end of the macro name.
%   This internal macro ``\cs{foobar }'' now contains the definition
%   ``|definition text|'', given in \cs{DeclareRobustCommand}.
%
%   Sometimes it can happen, that the internal macro already exists.
%   This can be caused by a previous \cs{DeclareRobustCommand} followed
%   by \cs{renewcommand}. Then the redefinition by \cs{MakeRobustCommand}
%   would be safe.
%
%   However, it can also be possible that the macro is already robust,
%   using the internal macro, but with a different protection code.
%   The redefinition by \cs{MakeRobustCommand} would then generate
%   an infinite loop.
%
%   Therefore \cs{MakeRobustCommand} raises an error message,
%   if the internal macro (with space at the end) already exists.
% \end{itemize}
%
% \subsection{Example}
%
%    \begin{macrocode}
%<*example>
\documentclass{article}
\usepackage{makerobust}
\MakeRobustCommand\(
\MakeRobustCommand\)
\pagestyle{headings}
\begin{document}
\tableofcontents
\section{Einstein: \(E=mc^2\)}
\newpage
Second page.
\end{document}
%</example>
%    \end{macrocode}
%
%
% \StopEventually{
% }
%
% \section{Implementation}
%
%    \begin{macrocode}
%<*package>
\NeedsTeXFormat{LaTeX2e}
\ProvidesPackage{makerobust}%
  [2006/03/18 v1.0 Make existing macro robust (HO)]%
%    \end{macrocode}
%
%    \begin{macrocode}
\def\MakeRobustCommand#1{%
  \begingroup
  \@ifundefined{\expandafter\@gobble\string#1}{%
    \endgroup
    \PackageError{makerobust}{%
      Macro \string`\string#1\string' is not defined%
    }\@ehc
  }{%
    \global\let\MR@gtemp#1%
    \let#1\@undefined
    \expandafter\let\expandafter\MR@temp
        \csname\expandafter\@gobble\string#1 \endcsname
    \DeclareRobustCommand#1{}%
    \ifx#1\MR@gtemp
      \endgroup
      \PackageInfo{makerobust}{%
        \string`\string#1\string' is already robust%
      }%
    \else
      \@ifundefined{MR@temp}{%
        \global\let\MR@gtemp#1%
        \endgroup
        \expandafter\let\csname\expandafter\@gobble\string#1 \endcsname#1%
        \let#1\MR@gtemp
      }{%
        \endgroup
        \PackageError{makerobust}{%
          Internal macro \string`\string#1 \string' already exists%
        }\@ehc
      }%
    \fi
  }%
}
%    \end{macrocode}
%
%    \begin{macrocode}
%</package>
%    \end{macrocode}
%
% \section{Installation}
%
% \subsection{Download}
%
% \paragraph{Package.} This package is available on
% CTAN\footnote{\url{ftp://ftp.ctan.org/tex-archive/}}:
% \begin{description}
% \item[\CTAN{macros/latex/contrib/oberdiek/makerobust.dtx}] The source file.
% \item[\CTAN{macros/latex/contrib/oberdiek/makerobust.pdf}] Documentation.
% \end{description}
%
%
% \paragraph{Bundle.} All the packages of the bundle `oberdiek'
% are also available in a TDS compliant ZIP archive. There
% the packages are already unpacked and the documentation files
% are generated. The files and directories obey the TDS standard.
% \begin{description}
% \item[\CTAN{install/macros/latex/contrib/oberdiek.tds.zip}]
% \end{description}
% \emph{TDS} refers to the standard ``A Directory Structure
% for \TeX\ Files'' (\CTAN{tds/tds.pdf}). Directories
% with \xfile{texmf} in their name are usually organized this way.
%
% \subsection{Bundle installation}
%
% \paragraph{Unpacking.} Unpack the \xfile{oberdiek.tds.zip} in the
% TDS tree (also known as \xfile{texmf} tree) of your choice.
% Example (linux):
% \begin{quote}
%   |unzip oberdiek.tds.zip -d ~/texmf|
% \end{quote}
%
% \paragraph{Script installation.}
% Check the directory \xfile{TDS:scripts/oberdiek/} for
% scripts that need further installation steps.
% Package \xpackage{attachfile2} comes with the Perl script
% \xfile{pdfatfi.pl} that should be installed in such a way
% that it can be called as \texttt{pdfatfi}.
% Example (linux):
% \begin{quote}
%   |chmod +x scripts/oberdiek/pdfatfi.pl|\\
%   |cp scripts/oberdiek/pdfatfi.pl /usr/local/bin/|
% \end{quote}
%
% \subsection{Package installation}
%
% \paragraph{Unpacking.} The \xfile{.dtx} file is a self-extracting
% \docstrip\ archive. The files are extracted by running the
% \xfile{.dtx} through \plainTeX:
% \begin{quote}
%   \verb|tex makerobust.dtx|
% \end{quote}
%
% \paragraph{TDS.} Now the different files must be moved into
% the different directories in your installation TDS tree
% (also known as \xfile{texmf} tree):
% \begin{quote}
% \def\t{^^A
% \begin{tabular}{@{}>{\ttfamily}l@{ $\rightarrow$ }>{\ttfamily}l@{}}
%   makerobust.sty & tex/latex/oberdiek/makerobust.sty\\
%   makerobust.pdf & doc/latex/oberdiek/makerobust.pdf\\
%   makerobust-example.tex & doc/latex/oberdiek/makerobust-example.tex\\
%   makerobust.dtx & source/latex/oberdiek/makerobust.dtx\\
% \end{tabular}^^A
% }^^A
% \sbox0{\t}^^A
% \ifdim\wd0>\linewidth
%   \begingroup
%     \advance\linewidth by\leftmargin
%     \advance\linewidth by\rightmargin
%   \edef\x{\endgroup
%     \def\noexpand\lw{\the\linewidth}^^A
%   }\x
%   \def\lwbox{^^A
%     \leavevmode
%     \hbox to \linewidth{^^A
%       \kern-\leftmargin\relax
%       \hss
%       \usebox0
%       \hss
%       \kern-\rightmargin\relax
%     }^^A
%   }^^A
%   \ifdim\wd0>\lw
%     \sbox0{\small\t}^^A
%     \ifdim\wd0>\linewidth
%       \ifdim\wd0>\lw
%         \sbox0{\footnotesize\t}^^A
%         \ifdim\wd0>\linewidth
%           \ifdim\wd0>\lw
%             \sbox0{\scriptsize\t}^^A
%             \ifdim\wd0>\linewidth
%               \ifdim\wd0>\lw
%                 \sbox0{\tiny\t}^^A
%                 \ifdim\wd0>\linewidth
%                   \lwbox
%                 \else
%                   \usebox0
%                 \fi
%               \else
%                 \lwbox
%               \fi
%             \else
%               \usebox0
%             \fi
%           \else
%             \lwbox
%           \fi
%         \else
%           \usebox0
%         \fi
%       \else
%         \lwbox
%       \fi
%     \else
%       \usebox0
%     \fi
%   \else
%     \lwbox
%   \fi
% \else
%   \usebox0
% \fi
% \end{quote}
% If you have a \xfile{docstrip.cfg} that configures and enables \docstrip's
% TDS installing feature, then some files can already be in the right
% place, see the documentation of \docstrip.
%
% \subsection{Refresh file name databases}
%
% If your \TeX~distribution
% (\teTeX, \mikTeX, \dots) relies on file name databases, you must refresh
% these. For example, \teTeX\ users run \verb|texhash| or
% \verb|mktexlsr|.
%
% \subsection{Some details for the interested}
%
% \paragraph{Attached source.}
%
% The PDF documentation on CTAN also includes the
% \xfile{.dtx} source file. It can be extracted by
% AcrobatReader 6 or higher. Another option is \textsf{pdftk},
% e.g. unpack the file into the current directory:
% \begin{quote}
%   \verb|pdftk makerobust.pdf unpack_files output .|
% \end{quote}
%
% \paragraph{Unpacking with \LaTeX.}
% The \xfile{.dtx} chooses its action depending on the format:
% \begin{description}
% \item[\plainTeX:] Run \docstrip\ and extract the files.
% \item[\LaTeX:] Generate the documentation.
% \end{description}
% If you insist on using \LaTeX\ for \docstrip\ (really,
% \docstrip\ does not need \LaTeX), then inform the autodetect routine
% about your intention:
% \begin{quote}
%   \verb|latex \let\install=y\input{makerobust.dtx}|
% \end{quote}
% Do not forget to quote the argument according to the demands
% of your shell.
%
% \paragraph{Generating the documentation.}
% You can use both the \xfile{.dtx} or the \xfile{.drv} to generate
% the documentation. The process can be configured by the
% configuration file \xfile{ltxdoc.cfg}. For instance, put this
% line into this file, if you want to have A4 as paper format:
% \begin{quote}
%   \verb|\PassOptionsToClass{a4paper}{article}|
% \end{quote}
% An example follows how to generate the
% documentation with pdf\LaTeX:
% \begin{quote}
%\begin{verbatim}
%pdflatex makerobust.dtx
%makeindex -s gind.ist makerobust.idx
%pdflatex makerobust.dtx
%makeindex -s gind.ist makerobust.idx
%pdflatex makerobust.dtx
%\end{verbatim}
% \end{quote}
%
% \section{Catalogue}
%
% The following XML file can be used as source for the
% \href{http://mirror.ctan.org/help/Catalogue/catalogue.html}{\TeX\ Catalogue}.
% The elements \texttt{caption} and \texttt{description} are imported
% from the original XML file from the Catalogue.
% The name of the XML file in the Catalogue is \xfile{makerobust.xml}.
%    \begin{macrocode}
%<*catalogue>
<?xml version='1.0' encoding='us-ascii'?>
<!DOCTYPE entry SYSTEM 'catalogue.dtd'>
<entry datestamp='$Date$' modifier='$Author$' id='makerobust'>
  <name>makerobust</name>
  <caption>Making a macro robust.</caption>
  <authorref id='auth:oberdiek'/>
  <copyright owner='Heiko Oberdiek' year='2006'/>
  <license type='lppl1.3'/>
  <version number='1.0'/>
  <description>
    This package provides the command MakeRobustCommand
    that converts an existing macro to a robust one.
    <p/>
    The package is part of the <xref refid='oberdiek'>oberdiek</xref>
    bundle.
  </description>
  <documentation details='Package documentation'
      href='ctan:/macros/latex/contrib/oberdiek/makerobust.pdf'/>
  <ctan file='true' path='/macros/latex/contrib/oberdiek/makerobust.dtx'/>
  <miktex location='oberdiek'/>
  <texlive location='oberdiek'/>
  <install path='/macros/latex/contrib/oberdiek/oberdiek.tds.zip'/>
</entry>
%</catalogue>
%    \end{macrocode}
%
% \begin{History}
%   \begin{Version}{2006/03/18 v1.0}
%   \item
%     First version.
%   \end{Version}
% \end{History}
%
% \PrintIndex
%
% \Finale
\endinput

%        (quote the arguments according to the demands of your shell)
%
% Documentation:
%    (a) If makerobust.drv is present:
%           latex makerobust.drv
%    (b) Without makerobust.drv:
%           latex makerobust.dtx; ...
%    The class ltxdoc loads the configuration file ltxdoc.cfg
%    if available. Here you can specify further options, e.g.
%    use A4 as paper format:
%       \PassOptionsToClass{a4paper}{article}
%
%    Programm calls to get the documentation (example):
%       pdflatex makerobust.dtx
%       makeindex -s gind.ist makerobust.idx
%       pdflatex makerobust.dtx
%       makeindex -s gind.ist makerobust.idx
%       pdflatex makerobust.dtx
%
% Installation:
%    TDS:tex/latex/oberdiek/makerobust.sty
%    TDS:doc/latex/oberdiek/makerobust.pdf
%    TDS:doc/latex/oberdiek/makerobust-example.tex
%    TDS:source/latex/oberdiek/makerobust.dtx
%
%<*ignore>
\begingroup
  \catcode123=1 %
  \catcode125=2 %
  \def\x{LaTeX2e}%
\expandafter\endgroup
\ifcase 0\ifx\install y1\fi\expandafter
         \ifx\csname processbatchFile\endcsname\relax\else1\fi
         \ifx\fmtname\x\else 1\fi\relax
\else\csname fi\endcsname
%</ignore>
%<*install>
\input docstrip.tex
\Msg{************************************************************************}
\Msg{* Installation}
\Msg{* Package: makerobust 2006/03/18 v1.0 Make existing macro robust (HO)}
\Msg{************************************************************************}

\keepsilent
\askforoverwritefalse

\let\MetaPrefix\relax
\preamble

This is a generated file.

Project: makerobust
Version: 2006/03/18 v1.0

Copyright (C) 2006 by
   Heiko Oberdiek <heiko.oberdiek at googlemail.com>

This work may be distributed and/or modified under the
conditions of the LaTeX Project Public License, either
version 1.3c of this license or (at your option) any later
version. This version of this license is in
   http://www.latex-project.org/lppl/lppl-1-3c.txt
and the latest version of this license is in
   http://www.latex-project.org/lppl.txt
and version 1.3 or later is part of all distributions of
LaTeX version 2005/12/01 or later.

This work has the LPPL maintenance status "maintained".

This Current Maintainer of this work is Heiko Oberdiek.

This work consists of the main source file makerobust.dtx
and the derived files
   makerobust.sty, makerobust.pdf, makerobust.ins, makerobust.drv,
   makerobust-example.tex.

\endpreamble
\let\MetaPrefix\DoubleperCent

\generate{%
  \file{makerobust.ins}{\from{makerobust.dtx}{install}}%
  \file{makerobust.drv}{\from{makerobust.dtx}{driver}}%
  \usedir{tex/latex/oberdiek}%
  \file{makerobust.sty}{\from{makerobust.dtx}{package}}%
  \usedir{doc/latex/oberdiek}%
  \file{makerobust-example.tex}{\from{makerobust.dtx}{example}}%
  \nopreamble
  \nopostamble
  \usedir{source/latex/oberdiek/catalogue}%
  \file{makerobust.xml}{\from{makerobust.dtx}{catalogue}}%
}

\catcode32=13\relax% active space
\let =\space%
\Msg{************************************************************************}
\Msg{*}
\Msg{* To finish the installation you have to move the following}
\Msg{* file into a directory searched by TeX:}
\Msg{*}
\Msg{*     makerobust.sty}
\Msg{*}
\Msg{* To produce the documentation run the file `makerobust.drv'}
\Msg{* through LaTeX.}
\Msg{*}
\Msg{* Happy TeXing!}
\Msg{*}
\Msg{************************************************************************}

\endbatchfile
%</install>
%<*ignore>
\fi
%</ignore>
%<*driver>
\NeedsTeXFormat{LaTeX2e}
\ProvidesFile{makerobust.drv}%
  [2006/03/18 v1.0 Make existing macro robust (HO)]%
\documentclass{ltxdoc}
\usepackage{holtxdoc}[2011/11/22]
\begin{document}
  \DocInput{makerobust.dtx}%
\end{document}
%</driver>
% \fi
%
% \CheckSum{59}
%
% \CharacterTable
%  {Upper-case    \A\B\C\D\E\F\G\H\I\J\K\L\M\N\O\P\Q\R\S\T\U\V\W\X\Y\Z
%   Lower-case    \a\b\c\d\e\f\g\h\i\j\k\l\m\n\o\p\q\r\s\t\u\v\w\x\y\z
%   Digits        \0\1\2\3\4\5\6\7\8\9
%   Exclamation   \!     Double quote  \"     Hash (number) \#
%   Dollar        \$     Percent       \%     Ampersand     \&
%   Acute accent  \'     Left paren    \(     Right paren   \)
%   Asterisk      \*     Plus          \+     Comma         \,
%   Minus         \-     Point         \.     Solidus       \/
%   Colon         \:     Semicolon     \;     Less than     \<
%   Equals        \=     Greater than  \>     Question mark \?
%   Commercial at \@     Left bracket  \[     Backslash     \\
%   Right bracket \]     Circumflex    \^     Underscore    \_
%   Grave accent  \`     Left brace    \{     Vertical bar  \|
%   Right brace   \}     Tilde         \~}
%
% \GetFileInfo{makerobust.drv}
%
% \title{The \xpackage{makerobust} package}
% \date{2006/03/18 v1.0}
% \author{Heiko Oberdiek\\\xemail{heiko.oberdiek at googlemail.com}}
%
% \maketitle
%
% \begin{abstract}
% Package \xpackage{makerobust} provides \cs{MakeRobustCommand}
% that converts an existing macro to a robust one.
% \end{abstract}
%
% \tableofcontents
%
% \section{User interface}
%
% \LaTeX\ offers \cs{DeclareRobustCommand} to define a robust macro
% that does not break if it is used in moving arguments.
% Sometimes a macro is already defined, but not robust. For
% example, \cs{(} and \cs{)} are not robust, inside \cs{section}
% the user must use \cs{protect} explicitly. This could be
% avoided by making \cs{(} and \cs{)} robust.
%
% \begin{declcs}{MakeRobustCommand}\M{cmd}
% \end{declcs}
% \cs{MakeRobustCommand} redefines the macro \meta{cmd}
% by using \cs{DeclareRobustCommand} and the existing definition
% of the macro \meta{cmd}.
% \begin{itemize}
% \item It is an error if \meta{cmd} is undefined. If you want to
%   define a robust command, then you can use \cs{DeclareRobustCommand}
%   directly.
% \item If the macro has previously been
%   defined by \cs{DeclareRobustCommand} then the redefinition of
%   \cs{MakeRobustCommand} is omitted, because the macro is already robust.
%   Only an information entry is written to the \xfile{.log} file.
%   Thus you do not get a warning or an error if the macro is already
%   robust because of an updated LaTeX or package that defines the macro.
% \item Two macros are defined for a macro, defined
%   by \cs{DeclareRobustCommand}. Example:
%   \begin{quote}
%   |\DeclareRobustCommand{\foobar}{definition text}|
%   \end{quote}
%   Then the macro ``\cs{foobar}'' contains the protection code
%   and, depending on the protection mode,
%   calls the internal macro ``\cs{foobar }''. Notice the space
%   at the end of the macro name.
%   This internal macro ``\cs{foobar }'' now contains the definition
%   ``|definition text|'', given in \cs{DeclareRobustCommand}.
%
%   Sometimes it can happen, that the internal macro already exists.
%   This can be caused by a previous \cs{DeclareRobustCommand} followed
%   by \cs{renewcommand}. Then the redefinition by \cs{MakeRobustCommand}
%   would be safe.
%
%   However, it can also be possible that the macro is already robust,
%   using the internal macro, but with a different protection code.
%   The redefinition by \cs{MakeRobustCommand} would then generate
%   an infinite loop.
%
%   Therefore \cs{MakeRobustCommand} raises an error message,
%   if the internal macro (with space at the end) already exists.
% \end{itemize}
%
% \subsection{Example}
%
%    \begin{macrocode}
%<*example>
\documentclass{article}
\usepackage{makerobust}
\MakeRobustCommand\(
\MakeRobustCommand\)
\pagestyle{headings}
\begin{document}
\tableofcontents
\section{Einstein: \(E=mc^2\)}
\newpage
Second page.
\end{document}
%</example>
%    \end{macrocode}
%
%
% \StopEventually{
% }
%
% \section{Implementation}
%
%    \begin{macrocode}
%<*package>
\NeedsTeXFormat{LaTeX2e}
\ProvidesPackage{makerobust}%
  [2006/03/18 v1.0 Make existing macro robust (HO)]%
%    \end{macrocode}
%
%    \begin{macrocode}
\def\MakeRobustCommand#1{%
  \begingroup
  \@ifundefined{\expandafter\@gobble\string#1}{%
    \endgroup
    \PackageError{makerobust}{%
      Macro \string`\string#1\string' is not defined%
    }\@ehc
  }{%
    \global\let\MR@gtemp#1%
    \let#1\@undefined
    \expandafter\let\expandafter\MR@temp
        \csname\expandafter\@gobble\string#1 \endcsname
    \DeclareRobustCommand#1{}%
    \ifx#1\MR@gtemp
      \endgroup
      \PackageInfo{makerobust}{%
        \string`\string#1\string' is already robust%
      }%
    \else
      \@ifundefined{MR@temp}{%
        \global\let\MR@gtemp#1%
        \endgroup
        \expandafter\let\csname\expandafter\@gobble\string#1 \endcsname#1%
        \let#1\MR@gtemp
      }{%
        \endgroup
        \PackageError{makerobust}{%
          Internal macro \string`\string#1 \string' already exists%
        }\@ehc
      }%
    \fi
  }%
}
%    \end{macrocode}
%
%    \begin{macrocode}
%</package>
%    \end{macrocode}
%
% \section{Installation}
%
% \subsection{Download}
%
% \paragraph{Package.} This package is available on
% CTAN\footnote{\url{ftp://ftp.ctan.org/tex-archive/}}:
% \begin{description}
% \item[\CTAN{macros/latex/contrib/oberdiek/makerobust.dtx}] The source file.
% \item[\CTAN{macros/latex/contrib/oberdiek/makerobust.pdf}] Documentation.
% \end{description}
%
%
% \paragraph{Bundle.} All the packages of the bundle `oberdiek'
% are also available in a TDS compliant ZIP archive. There
% the packages are already unpacked and the documentation files
% are generated. The files and directories obey the TDS standard.
% \begin{description}
% \item[\CTAN{install/macros/latex/contrib/oberdiek.tds.zip}]
% \end{description}
% \emph{TDS} refers to the standard ``A Directory Structure
% for \TeX\ Files'' (\CTAN{tds/tds.pdf}). Directories
% with \xfile{texmf} in their name are usually organized this way.
%
% \subsection{Bundle installation}
%
% \paragraph{Unpacking.} Unpack the \xfile{oberdiek.tds.zip} in the
% TDS tree (also known as \xfile{texmf} tree) of your choice.
% Example (linux):
% \begin{quote}
%   |unzip oberdiek.tds.zip -d ~/texmf|
% \end{quote}
%
% \paragraph{Script installation.}
% Check the directory \xfile{TDS:scripts/oberdiek/} for
% scripts that need further installation steps.
% Package \xpackage{attachfile2} comes with the Perl script
% \xfile{pdfatfi.pl} that should be installed in such a way
% that it can be called as \texttt{pdfatfi}.
% Example (linux):
% \begin{quote}
%   |chmod +x scripts/oberdiek/pdfatfi.pl|\\
%   |cp scripts/oberdiek/pdfatfi.pl /usr/local/bin/|
% \end{quote}
%
% \subsection{Package installation}
%
% \paragraph{Unpacking.} The \xfile{.dtx} file is a self-extracting
% \docstrip\ archive. The files are extracted by running the
% \xfile{.dtx} through \plainTeX:
% \begin{quote}
%   \verb|tex makerobust.dtx|
% \end{quote}
%
% \paragraph{TDS.} Now the different files must be moved into
% the different directories in your installation TDS tree
% (also known as \xfile{texmf} tree):
% \begin{quote}
% \def\t{^^A
% \begin{tabular}{@{}>{\ttfamily}l@{ $\rightarrow$ }>{\ttfamily}l@{}}
%   makerobust.sty & tex/latex/oberdiek/makerobust.sty\\
%   makerobust.pdf & doc/latex/oberdiek/makerobust.pdf\\
%   makerobust-example.tex & doc/latex/oberdiek/makerobust-example.tex\\
%   makerobust.dtx & source/latex/oberdiek/makerobust.dtx\\
% \end{tabular}^^A
% }^^A
% \sbox0{\t}^^A
% \ifdim\wd0>\linewidth
%   \begingroup
%     \advance\linewidth by\leftmargin
%     \advance\linewidth by\rightmargin
%   \edef\x{\endgroup
%     \def\noexpand\lw{\the\linewidth}^^A
%   }\x
%   \def\lwbox{^^A
%     \leavevmode
%     \hbox to \linewidth{^^A
%       \kern-\leftmargin\relax
%       \hss
%       \usebox0
%       \hss
%       \kern-\rightmargin\relax
%     }^^A
%   }^^A
%   \ifdim\wd0>\lw
%     \sbox0{\small\t}^^A
%     \ifdim\wd0>\linewidth
%       \ifdim\wd0>\lw
%         \sbox0{\footnotesize\t}^^A
%         \ifdim\wd0>\linewidth
%           \ifdim\wd0>\lw
%             \sbox0{\scriptsize\t}^^A
%             \ifdim\wd0>\linewidth
%               \ifdim\wd0>\lw
%                 \sbox0{\tiny\t}^^A
%                 \ifdim\wd0>\linewidth
%                   \lwbox
%                 \else
%                   \usebox0
%                 \fi
%               \else
%                 \lwbox
%               \fi
%             \else
%               \usebox0
%             \fi
%           \else
%             \lwbox
%           \fi
%         \else
%           \usebox0
%         \fi
%       \else
%         \lwbox
%       \fi
%     \else
%       \usebox0
%     \fi
%   \else
%     \lwbox
%   \fi
% \else
%   \usebox0
% \fi
% \end{quote}
% If you have a \xfile{docstrip.cfg} that configures and enables \docstrip's
% TDS installing feature, then some files can already be in the right
% place, see the documentation of \docstrip.
%
% \subsection{Refresh file name databases}
%
% If your \TeX~distribution
% (\teTeX, \mikTeX, \dots) relies on file name databases, you must refresh
% these. For example, \teTeX\ users run \verb|texhash| or
% \verb|mktexlsr|.
%
% \subsection{Some details for the interested}
%
% \paragraph{Attached source.}
%
% The PDF documentation on CTAN also includes the
% \xfile{.dtx} source file. It can be extracted by
% AcrobatReader 6 or higher. Another option is \textsf{pdftk},
% e.g. unpack the file into the current directory:
% \begin{quote}
%   \verb|pdftk makerobust.pdf unpack_files output .|
% \end{quote}
%
% \paragraph{Unpacking with \LaTeX.}
% The \xfile{.dtx} chooses its action depending on the format:
% \begin{description}
% \item[\plainTeX:] Run \docstrip\ and extract the files.
% \item[\LaTeX:] Generate the documentation.
% \end{description}
% If you insist on using \LaTeX\ for \docstrip\ (really,
% \docstrip\ does not need \LaTeX), then inform the autodetect routine
% about your intention:
% \begin{quote}
%   \verb|latex \let\install=y% \iffalse meta-comment
%
% File: makerobust.dtx
% Version: 2006/03/18 v1.0
% Info: Make existing macro robust
%
% Copyright (C) 2006 by
%    Heiko Oberdiek <heiko.oberdiek at googlemail.com>
%
% This work may be distributed and/or modified under the
% conditions of the LaTeX Project Public License, either
% version 1.3c of this license or (at your option) any later
% version. This version of this license is in
%    http://www.latex-project.org/lppl/lppl-1-3c.txt
% and the latest version of this license is in
%    http://www.latex-project.org/lppl.txt
% and version 1.3 or later is part of all distributions of
% LaTeX version 2005/12/01 or later.
%
% This work has the LPPL maintenance status "maintained".
%
% This Current Maintainer of this work is Heiko Oberdiek.
%
% This work consists of the main source file makerobust.dtx
% and the derived files
%    makerobust.sty, makerobust.pdf, makerobust.ins, makerobust.drv,
%    makerobust-example.tex.
%
% Distribution:
%    CTAN:macros/latex/contrib/oberdiek/makerobust.dtx
%    CTAN:macros/latex/contrib/oberdiek/makerobust.pdf
%
% Unpacking:
%    (a) If makerobust.ins is present:
%           tex makerobust.ins
%    (b) Without makerobust.ins:
%           tex makerobust.dtx
%    (c) If you insist on using LaTeX
%           latex \let\install=y\input{makerobust.dtx}
%        (quote the arguments according to the demands of your shell)
%
% Documentation:
%    (a) If makerobust.drv is present:
%           latex makerobust.drv
%    (b) Without makerobust.drv:
%           latex makerobust.dtx; ...
%    The class ltxdoc loads the configuration file ltxdoc.cfg
%    if available. Here you can specify further options, e.g.
%    use A4 as paper format:
%       \PassOptionsToClass{a4paper}{article}
%
%    Programm calls to get the documentation (example):
%       pdflatex makerobust.dtx
%       makeindex -s gind.ist makerobust.idx
%       pdflatex makerobust.dtx
%       makeindex -s gind.ist makerobust.idx
%       pdflatex makerobust.dtx
%
% Installation:
%    TDS:tex/latex/oberdiek/makerobust.sty
%    TDS:doc/latex/oberdiek/makerobust.pdf
%    TDS:doc/latex/oberdiek/makerobust-example.tex
%    TDS:source/latex/oberdiek/makerobust.dtx
%
%<*ignore>
\begingroup
  \catcode123=1 %
  \catcode125=2 %
  \def\x{LaTeX2e}%
\expandafter\endgroup
\ifcase 0\ifx\install y1\fi\expandafter
         \ifx\csname processbatchFile\endcsname\relax\else1\fi
         \ifx\fmtname\x\else 1\fi\relax
\else\csname fi\endcsname
%</ignore>
%<*install>
\input docstrip.tex
\Msg{************************************************************************}
\Msg{* Installation}
\Msg{* Package: makerobust 2006/03/18 v1.0 Make existing macro robust (HO)}
\Msg{************************************************************************}

\keepsilent
\askforoverwritefalse

\let\MetaPrefix\relax
\preamble

This is a generated file.

Project: makerobust
Version: 2006/03/18 v1.0

Copyright (C) 2006 by
   Heiko Oberdiek <heiko.oberdiek at googlemail.com>

This work may be distributed and/or modified under the
conditions of the LaTeX Project Public License, either
version 1.3c of this license or (at your option) any later
version. This version of this license is in
   http://www.latex-project.org/lppl/lppl-1-3c.txt
and the latest version of this license is in
   http://www.latex-project.org/lppl.txt
and version 1.3 or later is part of all distributions of
LaTeX version 2005/12/01 or later.

This work has the LPPL maintenance status "maintained".

This Current Maintainer of this work is Heiko Oberdiek.

This work consists of the main source file makerobust.dtx
and the derived files
   makerobust.sty, makerobust.pdf, makerobust.ins, makerobust.drv,
   makerobust-example.tex.

\endpreamble
\let\MetaPrefix\DoubleperCent

\generate{%
  \file{makerobust.ins}{\from{makerobust.dtx}{install}}%
  \file{makerobust.drv}{\from{makerobust.dtx}{driver}}%
  \usedir{tex/latex/oberdiek}%
  \file{makerobust.sty}{\from{makerobust.dtx}{package}}%
  \usedir{doc/latex/oberdiek}%
  \file{makerobust-example.tex}{\from{makerobust.dtx}{example}}%
  \nopreamble
  \nopostamble
  \usedir{source/latex/oberdiek/catalogue}%
  \file{makerobust.xml}{\from{makerobust.dtx}{catalogue}}%
}

\catcode32=13\relax% active space
\let =\space%
\Msg{************************************************************************}
\Msg{*}
\Msg{* To finish the installation you have to move the following}
\Msg{* file into a directory searched by TeX:}
\Msg{*}
\Msg{*     makerobust.sty}
\Msg{*}
\Msg{* To produce the documentation run the file `makerobust.drv'}
\Msg{* through LaTeX.}
\Msg{*}
\Msg{* Happy TeXing!}
\Msg{*}
\Msg{************************************************************************}

\endbatchfile
%</install>
%<*ignore>
\fi
%</ignore>
%<*driver>
\NeedsTeXFormat{LaTeX2e}
\ProvidesFile{makerobust.drv}%
  [2006/03/18 v1.0 Make existing macro robust (HO)]%
\documentclass{ltxdoc}
\usepackage{holtxdoc}[2011/11/22]
\begin{document}
  \DocInput{makerobust.dtx}%
\end{document}
%</driver>
% \fi
%
% \CheckSum{59}
%
% \CharacterTable
%  {Upper-case    \A\B\C\D\E\F\G\H\I\J\K\L\M\N\O\P\Q\R\S\T\U\V\W\X\Y\Z
%   Lower-case    \a\b\c\d\e\f\g\h\i\j\k\l\m\n\o\p\q\r\s\t\u\v\w\x\y\z
%   Digits        \0\1\2\3\4\5\6\7\8\9
%   Exclamation   \!     Double quote  \"     Hash (number) \#
%   Dollar        \$     Percent       \%     Ampersand     \&
%   Acute accent  \'     Left paren    \(     Right paren   \)
%   Asterisk      \*     Plus          \+     Comma         \,
%   Minus         \-     Point         \.     Solidus       \/
%   Colon         \:     Semicolon     \;     Less than     \<
%   Equals        \=     Greater than  \>     Question mark \?
%   Commercial at \@     Left bracket  \[     Backslash     \\
%   Right bracket \]     Circumflex    \^     Underscore    \_
%   Grave accent  \`     Left brace    \{     Vertical bar  \|
%   Right brace   \}     Tilde         \~}
%
% \GetFileInfo{makerobust.drv}
%
% \title{The \xpackage{makerobust} package}
% \date{2006/03/18 v1.0}
% \author{Heiko Oberdiek\\\xemail{heiko.oberdiek at googlemail.com}}
%
% \maketitle
%
% \begin{abstract}
% Package \xpackage{makerobust} provides \cs{MakeRobustCommand}
% that converts an existing macro to a robust one.
% \end{abstract}
%
% \tableofcontents
%
% \section{User interface}
%
% \LaTeX\ offers \cs{DeclareRobustCommand} to define a robust macro
% that does not break if it is used in moving arguments.
% Sometimes a macro is already defined, but not robust. For
% example, \cs{(} and \cs{)} are not robust, inside \cs{section}
% the user must use \cs{protect} explicitly. This could be
% avoided by making \cs{(} and \cs{)} robust.
%
% \begin{declcs}{MakeRobustCommand}\M{cmd}
% \end{declcs}
% \cs{MakeRobustCommand} redefines the macro \meta{cmd}
% by using \cs{DeclareRobustCommand} and the existing definition
% of the macro \meta{cmd}.
% \begin{itemize}
% \item It is an error if \meta{cmd} is undefined. If you want to
%   define a robust command, then you can use \cs{DeclareRobustCommand}
%   directly.
% \item If the macro has previously been
%   defined by \cs{DeclareRobustCommand} then the redefinition of
%   \cs{MakeRobustCommand} is omitted, because the macro is already robust.
%   Only an information entry is written to the \xfile{.log} file.
%   Thus you do not get a warning or an error if the macro is already
%   robust because of an updated LaTeX or package that defines the macro.
% \item Two macros are defined for a macro, defined
%   by \cs{DeclareRobustCommand}. Example:
%   \begin{quote}
%   |\DeclareRobustCommand{\foobar}{definition text}|
%   \end{quote}
%   Then the macro ``\cs{foobar}'' contains the protection code
%   and, depending on the protection mode,
%   calls the internal macro ``\cs{foobar }''. Notice the space
%   at the end of the macro name.
%   This internal macro ``\cs{foobar }'' now contains the definition
%   ``|definition text|'', given in \cs{DeclareRobustCommand}.
%
%   Sometimes it can happen, that the internal macro already exists.
%   This can be caused by a previous \cs{DeclareRobustCommand} followed
%   by \cs{renewcommand}. Then the redefinition by \cs{MakeRobustCommand}
%   would be safe.
%
%   However, it can also be possible that the macro is already robust,
%   using the internal macro, but with a different protection code.
%   The redefinition by \cs{MakeRobustCommand} would then generate
%   an infinite loop.
%
%   Therefore \cs{MakeRobustCommand} raises an error message,
%   if the internal macro (with space at the end) already exists.
% \end{itemize}
%
% \subsection{Example}
%
%    \begin{macrocode}
%<*example>
\documentclass{article}
\usepackage{makerobust}
\MakeRobustCommand\(
\MakeRobustCommand\)
\pagestyle{headings}
\begin{document}
\tableofcontents
\section{Einstein: \(E=mc^2\)}
\newpage
Second page.
\end{document}
%</example>
%    \end{macrocode}
%
%
% \StopEventually{
% }
%
% \section{Implementation}
%
%    \begin{macrocode}
%<*package>
\NeedsTeXFormat{LaTeX2e}
\ProvidesPackage{makerobust}%
  [2006/03/18 v1.0 Make existing macro robust (HO)]%
%    \end{macrocode}
%
%    \begin{macrocode}
\def\MakeRobustCommand#1{%
  \begingroup
  \@ifundefined{\expandafter\@gobble\string#1}{%
    \endgroup
    \PackageError{makerobust}{%
      Macro \string`\string#1\string' is not defined%
    }\@ehc
  }{%
    \global\let\MR@gtemp#1%
    \let#1\@undefined
    \expandafter\let\expandafter\MR@temp
        \csname\expandafter\@gobble\string#1 \endcsname
    \DeclareRobustCommand#1{}%
    \ifx#1\MR@gtemp
      \endgroup
      \PackageInfo{makerobust}{%
        \string`\string#1\string' is already robust%
      }%
    \else
      \@ifundefined{MR@temp}{%
        \global\let\MR@gtemp#1%
        \endgroup
        \expandafter\let\csname\expandafter\@gobble\string#1 \endcsname#1%
        \let#1\MR@gtemp
      }{%
        \endgroup
        \PackageError{makerobust}{%
          Internal macro \string`\string#1 \string' already exists%
        }\@ehc
      }%
    \fi
  }%
}
%    \end{macrocode}
%
%    \begin{macrocode}
%</package>
%    \end{macrocode}
%
% \section{Installation}
%
% \subsection{Download}
%
% \paragraph{Package.} This package is available on
% CTAN\footnote{\url{ftp://ftp.ctan.org/tex-archive/}}:
% \begin{description}
% \item[\CTAN{macros/latex/contrib/oberdiek/makerobust.dtx}] The source file.
% \item[\CTAN{macros/latex/contrib/oberdiek/makerobust.pdf}] Documentation.
% \end{description}
%
%
% \paragraph{Bundle.} All the packages of the bundle `oberdiek'
% are also available in a TDS compliant ZIP archive. There
% the packages are already unpacked and the documentation files
% are generated. The files and directories obey the TDS standard.
% \begin{description}
% \item[\CTAN{install/macros/latex/contrib/oberdiek.tds.zip}]
% \end{description}
% \emph{TDS} refers to the standard ``A Directory Structure
% for \TeX\ Files'' (\CTAN{tds/tds.pdf}). Directories
% with \xfile{texmf} in their name are usually organized this way.
%
% \subsection{Bundle installation}
%
% \paragraph{Unpacking.} Unpack the \xfile{oberdiek.tds.zip} in the
% TDS tree (also known as \xfile{texmf} tree) of your choice.
% Example (linux):
% \begin{quote}
%   |unzip oberdiek.tds.zip -d ~/texmf|
% \end{quote}
%
% \paragraph{Script installation.}
% Check the directory \xfile{TDS:scripts/oberdiek/} for
% scripts that need further installation steps.
% Package \xpackage{attachfile2} comes with the Perl script
% \xfile{pdfatfi.pl} that should be installed in such a way
% that it can be called as \texttt{pdfatfi}.
% Example (linux):
% \begin{quote}
%   |chmod +x scripts/oberdiek/pdfatfi.pl|\\
%   |cp scripts/oberdiek/pdfatfi.pl /usr/local/bin/|
% \end{quote}
%
% \subsection{Package installation}
%
% \paragraph{Unpacking.} The \xfile{.dtx} file is a self-extracting
% \docstrip\ archive. The files are extracted by running the
% \xfile{.dtx} through \plainTeX:
% \begin{quote}
%   \verb|tex makerobust.dtx|
% \end{quote}
%
% \paragraph{TDS.} Now the different files must be moved into
% the different directories in your installation TDS tree
% (also known as \xfile{texmf} tree):
% \begin{quote}
% \def\t{^^A
% \begin{tabular}{@{}>{\ttfamily}l@{ $\rightarrow$ }>{\ttfamily}l@{}}
%   makerobust.sty & tex/latex/oberdiek/makerobust.sty\\
%   makerobust.pdf & doc/latex/oberdiek/makerobust.pdf\\
%   makerobust-example.tex & doc/latex/oberdiek/makerobust-example.tex\\
%   makerobust.dtx & source/latex/oberdiek/makerobust.dtx\\
% \end{tabular}^^A
% }^^A
% \sbox0{\t}^^A
% \ifdim\wd0>\linewidth
%   \begingroup
%     \advance\linewidth by\leftmargin
%     \advance\linewidth by\rightmargin
%   \edef\x{\endgroup
%     \def\noexpand\lw{\the\linewidth}^^A
%   }\x
%   \def\lwbox{^^A
%     \leavevmode
%     \hbox to \linewidth{^^A
%       \kern-\leftmargin\relax
%       \hss
%       \usebox0
%       \hss
%       \kern-\rightmargin\relax
%     }^^A
%   }^^A
%   \ifdim\wd0>\lw
%     \sbox0{\small\t}^^A
%     \ifdim\wd0>\linewidth
%       \ifdim\wd0>\lw
%         \sbox0{\footnotesize\t}^^A
%         \ifdim\wd0>\linewidth
%           \ifdim\wd0>\lw
%             \sbox0{\scriptsize\t}^^A
%             \ifdim\wd0>\linewidth
%               \ifdim\wd0>\lw
%                 \sbox0{\tiny\t}^^A
%                 \ifdim\wd0>\linewidth
%                   \lwbox
%                 \else
%                   \usebox0
%                 \fi
%               \else
%                 \lwbox
%               \fi
%             \else
%               \usebox0
%             \fi
%           \else
%             \lwbox
%           \fi
%         \else
%           \usebox0
%         \fi
%       \else
%         \lwbox
%       \fi
%     \else
%       \usebox0
%     \fi
%   \else
%     \lwbox
%   \fi
% \else
%   \usebox0
% \fi
% \end{quote}
% If you have a \xfile{docstrip.cfg} that configures and enables \docstrip's
% TDS installing feature, then some files can already be in the right
% place, see the documentation of \docstrip.
%
% \subsection{Refresh file name databases}
%
% If your \TeX~distribution
% (\teTeX, \mikTeX, \dots) relies on file name databases, you must refresh
% these. For example, \teTeX\ users run \verb|texhash| or
% \verb|mktexlsr|.
%
% \subsection{Some details for the interested}
%
% \paragraph{Attached source.}
%
% The PDF documentation on CTAN also includes the
% \xfile{.dtx} source file. It can be extracted by
% AcrobatReader 6 or higher. Another option is \textsf{pdftk},
% e.g. unpack the file into the current directory:
% \begin{quote}
%   \verb|pdftk makerobust.pdf unpack_files output .|
% \end{quote}
%
% \paragraph{Unpacking with \LaTeX.}
% The \xfile{.dtx} chooses its action depending on the format:
% \begin{description}
% \item[\plainTeX:] Run \docstrip\ and extract the files.
% \item[\LaTeX:] Generate the documentation.
% \end{description}
% If you insist on using \LaTeX\ for \docstrip\ (really,
% \docstrip\ does not need \LaTeX), then inform the autodetect routine
% about your intention:
% \begin{quote}
%   \verb|latex \let\install=y\input{makerobust.dtx}|
% \end{quote}
% Do not forget to quote the argument according to the demands
% of your shell.
%
% \paragraph{Generating the documentation.}
% You can use both the \xfile{.dtx} or the \xfile{.drv} to generate
% the documentation. The process can be configured by the
% configuration file \xfile{ltxdoc.cfg}. For instance, put this
% line into this file, if you want to have A4 as paper format:
% \begin{quote}
%   \verb|\PassOptionsToClass{a4paper}{article}|
% \end{quote}
% An example follows how to generate the
% documentation with pdf\LaTeX:
% \begin{quote}
%\begin{verbatim}
%pdflatex makerobust.dtx
%makeindex -s gind.ist makerobust.idx
%pdflatex makerobust.dtx
%makeindex -s gind.ist makerobust.idx
%pdflatex makerobust.dtx
%\end{verbatim}
% \end{quote}
%
% \section{Catalogue}
%
% The following XML file can be used as source for the
% \href{http://mirror.ctan.org/help/Catalogue/catalogue.html}{\TeX\ Catalogue}.
% The elements \texttt{caption} and \texttt{description} are imported
% from the original XML file from the Catalogue.
% The name of the XML file in the Catalogue is \xfile{makerobust.xml}.
%    \begin{macrocode}
%<*catalogue>
<?xml version='1.0' encoding='us-ascii'?>
<!DOCTYPE entry SYSTEM 'catalogue.dtd'>
<entry datestamp='$Date$' modifier='$Author$' id='makerobust'>
  <name>makerobust</name>
  <caption>Making a macro robust.</caption>
  <authorref id='auth:oberdiek'/>
  <copyright owner='Heiko Oberdiek' year='2006'/>
  <license type='lppl1.3'/>
  <version number='1.0'/>
  <description>
    This package provides the command MakeRobustCommand
    that converts an existing macro to a robust one.
    <p/>
    The package is part of the <xref refid='oberdiek'>oberdiek</xref>
    bundle.
  </description>
  <documentation details='Package documentation'
      href='ctan:/macros/latex/contrib/oberdiek/makerobust.pdf'/>
  <ctan file='true' path='/macros/latex/contrib/oberdiek/makerobust.dtx'/>
  <miktex location='oberdiek'/>
  <texlive location='oberdiek'/>
  <install path='/macros/latex/contrib/oberdiek/oberdiek.tds.zip'/>
</entry>
%</catalogue>
%    \end{macrocode}
%
% \begin{History}
%   \begin{Version}{2006/03/18 v1.0}
%   \item
%     First version.
%   \end{Version}
% \end{History}
%
% \PrintIndex
%
% \Finale
\endinput
|
% \end{quote}
% Do not forget to quote the argument according to the demands
% of your shell.
%
% \paragraph{Generating the documentation.}
% You can use both the \xfile{.dtx} or the \xfile{.drv} to generate
% the documentation. The process can be configured by the
% configuration file \xfile{ltxdoc.cfg}. For instance, put this
% line into this file, if you want to have A4 as paper format:
% \begin{quote}
%   \verb|\PassOptionsToClass{a4paper}{article}|
% \end{quote}
% An example follows how to generate the
% documentation with pdf\LaTeX:
% \begin{quote}
%\begin{verbatim}
%pdflatex makerobust.dtx
%makeindex -s gind.ist makerobust.idx
%pdflatex makerobust.dtx
%makeindex -s gind.ist makerobust.idx
%pdflatex makerobust.dtx
%\end{verbatim}
% \end{quote}
%
% \section{Catalogue}
%
% The following XML file can be used as source for the
% \href{http://mirror.ctan.org/help/Catalogue/catalogue.html}{\TeX\ Catalogue}.
% The elements \texttt{caption} and \texttt{description} are imported
% from the original XML file from the Catalogue.
% The name of the XML file in the Catalogue is \xfile{makerobust.xml}.
%    \begin{macrocode}
%<*catalogue>
<?xml version='1.0' encoding='us-ascii'?>
<!DOCTYPE entry SYSTEM 'catalogue.dtd'>
<entry datestamp='$Date$' modifier='$Author$' id='makerobust'>
  <name>makerobust</name>
  <caption>Making a macro robust.</caption>
  <authorref id='auth:oberdiek'/>
  <copyright owner='Heiko Oberdiek' year='2006'/>
  <license type='lppl1.3'/>
  <version number='1.0'/>
  <description>
    This package provides the command MakeRobustCommand
    that converts an existing macro to a robust one.
    <p/>
    The package is part of the <xref refid='oberdiek'>oberdiek</xref>
    bundle.
  </description>
  <documentation details='Package documentation'
      href='ctan:/macros/latex/contrib/oberdiek/makerobust.pdf'/>
  <ctan file='true' path='/macros/latex/contrib/oberdiek/makerobust.dtx'/>
  <miktex location='oberdiek'/>
  <texlive location='oberdiek'/>
  <install path='/macros/latex/contrib/oberdiek/oberdiek.tds.zip'/>
</entry>
%</catalogue>
%    \end{macrocode}
%
% \begin{History}
%   \begin{Version}{2006/03/18 v1.0}
%   \item
%     First version.
%   \end{Version}
% \end{History}
%
% \PrintIndex
%
% \Finale
\endinput

%        (quote the arguments according to the demands of your shell)
%
% Documentation:
%    (a) If makerobust.drv is present:
%           latex makerobust.drv
%    (b) Without makerobust.drv:
%           latex makerobust.dtx; ...
%    The class ltxdoc loads the configuration file ltxdoc.cfg
%    if available. Here you can specify further options, e.g.
%    use A4 as paper format:
%       \PassOptionsToClass{a4paper}{article}
%
%    Programm calls to get the documentation (example):
%       pdflatex makerobust.dtx
%       makeindex -s gind.ist makerobust.idx
%       pdflatex makerobust.dtx
%       makeindex -s gind.ist makerobust.idx
%       pdflatex makerobust.dtx
%
% Installation:
%    TDS:tex/latex/oberdiek/makerobust.sty
%    TDS:doc/latex/oberdiek/makerobust.pdf
%    TDS:doc/latex/oberdiek/makerobust-example.tex
%    TDS:source/latex/oberdiek/makerobust.dtx
%
%<*ignore>
\begingroup
  \catcode123=1 %
  \catcode125=2 %
  \def\x{LaTeX2e}%
\expandafter\endgroup
\ifcase 0\ifx\install y1\fi\expandafter
         \ifx\csname processbatchFile\endcsname\relax\else1\fi
         \ifx\fmtname\x\else 1\fi\relax
\else\csname fi\endcsname
%</ignore>
%<*install>
\input docstrip.tex
\Msg{************************************************************************}
\Msg{* Installation}
\Msg{* Package: makerobust 2006/03/18 v1.0 Make existing macro robust (HO)}
\Msg{************************************************************************}

\keepsilent
\askforoverwritefalse

\let\MetaPrefix\relax
\preamble

This is a generated file.

Project: makerobust
Version: 2006/03/18 v1.0

Copyright (C) 2006 by
   Heiko Oberdiek <heiko.oberdiek at googlemail.com>

This work may be distributed and/or modified under the
conditions of the LaTeX Project Public License, either
version 1.3c of this license or (at your option) any later
version. This version of this license is in
   http://www.latex-project.org/lppl/lppl-1-3c.txt
and the latest version of this license is in
   http://www.latex-project.org/lppl.txt
and version 1.3 or later is part of all distributions of
LaTeX version 2005/12/01 or later.

This work has the LPPL maintenance status "maintained".

This Current Maintainer of this work is Heiko Oberdiek.

This work consists of the main source file makerobust.dtx
and the derived files
   makerobust.sty, makerobust.pdf, makerobust.ins, makerobust.drv,
   makerobust-example.tex.

\endpreamble
\let\MetaPrefix\DoubleperCent

\generate{%
  \file{makerobust.ins}{\from{makerobust.dtx}{install}}%
  \file{makerobust.drv}{\from{makerobust.dtx}{driver}}%
  \usedir{tex/latex/oberdiek}%
  \file{makerobust.sty}{\from{makerobust.dtx}{package}}%
  \usedir{doc/latex/oberdiek}%
  \file{makerobust-example.tex}{\from{makerobust.dtx}{example}}%
  \nopreamble
  \nopostamble
  \usedir{source/latex/oberdiek/catalogue}%
  \file{makerobust.xml}{\from{makerobust.dtx}{catalogue}}%
}

\catcode32=13\relax% active space
\let =\space%
\Msg{************************************************************************}
\Msg{*}
\Msg{* To finish the installation you have to move the following}
\Msg{* file into a directory searched by TeX:}
\Msg{*}
\Msg{*     makerobust.sty}
\Msg{*}
\Msg{* To produce the documentation run the file `makerobust.drv'}
\Msg{* through LaTeX.}
\Msg{*}
\Msg{* Happy TeXing!}
\Msg{*}
\Msg{************************************************************************}

\endbatchfile
%</install>
%<*ignore>
\fi
%</ignore>
%<*driver>
\NeedsTeXFormat{LaTeX2e}
\ProvidesFile{makerobust.drv}%
  [2006/03/18 v1.0 Make existing macro robust (HO)]%
\documentclass{ltxdoc}
\usepackage{holtxdoc}[2011/11/22]
\begin{document}
  \DocInput{makerobust.dtx}%
\end{document}
%</driver>
% \fi
%
% \CheckSum{59}
%
% \CharacterTable
%  {Upper-case    \A\B\C\D\E\F\G\H\I\J\K\L\M\N\O\P\Q\R\S\T\U\V\W\X\Y\Z
%   Lower-case    \a\b\c\d\e\f\g\h\i\j\k\l\m\n\o\p\q\r\s\t\u\v\w\x\y\z
%   Digits        \0\1\2\3\4\5\6\7\8\9
%   Exclamation   \!     Double quote  \"     Hash (number) \#
%   Dollar        \$     Percent       \%     Ampersand     \&
%   Acute accent  \'     Left paren    \(     Right paren   \)
%   Asterisk      \*     Plus          \+     Comma         \,
%   Minus         \-     Point         \.     Solidus       \/
%   Colon         \:     Semicolon     \;     Less than     \<
%   Equals        \=     Greater than  \>     Question mark \?
%   Commercial at \@     Left bracket  \[     Backslash     \\
%   Right bracket \]     Circumflex    \^     Underscore    \_
%   Grave accent  \`     Left brace    \{     Vertical bar  \|
%   Right brace   \}     Tilde         \~}
%
% \GetFileInfo{makerobust.drv}
%
% \title{The \xpackage{makerobust} package}
% \date{2006/03/18 v1.0}
% \author{Heiko Oberdiek\\\xemail{heiko.oberdiek at googlemail.com}}
%
% \maketitle
%
% \begin{abstract}
% Package \xpackage{makerobust} provides \cs{MakeRobustCommand}
% that converts an existing macro to a robust one.
% \end{abstract}
%
% \tableofcontents
%
% \section{User interface}
%
% \LaTeX\ offers \cs{DeclareRobustCommand} to define a robust macro
% that does not break if it is used in moving arguments.
% Sometimes a macro is already defined, but not robust. For
% example, \cs{(} and \cs{)} are not robust, inside \cs{section}
% the user must use \cs{protect} explicitly. This could be
% avoided by making \cs{(} and \cs{)} robust.
%
% \begin{declcs}{MakeRobustCommand}\M{cmd}
% \end{declcs}
% \cs{MakeRobustCommand} redefines the macro \meta{cmd}
% by using \cs{DeclareRobustCommand} and the existing definition
% of the macro \meta{cmd}.
% \begin{itemize}
% \item It is an error if \meta{cmd} is undefined. If you want to
%   define a robust command, then you can use \cs{DeclareRobustCommand}
%   directly.
% \item If the macro has previously been
%   defined by \cs{DeclareRobustCommand} then the redefinition of
%   \cs{MakeRobustCommand} is omitted, because the macro is already robust.
%   Only an information entry is written to the \xfile{.log} file.
%   Thus you do not get a warning or an error if the macro is already
%   robust because of an updated LaTeX or package that defines the macro.
% \item Two macros are defined for a macro, defined
%   by \cs{DeclareRobustCommand}. Example:
%   \begin{quote}
%   |\DeclareRobustCommand{\foobar}{definition text}|
%   \end{quote}
%   Then the macro ``\cs{foobar}'' contains the protection code
%   and, depending on the protection mode,
%   calls the internal macro ``\cs{foobar }''. Notice the space
%   at the end of the macro name.
%   This internal macro ``\cs{foobar }'' now contains the definition
%   ``|definition text|'', given in \cs{DeclareRobustCommand}.
%
%   Sometimes it can happen, that the internal macro already exists.
%   This can be caused by a previous \cs{DeclareRobustCommand} followed
%   by \cs{renewcommand}. Then the redefinition by \cs{MakeRobustCommand}
%   would be safe.
%
%   However, it can also be possible that the macro is already robust,
%   using the internal macro, but with a different protection code.
%   The redefinition by \cs{MakeRobustCommand} would then generate
%   an infinite loop.
%
%   Therefore \cs{MakeRobustCommand} raises an error message,
%   if the internal macro (with space at the end) already exists.
% \end{itemize}
%
% \subsection{Example}
%
%    \begin{macrocode}
%<*example>
\documentclass{article}
\usepackage{makerobust}
\MakeRobustCommand\(
\MakeRobustCommand\)
\pagestyle{headings}
\begin{document}
\tableofcontents
\section{Einstein: \(E=mc^2\)}
\newpage
Second page.
\end{document}
%</example>
%    \end{macrocode}
%
%
% \StopEventually{
% }
%
% \section{Implementation}
%
%    \begin{macrocode}
%<*package>
\NeedsTeXFormat{LaTeX2e}
\ProvidesPackage{makerobust}%
  [2006/03/18 v1.0 Make existing macro robust (HO)]%
%    \end{macrocode}
%
%    \begin{macrocode}
\def\MakeRobustCommand#1{%
  \begingroup
  \@ifundefined{\expandafter\@gobble\string#1}{%
    \endgroup
    \PackageError{makerobust}{%
      Macro \string`\string#1\string' is not defined%
    }\@ehc
  }{%
    \global\let\MR@gtemp#1%
    \let#1\@undefined
    \expandafter\let\expandafter\MR@temp
        \csname\expandafter\@gobble\string#1 \endcsname
    \DeclareRobustCommand#1{}%
    \ifx#1\MR@gtemp
      \endgroup
      \PackageInfo{makerobust}{%
        \string`\string#1\string' is already robust%
      }%
    \else
      \@ifundefined{MR@temp}{%
        \global\let\MR@gtemp#1%
        \endgroup
        \expandafter\let\csname\expandafter\@gobble\string#1 \endcsname#1%
        \let#1\MR@gtemp
      }{%
        \endgroup
        \PackageError{makerobust}{%
          Internal macro \string`\string#1 \string' already exists%
        }\@ehc
      }%
    \fi
  }%
}
%    \end{macrocode}
%
%    \begin{macrocode}
%</package>
%    \end{macrocode}
%
% \section{Installation}
%
% \subsection{Download}
%
% \paragraph{Package.} This package is available on
% CTAN\footnote{\url{ftp://ftp.ctan.org/tex-archive/}}:
% \begin{description}
% \item[\CTAN{macros/latex/contrib/oberdiek/makerobust.dtx}] The source file.
% \item[\CTAN{macros/latex/contrib/oberdiek/makerobust.pdf}] Documentation.
% \end{description}
%
%
% \paragraph{Bundle.} All the packages of the bundle `oberdiek'
% are also available in a TDS compliant ZIP archive. There
% the packages are already unpacked and the documentation files
% are generated. The files and directories obey the TDS standard.
% \begin{description}
% \item[\CTAN{install/macros/latex/contrib/oberdiek.tds.zip}]
% \end{description}
% \emph{TDS} refers to the standard ``A Directory Structure
% for \TeX\ Files'' (\CTAN{tds/tds.pdf}). Directories
% with \xfile{texmf} in their name are usually organized this way.
%
% \subsection{Bundle installation}
%
% \paragraph{Unpacking.} Unpack the \xfile{oberdiek.tds.zip} in the
% TDS tree (also known as \xfile{texmf} tree) of your choice.
% Example (linux):
% \begin{quote}
%   |unzip oberdiek.tds.zip -d ~/texmf|
% \end{quote}
%
% \paragraph{Script installation.}
% Check the directory \xfile{TDS:scripts/oberdiek/} for
% scripts that need further installation steps.
% Package \xpackage{attachfile2} comes with the Perl script
% \xfile{pdfatfi.pl} that should be installed in such a way
% that it can be called as \texttt{pdfatfi}.
% Example (linux):
% \begin{quote}
%   |chmod +x scripts/oberdiek/pdfatfi.pl|\\
%   |cp scripts/oberdiek/pdfatfi.pl /usr/local/bin/|
% \end{quote}
%
% \subsection{Package installation}
%
% \paragraph{Unpacking.} The \xfile{.dtx} file is a self-extracting
% \docstrip\ archive. The files are extracted by running the
% \xfile{.dtx} through \plainTeX:
% \begin{quote}
%   \verb|tex makerobust.dtx|
% \end{quote}
%
% \paragraph{TDS.} Now the different files must be moved into
% the different directories in your installation TDS tree
% (also known as \xfile{texmf} tree):
% \begin{quote}
% \def\t{^^A
% \begin{tabular}{@{}>{\ttfamily}l@{ $\rightarrow$ }>{\ttfamily}l@{}}
%   makerobust.sty & tex/latex/oberdiek/makerobust.sty\\
%   makerobust.pdf & doc/latex/oberdiek/makerobust.pdf\\
%   makerobust-example.tex & doc/latex/oberdiek/makerobust-example.tex\\
%   makerobust.dtx & source/latex/oberdiek/makerobust.dtx\\
% \end{tabular}^^A
% }^^A
% \sbox0{\t}^^A
% \ifdim\wd0>\linewidth
%   \begingroup
%     \advance\linewidth by\leftmargin
%     \advance\linewidth by\rightmargin
%   \edef\x{\endgroup
%     \def\noexpand\lw{\the\linewidth}^^A
%   }\x
%   \def\lwbox{^^A
%     \leavevmode
%     \hbox to \linewidth{^^A
%       \kern-\leftmargin\relax
%       \hss
%       \usebox0
%       \hss
%       \kern-\rightmargin\relax
%     }^^A
%   }^^A
%   \ifdim\wd0>\lw
%     \sbox0{\small\t}^^A
%     \ifdim\wd0>\linewidth
%       \ifdim\wd0>\lw
%         \sbox0{\footnotesize\t}^^A
%         \ifdim\wd0>\linewidth
%           \ifdim\wd0>\lw
%             \sbox0{\scriptsize\t}^^A
%             \ifdim\wd0>\linewidth
%               \ifdim\wd0>\lw
%                 \sbox0{\tiny\t}^^A
%                 \ifdim\wd0>\linewidth
%                   \lwbox
%                 \else
%                   \usebox0
%                 \fi
%               \else
%                 \lwbox
%               \fi
%             \else
%               \usebox0
%             \fi
%           \else
%             \lwbox
%           \fi
%         \else
%           \usebox0
%         \fi
%       \else
%         \lwbox
%       \fi
%     \else
%       \usebox0
%     \fi
%   \else
%     \lwbox
%   \fi
% \else
%   \usebox0
% \fi
% \end{quote}
% If you have a \xfile{docstrip.cfg} that configures and enables \docstrip's
% TDS installing feature, then some files can already be in the right
% place, see the documentation of \docstrip.
%
% \subsection{Refresh file name databases}
%
% If your \TeX~distribution
% (\teTeX, \mikTeX, \dots) relies on file name databases, you must refresh
% these. For example, \teTeX\ users run \verb|texhash| or
% \verb|mktexlsr|.
%
% \subsection{Some details for the interested}
%
% \paragraph{Attached source.}
%
% The PDF documentation on CTAN also includes the
% \xfile{.dtx} source file. It can be extracted by
% AcrobatReader 6 or higher. Another option is \textsf{pdftk},
% e.g. unpack the file into the current directory:
% \begin{quote}
%   \verb|pdftk makerobust.pdf unpack_files output .|
% \end{quote}
%
% \paragraph{Unpacking with \LaTeX.}
% The \xfile{.dtx} chooses its action depending on the format:
% \begin{description}
% \item[\plainTeX:] Run \docstrip\ and extract the files.
% \item[\LaTeX:] Generate the documentation.
% \end{description}
% If you insist on using \LaTeX\ for \docstrip\ (really,
% \docstrip\ does not need \LaTeX), then inform the autodetect routine
% about your intention:
% \begin{quote}
%   \verb|latex \let\install=y% \iffalse meta-comment
%
% File: makerobust.dtx
% Version: 2006/03/18 v1.0
% Info: Make existing macro robust
%
% Copyright (C) 2006 by
%    Heiko Oberdiek <heiko.oberdiek at googlemail.com>
%
% This work may be distributed and/or modified under the
% conditions of the LaTeX Project Public License, either
% version 1.3c of this license or (at your option) any later
% version. This version of this license is in
%    http://www.latex-project.org/lppl/lppl-1-3c.txt
% and the latest version of this license is in
%    http://www.latex-project.org/lppl.txt
% and version 1.3 or later is part of all distributions of
% LaTeX version 2005/12/01 or later.
%
% This work has the LPPL maintenance status "maintained".
%
% This Current Maintainer of this work is Heiko Oberdiek.
%
% This work consists of the main source file makerobust.dtx
% and the derived files
%    makerobust.sty, makerobust.pdf, makerobust.ins, makerobust.drv,
%    makerobust-example.tex.
%
% Distribution:
%    CTAN:macros/latex/contrib/oberdiek/makerobust.dtx
%    CTAN:macros/latex/contrib/oberdiek/makerobust.pdf
%
% Unpacking:
%    (a) If makerobust.ins is present:
%           tex makerobust.ins
%    (b) Without makerobust.ins:
%           tex makerobust.dtx
%    (c) If you insist on using LaTeX
%           latex \let\install=y% \iffalse meta-comment
%
% File: makerobust.dtx
% Version: 2006/03/18 v1.0
% Info: Make existing macro robust
%
% Copyright (C) 2006 by
%    Heiko Oberdiek <heiko.oberdiek at googlemail.com>
%
% This work may be distributed and/or modified under the
% conditions of the LaTeX Project Public License, either
% version 1.3c of this license or (at your option) any later
% version. This version of this license is in
%    http://www.latex-project.org/lppl/lppl-1-3c.txt
% and the latest version of this license is in
%    http://www.latex-project.org/lppl.txt
% and version 1.3 or later is part of all distributions of
% LaTeX version 2005/12/01 or later.
%
% This work has the LPPL maintenance status "maintained".
%
% This Current Maintainer of this work is Heiko Oberdiek.
%
% This work consists of the main source file makerobust.dtx
% and the derived files
%    makerobust.sty, makerobust.pdf, makerobust.ins, makerobust.drv,
%    makerobust-example.tex.
%
% Distribution:
%    CTAN:macros/latex/contrib/oberdiek/makerobust.dtx
%    CTAN:macros/latex/contrib/oberdiek/makerobust.pdf
%
% Unpacking:
%    (a) If makerobust.ins is present:
%           tex makerobust.ins
%    (b) Without makerobust.ins:
%           tex makerobust.dtx
%    (c) If you insist on using LaTeX
%           latex \let\install=y\input{makerobust.dtx}
%        (quote the arguments according to the demands of your shell)
%
% Documentation:
%    (a) If makerobust.drv is present:
%           latex makerobust.drv
%    (b) Without makerobust.drv:
%           latex makerobust.dtx; ...
%    The class ltxdoc loads the configuration file ltxdoc.cfg
%    if available. Here you can specify further options, e.g.
%    use A4 as paper format:
%       \PassOptionsToClass{a4paper}{article}
%
%    Programm calls to get the documentation (example):
%       pdflatex makerobust.dtx
%       makeindex -s gind.ist makerobust.idx
%       pdflatex makerobust.dtx
%       makeindex -s gind.ist makerobust.idx
%       pdflatex makerobust.dtx
%
% Installation:
%    TDS:tex/latex/oberdiek/makerobust.sty
%    TDS:doc/latex/oberdiek/makerobust.pdf
%    TDS:doc/latex/oberdiek/makerobust-example.tex
%    TDS:source/latex/oberdiek/makerobust.dtx
%
%<*ignore>
\begingroup
  \catcode123=1 %
  \catcode125=2 %
  \def\x{LaTeX2e}%
\expandafter\endgroup
\ifcase 0\ifx\install y1\fi\expandafter
         \ifx\csname processbatchFile\endcsname\relax\else1\fi
         \ifx\fmtname\x\else 1\fi\relax
\else\csname fi\endcsname
%</ignore>
%<*install>
\input docstrip.tex
\Msg{************************************************************************}
\Msg{* Installation}
\Msg{* Package: makerobust 2006/03/18 v1.0 Make existing macro robust (HO)}
\Msg{************************************************************************}

\keepsilent
\askforoverwritefalse

\let\MetaPrefix\relax
\preamble

This is a generated file.

Project: makerobust
Version: 2006/03/18 v1.0

Copyright (C) 2006 by
   Heiko Oberdiek <heiko.oberdiek at googlemail.com>

This work may be distributed and/or modified under the
conditions of the LaTeX Project Public License, either
version 1.3c of this license or (at your option) any later
version. This version of this license is in
   http://www.latex-project.org/lppl/lppl-1-3c.txt
and the latest version of this license is in
   http://www.latex-project.org/lppl.txt
and version 1.3 or later is part of all distributions of
LaTeX version 2005/12/01 or later.

This work has the LPPL maintenance status "maintained".

This Current Maintainer of this work is Heiko Oberdiek.

This work consists of the main source file makerobust.dtx
and the derived files
   makerobust.sty, makerobust.pdf, makerobust.ins, makerobust.drv,
   makerobust-example.tex.

\endpreamble
\let\MetaPrefix\DoubleperCent

\generate{%
  \file{makerobust.ins}{\from{makerobust.dtx}{install}}%
  \file{makerobust.drv}{\from{makerobust.dtx}{driver}}%
  \usedir{tex/latex/oberdiek}%
  \file{makerobust.sty}{\from{makerobust.dtx}{package}}%
  \usedir{doc/latex/oberdiek}%
  \file{makerobust-example.tex}{\from{makerobust.dtx}{example}}%
  \nopreamble
  \nopostamble
  \usedir{source/latex/oberdiek/catalogue}%
  \file{makerobust.xml}{\from{makerobust.dtx}{catalogue}}%
}

\catcode32=13\relax% active space
\let =\space%
\Msg{************************************************************************}
\Msg{*}
\Msg{* To finish the installation you have to move the following}
\Msg{* file into a directory searched by TeX:}
\Msg{*}
\Msg{*     makerobust.sty}
\Msg{*}
\Msg{* To produce the documentation run the file `makerobust.drv'}
\Msg{* through LaTeX.}
\Msg{*}
\Msg{* Happy TeXing!}
\Msg{*}
\Msg{************************************************************************}

\endbatchfile
%</install>
%<*ignore>
\fi
%</ignore>
%<*driver>
\NeedsTeXFormat{LaTeX2e}
\ProvidesFile{makerobust.drv}%
  [2006/03/18 v1.0 Make existing macro robust (HO)]%
\documentclass{ltxdoc}
\usepackage{holtxdoc}[2011/11/22]
\begin{document}
  \DocInput{makerobust.dtx}%
\end{document}
%</driver>
% \fi
%
% \CheckSum{59}
%
% \CharacterTable
%  {Upper-case    \A\B\C\D\E\F\G\H\I\J\K\L\M\N\O\P\Q\R\S\T\U\V\W\X\Y\Z
%   Lower-case    \a\b\c\d\e\f\g\h\i\j\k\l\m\n\o\p\q\r\s\t\u\v\w\x\y\z
%   Digits        \0\1\2\3\4\5\6\7\8\9
%   Exclamation   \!     Double quote  \"     Hash (number) \#
%   Dollar        \$     Percent       \%     Ampersand     \&
%   Acute accent  \'     Left paren    \(     Right paren   \)
%   Asterisk      \*     Plus          \+     Comma         \,
%   Minus         \-     Point         \.     Solidus       \/
%   Colon         \:     Semicolon     \;     Less than     \<
%   Equals        \=     Greater than  \>     Question mark \?
%   Commercial at \@     Left bracket  \[     Backslash     \\
%   Right bracket \]     Circumflex    \^     Underscore    \_
%   Grave accent  \`     Left brace    \{     Vertical bar  \|
%   Right brace   \}     Tilde         \~}
%
% \GetFileInfo{makerobust.drv}
%
% \title{The \xpackage{makerobust} package}
% \date{2006/03/18 v1.0}
% \author{Heiko Oberdiek\\\xemail{heiko.oberdiek at googlemail.com}}
%
% \maketitle
%
% \begin{abstract}
% Package \xpackage{makerobust} provides \cs{MakeRobustCommand}
% that converts an existing macro to a robust one.
% \end{abstract}
%
% \tableofcontents
%
% \section{User interface}
%
% \LaTeX\ offers \cs{DeclareRobustCommand} to define a robust macro
% that does not break if it is used in moving arguments.
% Sometimes a macro is already defined, but not robust. For
% example, \cs{(} and \cs{)} are not robust, inside \cs{section}
% the user must use \cs{protect} explicitly. This could be
% avoided by making \cs{(} and \cs{)} robust.
%
% \begin{declcs}{MakeRobustCommand}\M{cmd}
% \end{declcs}
% \cs{MakeRobustCommand} redefines the macro \meta{cmd}
% by using \cs{DeclareRobustCommand} and the existing definition
% of the macro \meta{cmd}.
% \begin{itemize}
% \item It is an error if \meta{cmd} is undefined. If you want to
%   define a robust command, then you can use \cs{DeclareRobustCommand}
%   directly.
% \item If the macro has previously been
%   defined by \cs{DeclareRobustCommand} then the redefinition of
%   \cs{MakeRobustCommand} is omitted, because the macro is already robust.
%   Only an information entry is written to the \xfile{.log} file.
%   Thus you do not get a warning or an error if the macro is already
%   robust because of an updated LaTeX or package that defines the macro.
% \item Two macros are defined for a macro, defined
%   by \cs{DeclareRobustCommand}. Example:
%   \begin{quote}
%   |\DeclareRobustCommand{\foobar}{definition text}|
%   \end{quote}
%   Then the macro ``\cs{foobar}'' contains the protection code
%   and, depending on the protection mode,
%   calls the internal macro ``\cs{foobar }''. Notice the space
%   at the end of the macro name.
%   This internal macro ``\cs{foobar }'' now contains the definition
%   ``|definition text|'', given in \cs{DeclareRobustCommand}.
%
%   Sometimes it can happen, that the internal macro already exists.
%   This can be caused by a previous \cs{DeclareRobustCommand} followed
%   by \cs{renewcommand}. Then the redefinition by \cs{MakeRobustCommand}
%   would be safe.
%
%   However, it can also be possible that the macro is already robust,
%   using the internal macro, but with a different protection code.
%   The redefinition by \cs{MakeRobustCommand} would then generate
%   an infinite loop.
%
%   Therefore \cs{MakeRobustCommand} raises an error message,
%   if the internal macro (with space at the end) already exists.
% \end{itemize}
%
% \subsection{Example}
%
%    \begin{macrocode}
%<*example>
\documentclass{article}
\usepackage{makerobust}
\MakeRobustCommand\(
\MakeRobustCommand\)
\pagestyle{headings}
\begin{document}
\tableofcontents
\section{Einstein: \(E=mc^2\)}
\newpage
Second page.
\end{document}
%</example>
%    \end{macrocode}
%
%
% \StopEventually{
% }
%
% \section{Implementation}
%
%    \begin{macrocode}
%<*package>
\NeedsTeXFormat{LaTeX2e}
\ProvidesPackage{makerobust}%
  [2006/03/18 v1.0 Make existing macro robust (HO)]%
%    \end{macrocode}
%
%    \begin{macrocode}
\def\MakeRobustCommand#1{%
  \begingroup
  \@ifundefined{\expandafter\@gobble\string#1}{%
    \endgroup
    \PackageError{makerobust}{%
      Macro \string`\string#1\string' is not defined%
    }\@ehc
  }{%
    \global\let\MR@gtemp#1%
    \let#1\@undefined
    \expandafter\let\expandafter\MR@temp
        \csname\expandafter\@gobble\string#1 \endcsname
    \DeclareRobustCommand#1{}%
    \ifx#1\MR@gtemp
      \endgroup
      \PackageInfo{makerobust}{%
        \string`\string#1\string' is already robust%
      }%
    \else
      \@ifundefined{MR@temp}{%
        \global\let\MR@gtemp#1%
        \endgroup
        \expandafter\let\csname\expandafter\@gobble\string#1 \endcsname#1%
        \let#1\MR@gtemp
      }{%
        \endgroup
        \PackageError{makerobust}{%
          Internal macro \string`\string#1 \string' already exists%
        }\@ehc
      }%
    \fi
  }%
}
%    \end{macrocode}
%
%    \begin{macrocode}
%</package>
%    \end{macrocode}
%
% \section{Installation}
%
% \subsection{Download}
%
% \paragraph{Package.} This package is available on
% CTAN\footnote{\url{ftp://ftp.ctan.org/tex-archive/}}:
% \begin{description}
% \item[\CTAN{macros/latex/contrib/oberdiek/makerobust.dtx}] The source file.
% \item[\CTAN{macros/latex/contrib/oberdiek/makerobust.pdf}] Documentation.
% \end{description}
%
%
% \paragraph{Bundle.} All the packages of the bundle `oberdiek'
% are also available in a TDS compliant ZIP archive. There
% the packages are already unpacked and the documentation files
% are generated. The files and directories obey the TDS standard.
% \begin{description}
% \item[\CTAN{install/macros/latex/contrib/oberdiek.tds.zip}]
% \end{description}
% \emph{TDS} refers to the standard ``A Directory Structure
% for \TeX\ Files'' (\CTAN{tds/tds.pdf}). Directories
% with \xfile{texmf} in their name are usually organized this way.
%
% \subsection{Bundle installation}
%
% \paragraph{Unpacking.} Unpack the \xfile{oberdiek.tds.zip} in the
% TDS tree (also known as \xfile{texmf} tree) of your choice.
% Example (linux):
% \begin{quote}
%   |unzip oberdiek.tds.zip -d ~/texmf|
% \end{quote}
%
% \paragraph{Script installation.}
% Check the directory \xfile{TDS:scripts/oberdiek/} for
% scripts that need further installation steps.
% Package \xpackage{attachfile2} comes with the Perl script
% \xfile{pdfatfi.pl} that should be installed in such a way
% that it can be called as \texttt{pdfatfi}.
% Example (linux):
% \begin{quote}
%   |chmod +x scripts/oberdiek/pdfatfi.pl|\\
%   |cp scripts/oberdiek/pdfatfi.pl /usr/local/bin/|
% \end{quote}
%
% \subsection{Package installation}
%
% \paragraph{Unpacking.} The \xfile{.dtx} file is a self-extracting
% \docstrip\ archive. The files are extracted by running the
% \xfile{.dtx} through \plainTeX:
% \begin{quote}
%   \verb|tex makerobust.dtx|
% \end{quote}
%
% \paragraph{TDS.} Now the different files must be moved into
% the different directories in your installation TDS tree
% (also known as \xfile{texmf} tree):
% \begin{quote}
% \def\t{^^A
% \begin{tabular}{@{}>{\ttfamily}l@{ $\rightarrow$ }>{\ttfamily}l@{}}
%   makerobust.sty & tex/latex/oberdiek/makerobust.sty\\
%   makerobust.pdf & doc/latex/oberdiek/makerobust.pdf\\
%   makerobust-example.tex & doc/latex/oberdiek/makerobust-example.tex\\
%   makerobust.dtx & source/latex/oberdiek/makerobust.dtx\\
% \end{tabular}^^A
% }^^A
% \sbox0{\t}^^A
% \ifdim\wd0>\linewidth
%   \begingroup
%     \advance\linewidth by\leftmargin
%     \advance\linewidth by\rightmargin
%   \edef\x{\endgroup
%     \def\noexpand\lw{\the\linewidth}^^A
%   }\x
%   \def\lwbox{^^A
%     \leavevmode
%     \hbox to \linewidth{^^A
%       \kern-\leftmargin\relax
%       \hss
%       \usebox0
%       \hss
%       \kern-\rightmargin\relax
%     }^^A
%   }^^A
%   \ifdim\wd0>\lw
%     \sbox0{\small\t}^^A
%     \ifdim\wd0>\linewidth
%       \ifdim\wd0>\lw
%         \sbox0{\footnotesize\t}^^A
%         \ifdim\wd0>\linewidth
%           \ifdim\wd0>\lw
%             \sbox0{\scriptsize\t}^^A
%             \ifdim\wd0>\linewidth
%               \ifdim\wd0>\lw
%                 \sbox0{\tiny\t}^^A
%                 \ifdim\wd0>\linewidth
%                   \lwbox
%                 \else
%                   \usebox0
%                 \fi
%               \else
%                 \lwbox
%               \fi
%             \else
%               \usebox0
%             \fi
%           \else
%             \lwbox
%           \fi
%         \else
%           \usebox0
%         \fi
%       \else
%         \lwbox
%       \fi
%     \else
%       \usebox0
%     \fi
%   \else
%     \lwbox
%   \fi
% \else
%   \usebox0
% \fi
% \end{quote}
% If you have a \xfile{docstrip.cfg} that configures and enables \docstrip's
% TDS installing feature, then some files can already be in the right
% place, see the documentation of \docstrip.
%
% \subsection{Refresh file name databases}
%
% If your \TeX~distribution
% (\teTeX, \mikTeX, \dots) relies on file name databases, you must refresh
% these. For example, \teTeX\ users run \verb|texhash| or
% \verb|mktexlsr|.
%
% \subsection{Some details for the interested}
%
% \paragraph{Attached source.}
%
% The PDF documentation on CTAN also includes the
% \xfile{.dtx} source file. It can be extracted by
% AcrobatReader 6 or higher. Another option is \textsf{pdftk},
% e.g. unpack the file into the current directory:
% \begin{quote}
%   \verb|pdftk makerobust.pdf unpack_files output .|
% \end{quote}
%
% \paragraph{Unpacking with \LaTeX.}
% The \xfile{.dtx} chooses its action depending on the format:
% \begin{description}
% \item[\plainTeX:] Run \docstrip\ and extract the files.
% \item[\LaTeX:] Generate the documentation.
% \end{description}
% If you insist on using \LaTeX\ for \docstrip\ (really,
% \docstrip\ does not need \LaTeX), then inform the autodetect routine
% about your intention:
% \begin{quote}
%   \verb|latex \let\install=y\input{makerobust.dtx}|
% \end{quote}
% Do not forget to quote the argument according to the demands
% of your shell.
%
% \paragraph{Generating the documentation.}
% You can use both the \xfile{.dtx} or the \xfile{.drv} to generate
% the documentation. The process can be configured by the
% configuration file \xfile{ltxdoc.cfg}. For instance, put this
% line into this file, if you want to have A4 as paper format:
% \begin{quote}
%   \verb|\PassOptionsToClass{a4paper}{article}|
% \end{quote}
% An example follows how to generate the
% documentation with pdf\LaTeX:
% \begin{quote}
%\begin{verbatim}
%pdflatex makerobust.dtx
%makeindex -s gind.ist makerobust.idx
%pdflatex makerobust.dtx
%makeindex -s gind.ist makerobust.idx
%pdflatex makerobust.dtx
%\end{verbatim}
% \end{quote}
%
% \section{Catalogue}
%
% The following XML file can be used as source for the
% \href{http://mirror.ctan.org/help/Catalogue/catalogue.html}{\TeX\ Catalogue}.
% The elements \texttt{caption} and \texttt{description} are imported
% from the original XML file from the Catalogue.
% The name of the XML file in the Catalogue is \xfile{makerobust.xml}.
%    \begin{macrocode}
%<*catalogue>
<?xml version='1.0' encoding='us-ascii'?>
<!DOCTYPE entry SYSTEM 'catalogue.dtd'>
<entry datestamp='$Date$' modifier='$Author$' id='makerobust'>
  <name>makerobust</name>
  <caption>Making a macro robust.</caption>
  <authorref id='auth:oberdiek'/>
  <copyright owner='Heiko Oberdiek' year='2006'/>
  <license type='lppl1.3'/>
  <version number='1.0'/>
  <description>
    This package provides the command MakeRobustCommand
    that converts an existing macro to a robust one.
    <p/>
    The package is part of the <xref refid='oberdiek'>oberdiek</xref>
    bundle.
  </description>
  <documentation details='Package documentation'
      href='ctan:/macros/latex/contrib/oberdiek/makerobust.pdf'/>
  <ctan file='true' path='/macros/latex/contrib/oberdiek/makerobust.dtx'/>
  <miktex location='oberdiek'/>
  <texlive location='oberdiek'/>
  <install path='/macros/latex/contrib/oberdiek/oberdiek.tds.zip'/>
</entry>
%</catalogue>
%    \end{macrocode}
%
% \begin{History}
%   \begin{Version}{2006/03/18 v1.0}
%   \item
%     First version.
%   \end{Version}
% \end{History}
%
% \PrintIndex
%
% \Finale
\endinput

%        (quote the arguments according to the demands of your shell)
%
% Documentation:
%    (a) If makerobust.drv is present:
%           latex makerobust.drv
%    (b) Without makerobust.drv:
%           latex makerobust.dtx; ...
%    The class ltxdoc loads the configuration file ltxdoc.cfg
%    if available. Here you can specify further options, e.g.
%    use A4 as paper format:
%       \PassOptionsToClass{a4paper}{article}
%
%    Programm calls to get the documentation (example):
%       pdflatex makerobust.dtx
%       makeindex -s gind.ist makerobust.idx
%       pdflatex makerobust.dtx
%       makeindex -s gind.ist makerobust.idx
%       pdflatex makerobust.dtx
%
% Installation:
%    TDS:tex/latex/oberdiek/makerobust.sty
%    TDS:doc/latex/oberdiek/makerobust.pdf
%    TDS:doc/latex/oberdiek/makerobust-example.tex
%    TDS:source/latex/oberdiek/makerobust.dtx
%
%<*ignore>
\begingroup
  \catcode123=1 %
  \catcode125=2 %
  \def\x{LaTeX2e}%
\expandafter\endgroup
\ifcase 0\ifx\install y1\fi\expandafter
         \ifx\csname processbatchFile\endcsname\relax\else1\fi
         \ifx\fmtname\x\else 1\fi\relax
\else\csname fi\endcsname
%</ignore>
%<*install>
\input docstrip.tex
\Msg{************************************************************************}
\Msg{* Installation}
\Msg{* Package: makerobust 2006/03/18 v1.0 Make existing macro robust (HO)}
\Msg{************************************************************************}

\keepsilent
\askforoverwritefalse

\let\MetaPrefix\relax
\preamble

This is a generated file.

Project: makerobust
Version: 2006/03/18 v1.0

Copyright (C) 2006 by
   Heiko Oberdiek <heiko.oberdiek at googlemail.com>

This work may be distributed and/or modified under the
conditions of the LaTeX Project Public License, either
version 1.3c of this license or (at your option) any later
version. This version of this license is in
   http://www.latex-project.org/lppl/lppl-1-3c.txt
and the latest version of this license is in
   http://www.latex-project.org/lppl.txt
and version 1.3 or later is part of all distributions of
LaTeX version 2005/12/01 or later.

This work has the LPPL maintenance status "maintained".

This Current Maintainer of this work is Heiko Oberdiek.

This work consists of the main source file makerobust.dtx
and the derived files
   makerobust.sty, makerobust.pdf, makerobust.ins, makerobust.drv,
   makerobust-example.tex.

\endpreamble
\let\MetaPrefix\DoubleperCent

\generate{%
  \file{makerobust.ins}{\from{makerobust.dtx}{install}}%
  \file{makerobust.drv}{\from{makerobust.dtx}{driver}}%
  \usedir{tex/latex/oberdiek}%
  \file{makerobust.sty}{\from{makerobust.dtx}{package}}%
  \usedir{doc/latex/oberdiek}%
  \file{makerobust-example.tex}{\from{makerobust.dtx}{example}}%
  \nopreamble
  \nopostamble
  \usedir{source/latex/oberdiek/catalogue}%
  \file{makerobust.xml}{\from{makerobust.dtx}{catalogue}}%
}

\catcode32=13\relax% active space
\let =\space%
\Msg{************************************************************************}
\Msg{*}
\Msg{* To finish the installation you have to move the following}
\Msg{* file into a directory searched by TeX:}
\Msg{*}
\Msg{*     makerobust.sty}
\Msg{*}
\Msg{* To produce the documentation run the file `makerobust.drv'}
\Msg{* through LaTeX.}
\Msg{*}
\Msg{* Happy TeXing!}
\Msg{*}
\Msg{************************************************************************}

\endbatchfile
%</install>
%<*ignore>
\fi
%</ignore>
%<*driver>
\NeedsTeXFormat{LaTeX2e}
\ProvidesFile{makerobust.drv}%
  [2006/03/18 v1.0 Make existing macro robust (HO)]%
\documentclass{ltxdoc}
\usepackage{holtxdoc}[2011/11/22]
\begin{document}
  \DocInput{makerobust.dtx}%
\end{document}
%</driver>
% \fi
%
% \CheckSum{59}
%
% \CharacterTable
%  {Upper-case    \A\B\C\D\E\F\G\H\I\J\K\L\M\N\O\P\Q\R\S\T\U\V\W\X\Y\Z
%   Lower-case    \a\b\c\d\e\f\g\h\i\j\k\l\m\n\o\p\q\r\s\t\u\v\w\x\y\z
%   Digits        \0\1\2\3\4\5\6\7\8\9
%   Exclamation   \!     Double quote  \"     Hash (number) \#
%   Dollar        \$     Percent       \%     Ampersand     \&
%   Acute accent  \'     Left paren    \(     Right paren   \)
%   Asterisk      \*     Plus          \+     Comma         \,
%   Minus         \-     Point         \.     Solidus       \/
%   Colon         \:     Semicolon     \;     Less than     \<
%   Equals        \=     Greater than  \>     Question mark \?
%   Commercial at \@     Left bracket  \[     Backslash     \\
%   Right bracket \]     Circumflex    \^     Underscore    \_
%   Grave accent  \`     Left brace    \{     Vertical bar  \|
%   Right brace   \}     Tilde         \~}
%
% \GetFileInfo{makerobust.drv}
%
% \title{The \xpackage{makerobust} package}
% \date{2006/03/18 v1.0}
% \author{Heiko Oberdiek\\\xemail{heiko.oberdiek at googlemail.com}}
%
% \maketitle
%
% \begin{abstract}
% Package \xpackage{makerobust} provides \cs{MakeRobustCommand}
% that converts an existing macro to a robust one.
% \end{abstract}
%
% \tableofcontents
%
% \section{User interface}
%
% \LaTeX\ offers \cs{DeclareRobustCommand} to define a robust macro
% that does not break if it is used in moving arguments.
% Sometimes a macro is already defined, but not robust. For
% example, \cs{(} and \cs{)} are not robust, inside \cs{section}
% the user must use \cs{protect} explicitly. This could be
% avoided by making \cs{(} and \cs{)} robust.
%
% \begin{declcs}{MakeRobustCommand}\M{cmd}
% \end{declcs}
% \cs{MakeRobustCommand} redefines the macro \meta{cmd}
% by using \cs{DeclareRobustCommand} and the existing definition
% of the macro \meta{cmd}.
% \begin{itemize}
% \item It is an error if \meta{cmd} is undefined. If you want to
%   define a robust command, then you can use \cs{DeclareRobustCommand}
%   directly.
% \item If the macro has previously been
%   defined by \cs{DeclareRobustCommand} then the redefinition of
%   \cs{MakeRobustCommand} is omitted, because the macro is already robust.
%   Only an information entry is written to the \xfile{.log} file.
%   Thus you do not get a warning or an error if the macro is already
%   robust because of an updated LaTeX or package that defines the macro.
% \item Two macros are defined for a macro, defined
%   by \cs{DeclareRobustCommand}. Example:
%   \begin{quote}
%   |\DeclareRobustCommand{\foobar}{definition text}|
%   \end{quote}
%   Then the macro ``\cs{foobar}'' contains the protection code
%   and, depending on the protection mode,
%   calls the internal macro ``\cs{foobar }''. Notice the space
%   at the end of the macro name.
%   This internal macro ``\cs{foobar }'' now contains the definition
%   ``|definition text|'', given in \cs{DeclareRobustCommand}.
%
%   Sometimes it can happen, that the internal macro already exists.
%   This can be caused by a previous \cs{DeclareRobustCommand} followed
%   by \cs{renewcommand}. Then the redefinition by \cs{MakeRobustCommand}
%   would be safe.
%
%   However, it can also be possible that the macro is already robust,
%   using the internal macro, but with a different protection code.
%   The redefinition by \cs{MakeRobustCommand} would then generate
%   an infinite loop.
%
%   Therefore \cs{MakeRobustCommand} raises an error message,
%   if the internal macro (with space at the end) already exists.
% \end{itemize}
%
% \subsection{Example}
%
%    \begin{macrocode}
%<*example>
\documentclass{article}
\usepackage{makerobust}
\MakeRobustCommand\(
\MakeRobustCommand\)
\pagestyle{headings}
\begin{document}
\tableofcontents
\section{Einstein: \(E=mc^2\)}
\newpage
Second page.
\end{document}
%</example>
%    \end{macrocode}
%
%
% \StopEventually{
% }
%
% \section{Implementation}
%
%    \begin{macrocode}
%<*package>
\NeedsTeXFormat{LaTeX2e}
\ProvidesPackage{makerobust}%
  [2006/03/18 v1.0 Make existing macro robust (HO)]%
%    \end{macrocode}
%
%    \begin{macrocode}
\def\MakeRobustCommand#1{%
  \begingroup
  \@ifundefined{\expandafter\@gobble\string#1}{%
    \endgroup
    \PackageError{makerobust}{%
      Macro \string`\string#1\string' is not defined%
    }\@ehc
  }{%
    \global\let\MR@gtemp#1%
    \let#1\@undefined
    \expandafter\let\expandafter\MR@temp
        \csname\expandafter\@gobble\string#1 \endcsname
    \DeclareRobustCommand#1{}%
    \ifx#1\MR@gtemp
      \endgroup
      \PackageInfo{makerobust}{%
        \string`\string#1\string' is already robust%
      }%
    \else
      \@ifundefined{MR@temp}{%
        \global\let\MR@gtemp#1%
        \endgroup
        \expandafter\let\csname\expandafter\@gobble\string#1 \endcsname#1%
        \let#1\MR@gtemp
      }{%
        \endgroup
        \PackageError{makerobust}{%
          Internal macro \string`\string#1 \string' already exists%
        }\@ehc
      }%
    \fi
  }%
}
%    \end{macrocode}
%
%    \begin{macrocode}
%</package>
%    \end{macrocode}
%
% \section{Installation}
%
% \subsection{Download}
%
% \paragraph{Package.} This package is available on
% CTAN\footnote{\url{ftp://ftp.ctan.org/tex-archive/}}:
% \begin{description}
% \item[\CTAN{macros/latex/contrib/oberdiek/makerobust.dtx}] The source file.
% \item[\CTAN{macros/latex/contrib/oberdiek/makerobust.pdf}] Documentation.
% \end{description}
%
%
% \paragraph{Bundle.} All the packages of the bundle `oberdiek'
% are also available in a TDS compliant ZIP archive. There
% the packages are already unpacked and the documentation files
% are generated. The files and directories obey the TDS standard.
% \begin{description}
% \item[\CTAN{install/macros/latex/contrib/oberdiek.tds.zip}]
% \end{description}
% \emph{TDS} refers to the standard ``A Directory Structure
% for \TeX\ Files'' (\CTAN{tds/tds.pdf}). Directories
% with \xfile{texmf} in their name are usually organized this way.
%
% \subsection{Bundle installation}
%
% \paragraph{Unpacking.} Unpack the \xfile{oberdiek.tds.zip} in the
% TDS tree (also known as \xfile{texmf} tree) of your choice.
% Example (linux):
% \begin{quote}
%   |unzip oberdiek.tds.zip -d ~/texmf|
% \end{quote}
%
% \paragraph{Script installation.}
% Check the directory \xfile{TDS:scripts/oberdiek/} for
% scripts that need further installation steps.
% Package \xpackage{attachfile2} comes with the Perl script
% \xfile{pdfatfi.pl} that should be installed in such a way
% that it can be called as \texttt{pdfatfi}.
% Example (linux):
% \begin{quote}
%   |chmod +x scripts/oberdiek/pdfatfi.pl|\\
%   |cp scripts/oberdiek/pdfatfi.pl /usr/local/bin/|
% \end{quote}
%
% \subsection{Package installation}
%
% \paragraph{Unpacking.} The \xfile{.dtx} file is a self-extracting
% \docstrip\ archive. The files are extracted by running the
% \xfile{.dtx} through \plainTeX:
% \begin{quote}
%   \verb|tex makerobust.dtx|
% \end{quote}
%
% \paragraph{TDS.} Now the different files must be moved into
% the different directories in your installation TDS tree
% (also known as \xfile{texmf} tree):
% \begin{quote}
% \def\t{^^A
% \begin{tabular}{@{}>{\ttfamily}l@{ $\rightarrow$ }>{\ttfamily}l@{}}
%   makerobust.sty & tex/latex/oberdiek/makerobust.sty\\
%   makerobust.pdf & doc/latex/oberdiek/makerobust.pdf\\
%   makerobust-example.tex & doc/latex/oberdiek/makerobust-example.tex\\
%   makerobust.dtx & source/latex/oberdiek/makerobust.dtx\\
% \end{tabular}^^A
% }^^A
% \sbox0{\t}^^A
% \ifdim\wd0>\linewidth
%   \begingroup
%     \advance\linewidth by\leftmargin
%     \advance\linewidth by\rightmargin
%   \edef\x{\endgroup
%     \def\noexpand\lw{\the\linewidth}^^A
%   }\x
%   \def\lwbox{^^A
%     \leavevmode
%     \hbox to \linewidth{^^A
%       \kern-\leftmargin\relax
%       \hss
%       \usebox0
%       \hss
%       \kern-\rightmargin\relax
%     }^^A
%   }^^A
%   \ifdim\wd0>\lw
%     \sbox0{\small\t}^^A
%     \ifdim\wd0>\linewidth
%       \ifdim\wd0>\lw
%         \sbox0{\footnotesize\t}^^A
%         \ifdim\wd0>\linewidth
%           \ifdim\wd0>\lw
%             \sbox0{\scriptsize\t}^^A
%             \ifdim\wd0>\linewidth
%               \ifdim\wd0>\lw
%                 \sbox0{\tiny\t}^^A
%                 \ifdim\wd0>\linewidth
%                   \lwbox
%                 \else
%                   \usebox0
%                 \fi
%               \else
%                 \lwbox
%               \fi
%             \else
%               \usebox0
%             \fi
%           \else
%             \lwbox
%           \fi
%         \else
%           \usebox0
%         \fi
%       \else
%         \lwbox
%       \fi
%     \else
%       \usebox0
%     \fi
%   \else
%     \lwbox
%   \fi
% \else
%   \usebox0
% \fi
% \end{quote}
% If you have a \xfile{docstrip.cfg} that configures and enables \docstrip's
% TDS installing feature, then some files can already be in the right
% place, see the documentation of \docstrip.
%
% \subsection{Refresh file name databases}
%
% If your \TeX~distribution
% (\teTeX, \mikTeX, \dots) relies on file name databases, you must refresh
% these. For example, \teTeX\ users run \verb|texhash| or
% \verb|mktexlsr|.
%
% \subsection{Some details for the interested}
%
% \paragraph{Attached source.}
%
% The PDF documentation on CTAN also includes the
% \xfile{.dtx} source file. It can be extracted by
% AcrobatReader 6 or higher. Another option is \textsf{pdftk},
% e.g. unpack the file into the current directory:
% \begin{quote}
%   \verb|pdftk makerobust.pdf unpack_files output .|
% \end{quote}
%
% \paragraph{Unpacking with \LaTeX.}
% The \xfile{.dtx} chooses its action depending on the format:
% \begin{description}
% \item[\plainTeX:] Run \docstrip\ and extract the files.
% \item[\LaTeX:] Generate the documentation.
% \end{description}
% If you insist on using \LaTeX\ for \docstrip\ (really,
% \docstrip\ does not need \LaTeX), then inform the autodetect routine
% about your intention:
% \begin{quote}
%   \verb|latex \let\install=y% \iffalse meta-comment
%
% File: makerobust.dtx
% Version: 2006/03/18 v1.0
% Info: Make existing macro robust
%
% Copyright (C) 2006 by
%    Heiko Oberdiek <heiko.oberdiek at googlemail.com>
%
% This work may be distributed and/or modified under the
% conditions of the LaTeX Project Public License, either
% version 1.3c of this license or (at your option) any later
% version. This version of this license is in
%    http://www.latex-project.org/lppl/lppl-1-3c.txt
% and the latest version of this license is in
%    http://www.latex-project.org/lppl.txt
% and version 1.3 or later is part of all distributions of
% LaTeX version 2005/12/01 or later.
%
% This work has the LPPL maintenance status "maintained".
%
% This Current Maintainer of this work is Heiko Oberdiek.
%
% This work consists of the main source file makerobust.dtx
% and the derived files
%    makerobust.sty, makerobust.pdf, makerobust.ins, makerobust.drv,
%    makerobust-example.tex.
%
% Distribution:
%    CTAN:macros/latex/contrib/oberdiek/makerobust.dtx
%    CTAN:macros/latex/contrib/oberdiek/makerobust.pdf
%
% Unpacking:
%    (a) If makerobust.ins is present:
%           tex makerobust.ins
%    (b) Without makerobust.ins:
%           tex makerobust.dtx
%    (c) If you insist on using LaTeX
%           latex \let\install=y\input{makerobust.dtx}
%        (quote the arguments according to the demands of your shell)
%
% Documentation:
%    (a) If makerobust.drv is present:
%           latex makerobust.drv
%    (b) Without makerobust.drv:
%           latex makerobust.dtx; ...
%    The class ltxdoc loads the configuration file ltxdoc.cfg
%    if available. Here you can specify further options, e.g.
%    use A4 as paper format:
%       \PassOptionsToClass{a4paper}{article}
%
%    Programm calls to get the documentation (example):
%       pdflatex makerobust.dtx
%       makeindex -s gind.ist makerobust.idx
%       pdflatex makerobust.dtx
%       makeindex -s gind.ist makerobust.idx
%       pdflatex makerobust.dtx
%
% Installation:
%    TDS:tex/latex/oberdiek/makerobust.sty
%    TDS:doc/latex/oberdiek/makerobust.pdf
%    TDS:doc/latex/oberdiek/makerobust-example.tex
%    TDS:source/latex/oberdiek/makerobust.dtx
%
%<*ignore>
\begingroup
  \catcode123=1 %
  \catcode125=2 %
  \def\x{LaTeX2e}%
\expandafter\endgroup
\ifcase 0\ifx\install y1\fi\expandafter
         \ifx\csname processbatchFile\endcsname\relax\else1\fi
         \ifx\fmtname\x\else 1\fi\relax
\else\csname fi\endcsname
%</ignore>
%<*install>
\input docstrip.tex
\Msg{************************************************************************}
\Msg{* Installation}
\Msg{* Package: makerobust 2006/03/18 v1.0 Make existing macro robust (HO)}
\Msg{************************************************************************}

\keepsilent
\askforoverwritefalse

\let\MetaPrefix\relax
\preamble

This is a generated file.

Project: makerobust
Version: 2006/03/18 v1.0

Copyright (C) 2006 by
   Heiko Oberdiek <heiko.oberdiek at googlemail.com>

This work may be distributed and/or modified under the
conditions of the LaTeX Project Public License, either
version 1.3c of this license or (at your option) any later
version. This version of this license is in
   http://www.latex-project.org/lppl/lppl-1-3c.txt
and the latest version of this license is in
   http://www.latex-project.org/lppl.txt
and version 1.3 or later is part of all distributions of
LaTeX version 2005/12/01 or later.

This work has the LPPL maintenance status "maintained".

This Current Maintainer of this work is Heiko Oberdiek.

This work consists of the main source file makerobust.dtx
and the derived files
   makerobust.sty, makerobust.pdf, makerobust.ins, makerobust.drv,
   makerobust-example.tex.

\endpreamble
\let\MetaPrefix\DoubleperCent

\generate{%
  \file{makerobust.ins}{\from{makerobust.dtx}{install}}%
  \file{makerobust.drv}{\from{makerobust.dtx}{driver}}%
  \usedir{tex/latex/oberdiek}%
  \file{makerobust.sty}{\from{makerobust.dtx}{package}}%
  \usedir{doc/latex/oberdiek}%
  \file{makerobust-example.tex}{\from{makerobust.dtx}{example}}%
  \nopreamble
  \nopostamble
  \usedir{source/latex/oberdiek/catalogue}%
  \file{makerobust.xml}{\from{makerobust.dtx}{catalogue}}%
}

\catcode32=13\relax% active space
\let =\space%
\Msg{************************************************************************}
\Msg{*}
\Msg{* To finish the installation you have to move the following}
\Msg{* file into a directory searched by TeX:}
\Msg{*}
\Msg{*     makerobust.sty}
\Msg{*}
\Msg{* To produce the documentation run the file `makerobust.drv'}
\Msg{* through LaTeX.}
\Msg{*}
\Msg{* Happy TeXing!}
\Msg{*}
\Msg{************************************************************************}

\endbatchfile
%</install>
%<*ignore>
\fi
%</ignore>
%<*driver>
\NeedsTeXFormat{LaTeX2e}
\ProvidesFile{makerobust.drv}%
  [2006/03/18 v1.0 Make existing macro robust (HO)]%
\documentclass{ltxdoc}
\usepackage{holtxdoc}[2011/11/22]
\begin{document}
  \DocInput{makerobust.dtx}%
\end{document}
%</driver>
% \fi
%
% \CheckSum{59}
%
% \CharacterTable
%  {Upper-case    \A\B\C\D\E\F\G\H\I\J\K\L\M\N\O\P\Q\R\S\T\U\V\W\X\Y\Z
%   Lower-case    \a\b\c\d\e\f\g\h\i\j\k\l\m\n\o\p\q\r\s\t\u\v\w\x\y\z
%   Digits        \0\1\2\3\4\5\6\7\8\9
%   Exclamation   \!     Double quote  \"     Hash (number) \#
%   Dollar        \$     Percent       \%     Ampersand     \&
%   Acute accent  \'     Left paren    \(     Right paren   \)
%   Asterisk      \*     Plus          \+     Comma         \,
%   Minus         \-     Point         \.     Solidus       \/
%   Colon         \:     Semicolon     \;     Less than     \<
%   Equals        \=     Greater than  \>     Question mark \?
%   Commercial at \@     Left bracket  \[     Backslash     \\
%   Right bracket \]     Circumflex    \^     Underscore    \_
%   Grave accent  \`     Left brace    \{     Vertical bar  \|
%   Right brace   \}     Tilde         \~}
%
% \GetFileInfo{makerobust.drv}
%
% \title{The \xpackage{makerobust} package}
% \date{2006/03/18 v1.0}
% \author{Heiko Oberdiek\\\xemail{heiko.oberdiek at googlemail.com}}
%
% \maketitle
%
% \begin{abstract}
% Package \xpackage{makerobust} provides \cs{MakeRobustCommand}
% that converts an existing macro to a robust one.
% \end{abstract}
%
% \tableofcontents
%
% \section{User interface}
%
% \LaTeX\ offers \cs{DeclareRobustCommand} to define a robust macro
% that does not break if it is used in moving arguments.
% Sometimes a macro is already defined, but not robust. For
% example, \cs{(} and \cs{)} are not robust, inside \cs{section}
% the user must use \cs{protect} explicitly. This could be
% avoided by making \cs{(} and \cs{)} robust.
%
% \begin{declcs}{MakeRobustCommand}\M{cmd}
% \end{declcs}
% \cs{MakeRobustCommand} redefines the macro \meta{cmd}
% by using \cs{DeclareRobustCommand} and the existing definition
% of the macro \meta{cmd}.
% \begin{itemize}
% \item It is an error if \meta{cmd} is undefined. If you want to
%   define a robust command, then you can use \cs{DeclareRobustCommand}
%   directly.
% \item If the macro has previously been
%   defined by \cs{DeclareRobustCommand} then the redefinition of
%   \cs{MakeRobustCommand} is omitted, because the macro is already robust.
%   Only an information entry is written to the \xfile{.log} file.
%   Thus you do not get a warning or an error if the macro is already
%   robust because of an updated LaTeX or package that defines the macro.
% \item Two macros are defined for a macro, defined
%   by \cs{DeclareRobustCommand}. Example:
%   \begin{quote}
%   |\DeclareRobustCommand{\foobar}{definition text}|
%   \end{quote}
%   Then the macro ``\cs{foobar}'' contains the protection code
%   and, depending on the protection mode,
%   calls the internal macro ``\cs{foobar }''. Notice the space
%   at the end of the macro name.
%   This internal macro ``\cs{foobar }'' now contains the definition
%   ``|definition text|'', given in \cs{DeclareRobustCommand}.
%
%   Sometimes it can happen, that the internal macro already exists.
%   This can be caused by a previous \cs{DeclareRobustCommand} followed
%   by \cs{renewcommand}. Then the redefinition by \cs{MakeRobustCommand}
%   would be safe.
%
%   However, it can also be possible that the macro is already robust,
%   using the internal macro, but with a different protection code.
%   The redefinition by \cs{MakeRobustCommand} would then generate
%   an infinite loop.
%
%   Therefore \cs{MakeRobustCommand} raises an error message,
%   if the internal macro (with space at the end) already exists.
% \end{itemize}
%
% \subsection{Example}
%
%    \begin{macrocode}
%<*example>
\documentclass{article}
\usepackage{makerobust}
\MakeRobustCommand\(
\MakeRobustCommand\)
\pagestyle{headings}
\begin{document}
\tableofcontents
\section{Einstein: \(E=mc^2\)}
\newpage
Second page.
\end{document}
%</example>
%    \end{macrocode}
%
%
% \StopEventually{
% }
%
% \section{Implementation}
%
%    \begin{macrocode}
%<*package>
\NeedsTeXFormat{LaTeX2e}
\ProvidesPackage{makerobust}%
  [2006/03/18 v1.0 Make existing macro robust (HO)]%
%    \end{macrocode}
%
%    \begin{macrocode}
\def\MakeRobustCommand#1{%
  \begingroup
  \@ifundefined{\expandafter\@gobble\string#1}{%
    \endgroup
    \PackageError{makerobust}{%
      Macro \string`\string#1\string' is not defined%
    }\@ehc
  }{%
    \global\let\MR@gtemp#1%
    \let#1\@undefined
    \expandafter\let\expandafter\MR@temp
        \csname\expandafter\@gobble\string#1 \endcsname
    \DeclareRobustCommand#1{}%
    \ifx#1\MR@gtemp
      \endgroup
      \PackageInfo{makerobust}{%
        \string`\string#1\string' is already robust%
      }%
    \else
      \@ifundefined{MR@temp}{%
        \global\let\MR@gtemp#1%
        \endgroup
        \expandafter\let\csname\expandafter\@gobble\string#1 \endcsname#1%
        \let#1\MR@gtemp
      }{%
        \endgroup
        \PackageError{makerobust}{%
          Internal macro \string`\string#1 \string' already exists%
        }\@ehc
      }%
    \fi
  }%
}
%    \end{macrocode}
%
%    \begin{macrocode}
%</package>
%    \end{macrocode}
%
% \section{Installation}
%
% \subsection{Download}
%
% \paragraph{Package.} This package is available on
% CTAN\footnote{\url{ftp://ftp.ctan.org/tex-archive/}}:
% \begin{description}
% \item[\CTAN{macros/latex/contrib/oberdiek/makerobust.dtx}] The source file.
% \item[\CTAN{macros/latex/contrib/oberdiek/makerobust.pdf}] Documentation.
% \end{description}
%
%
% \paragraph{Bundle.} All the packages of the bundle `oberdiek'
% are also available in a TDS compliant ZIP archive. There
% the packages are already unpacked and the documentation files
% are generated. The files and directories obey the TDS standard.
% \begin{description}
% \item[\CTAN{install/macros/latex/contrib/oberdiek.tds.zip}]
% \end{description}
% \emph{TDS} refers to the standard ``A Directory Structure
% for \TeX\ Files'' (\CTAN{tds/tds.pdf}). Directories
% with \xfile{texmf} in their name are usually organized this way.
%
% \subsection{Bundle installation}
%
% \paragraph{Unpacking.} Unpack the \xfile{oberdiek.tds.zip} in the
% TDS tree (also known as \xfile{texmf} tree) of your choice.
% Example (linux):
% \begin{quote}
%   |unzip oberdiek.tds.zip -d ~/texmf|
% \end{quote}
%
% \paragraph{Script installation.}
% Check the directory \xfile{TDS:scripts/oberdiek/} for
% scripts that need further installation steps.
% Package \xpackage{attachfile2} comes with the Perl script
% \xfile{pdfatfi.pl} that should be installed in such a way
% that it can be called as \texttt{pdfatfi}.
% Example (linux):
% \begin{quote}
%   |chmod +x scripts/oberdiek/pdfatfi.pl|\\
%   |cp scripts/oberdiek/pdfatfi.pl /usr/local/bin/|
% \end{quote}
%
% \subsection{Package installation}
%
% \paragraph{Unpacking.} The \xfile{.dtx} file is a self-extracting
% \docstrip\ archive. The files are extracted by running the
% \xfile{.dtx} through \plainTeX:
% \begin{quote}
%   \verb|tex makerobust.dtx|
% \end{quote}
%
% \paragraph{TDS.} Now the different files must be moved into
% the different directories in your installation TDS tree
% (also known as \xfile{texmf} tree):
% \begin{quote}
% \def\t{^^A
% \begin{tabular}{@{}>{\ttfamily}l@{ $\rightarrow$ }>{\ttfamily}l@{}}
%   makerobust.sty & tex/latex/oberdiek/makerobust.sty\\
%   makerobust.pdf & doc/latex/oberdiek/makerobust.pdf\\
%   makerobust-example.tex & doc/latex/oberdiek/makerobust-example.tex\\
%   makerobust.dtx & source/latex/oberdiek/makerobust.dtx\\
% \end{tabular}^^A
% }^^A
% \sbox0{\t}^^A
% \ifdim\wd0>\linewidth
%   \begingroup
%     \advance\linewidth by\leftmargin
%     \advance\linewidth by\rightmargin
%   \edef\x{\endgroup
%     \def\noexpand\lw{\the\linewidth}^^A
%   }\x
%   \def\lwbox{^^A
%     \leavevmode
%     \hbox to \linewidth{^^A
%       \kern-\leftmargin\relax
%       \hss
%       \usebox0
%       \hss
%       \kern-\rightmargin\relax
%     }^^A
%   }^^A
%   \ifdim\wd0>\lw
%     \sbox0{\small\t}^^A
%     \ifdim\wd0>\linewidth
%       \ifdim\wd0>\lw
%         \sbox0{\footnotesize\t}^^A
%         \ifdim\wd0>\linewidth
%           \ifdim\wd0>\lw
%             \sbox0{\scriptsize\t}^^A
%             \ifdim\wd0>\linewidth
%               \ifdim\wd0>\lw
%                 \sbox0{\tiny\t}^^A
%                 \ifdim\wd0>\linewidth
%                   \lwbox
%                 \else
%                   \usebox0
%                 \fi
%               \else
%                 \lwbox
%               \fi
%             \else
%               \usebox0
%             \fi
%           \else
%             \lwbox
%           \fi
%         \else
%           \usebox0
%         \fi
%       \else
%         \lwbox
%       \fi
%     \else
%       \usebox0
%     \fi
%   \else
%     \lwbox
%   \fi
% \else
%   \usebox0
% \fi
% \end{quote}
% If you have a \xfile{docstrip.cfg} that configures and enables \docstrip's
% TDS installing feature, then some files can already be in the right
% place, see the documentation of \docstrip.
%
% \subsection{Refresh file name databases}
%
% If your \TeX~distribution
% (\teTeX, \mikTeX, \dots) relies on file name databases, you must refresh
% these. For example, \teTeX\ users run \verb|texhash| or
% \verb|mktexlsr|.
%
% \subsection{Some details for the interested}
%
% \paragraph{Attached source.}
%
% The PDF documentation on CTAN also includes the
% \xfile{.dtx} source file. It can be extracted by
% AcrobatReader 6 or higher. Another option is \textsf{pdftk},
% e.g. unpack the file into the current directory:
% \begin{quote}
%   \verb|pdftk makerobust.pdf unpack_files output .|
% \end{quote}
%
% \paragraph{Unpacking with \LaTeX.}
% The \xfile{.dtx} chooses its action depending on the format:
% \begin{description}
% \item[\plainTeX:] Run \docstrip\ and extract the files.
% \item[\LaTeX:] Generate the documentation.
% \end{description}
% If you insist on using \LaTeX\ for \docstrip\ (really,
% \docstrip\ does not need \LaTeX), then inform the autodetect routine
% about your intention:
% \begin{quote}
%   \verb|latex \let\install=y\input{makerobust.dtx}|
% \end{quote}
% Do not forget to quote the argument according to the demands
% of your shell.
%
% \paragraph{Generating the documentation.}
% You can use both the \xfile{.dtx} or the \xfile{.drv} to generate
% the documentation. The process can be configured by the
% configuration file \xfile{ltxdoc.cfg}. For instance, put this
% line into this file, if you want to have A4 as paper format:
% \begin{quote}
%   \verb|\PassOptionsToClass{a4paper}{article}|
% \end{quote}
% An example follows how to generate the
% documentation with pdf\LaTeX:
% \begin{quote}
%\begin{verbatim}
%pdflatex makerobust.dtx
%makeindex -s gind.ist makerobust.idx
%pdflatex makerobust.dtx
%makeindex -s gind.ist makerobust.idx
%pdflatex makerobust.dtx
%\end{verbatim}
% \end{quote}
%
% \section{Catalogue}
%
% The following XML file can be used as source for the
% \href{http://mirror.ctan.org/help/Catalogue/catalogue.html}{\TeX\ Catalogue}.
% The elements \texttt{caption} and \texttt{description} are imported
% from the original XML file from the Catalogue.
% The name of the XML file in the Catalogue is \xfile{makerobust.xml}.
%    \begin{macrocode}
%<*catalogue>
<?xml version='1.0' encoding='us-ascii'?>
<!DOCTYPE entry SYSTEM 'catalogue.dtd'>
<entry datestamp='$Date$' modifier='$Author$' id='makerobust'>
  <name>makerobust</name>
  <caption>Making a macro robust.</caption>
  <authorref id='auth:oberdiek'/>
  <copyright owner='Heiko Oberdiek' year='2006'/>
  <license type='lppl1.3'/>
  <version number='1.0'/>
  <description>
    This package provides the command MakeRobustCommand
    that converts an existing macro to a robust one.
    <p/>
    The package is part of the <xref refid='oberdiek'>oberdiek</xref>
    bundle.
  </description>
  <documentation details='Package documentation'
      href='ctan:/macros/latex/contrib/oberdiek/makerobust.pdf'/>
  <ctan file='true' path='/macros/latex/contrib/oberdiek/makerobust.dtx'/>
  <miktex location='oberdiek'/>
  <texlive location='oberdiek'/>
  <install path='/macros/latex/contrib/oberdiek/oberdiek.tds.zip'/>
</entry>
%</catalogue>
%    \end{macrocode}
%
% \begin{History}
%   \begin{Version}{2006/03/18 v1.0}
%   \item
%     First version.
%   \end{Version}
% \end{History}
%
% \PrintIndex
%
% \Finale
\endinput
|
% \end{quote}
% Do not forget to quote the argument according to the demands
% of your shell.
%
% \paragraph{Generating the documentation.}
% You can use both the \xfile{.dtx} or the \xfile{.drv} to generate
% the documentation. The process can be configured by the
% configuration file \xfile{ltxdoc.cfg}. For instance, put this
% line into this file, if you want to have A4 as paper format:
% \begin{quote}
%   \verb|\PassOptionsToClass{a4paper}{article}|
% \end{quote}
% An example follows how to generate the
% documentation with pdf\LaTeX:
% \begin{quote}
%\begin{verbatim}
%pdflatex makerobust.dtx
%makeindex -s gind.ist makerobust.idx
%pdflatex makerobust.dtx
%makeindex -s gind.ist makerobust.idx
%pdflatex makerobust.dtx
%\end{verbatim}
% \end{quote}
%
% \section{Catalogue}
%
% The following XML file can be used as source for the
% \href{http://mirror.ctan.org/help/Catalogue/catalogue.html}{\TeX\ Catalogue}.
% The elements \texttt{caption} and \texttt{description} are imported
% from the original XML file from the Catalogue.
% The name of the XML file in the Catalogue is \xfile{makerobust.xml}.
%    \begin{macrocode}
%<*catalogue>
<?xml version='1.0' encoding='us-ascii'?>
<!DOCTYPE entry SYSTEM 'catalogue.dtd'>
<entry datestamp='$Date$' modifier='$Author$' id='makerobust'>
  <name>makerobust</name>
  <caption>Making a macro robust.</caption>
  <authorref id='auth:oberdiek'/>
  <copyright owner='Heiko Oberdiek' year='2006'/>
  <license type='lppl1.3'/>
  <version number='1.0'/>
  <description>
    This package provides the command MakeRobustCommand
    that converts an existing macro to a robust one.
    <p/>
    The package is part of the <xref refid='oberdiek'>oberdiek</xref>
    bundle.
  </description>
  <documentation details='Package documentation'
      href='ctan:/macros/latex/contrib/oberdiek/makerobust.pdf'/>
  <ctan file='true' path='/macros/latex/contrib/oberdiek/makerobust.dtx'/>
  <miktex location='oberdiek'/>
  <texlive location='oberdiek'/>
  <install path='/macros/latex/contrib/oberdiek/oberdiek.tds.zip'/>
</entry>
%</catalogue>
%    \end{macrocode}
%
% \begin{History}
%   \begin{Version}{2006/03/18 v1.0}
%   \item
%     First version.
%   \end{Version}
% \end{History}
%
% \PrintIndex
%
% \Finale
\endinput
|
% \end{quote}
% Do not forget to quote the argument according to the demands
% of your shell.
%
% \paragraph{Generating the documentation.}
% You can use both the \xfile{.dtx} or the \xfile{.drv} to generate
% the documentation. The process can be configured by the
% configuration file \xfile{ltxdoc.cfg}. For instance, put this
% line into this file, if you want to have A4 as paper format:
% \begin{quote}
%   \verb|\PassOptionsToClass{a4paper}{article}|
% \end{quote}
% An example follows how to generate the
% documentation with pdf\LaTeX:
% \begin{quote}
%\begin{verbatim}
%pdflatex makerobust.dtx
%makeindex -s gind.ist makerobust.idx
%pdflatex makerobust.dtx
%makeindex -s gind.ist makerobust.idx
%pdflatex makerobust.dtx
%\end{verbatim}
% \end{quote}
%
% \section{Catalogue}
%
% The following XML file can be used as source for the
% \href{http://mirror.ctan.org/help/Catalogue/catalogue.html}{\TeX\ Catalogue}.
% The elements \texttt{caption} and \texttt{description} are imported
% from the original XML file from the Catalogue.
% The name of the XML file in the Catalogue is \xfile{makerobust.xml}.
%    \begin{macrocode}
%<*catalogue>
<?xml version='1.0' encoding='us-ascii'?>
<!DOCTYPE entry SYSTEM 'catalogue.dtd'>
<entry datestamp='$Date$' modifier='$Author$' id='makerobust'>
  <name>makerobust</name>
  <caption>Making a macro robust.</caption>
  <authorref id='auth:oberdiek'/>
  <copyright owner='Heiko Oberdiek' year='2006'/>
  <license type='lppl1.3'/>
  <version number='1.0'/>
  <description>
    This package provides the command MakeRobustCommand
    that converts an existing macro to a robust one.
    <p/>
    The package is part of the <xref refid='oberdiek'>oberdiek</xref>
    bundle.
  </description>
  <documentation details='Package documentation'
      href='ctan:/macros/latex/contrib/oberdiek/makerobust.pdf'/>
  <ctan file='true' path='/macros/latex/contrib/oberdiek/makerobust.dtx'/>
  <miktex location='oberdiek'/>
  <texlive location='oberdiek'/>
  <install path='/macros/latex/contrib/oberdiek/oberdiek.tds.zip'/>
</entry>
%</catalogue>
%    \end{macrocode}
%
% \begin{History}
%   \begin{Version}{2006/03/18 v1.0}
%   \item
%     First version.
%   \end{Version}
% \end{History}
%
% \PrintIndex
%
% \Finale
\endinput

%        (quote the arguments according to the demands of your shell)
%
% Documentation:
%    (a) If makerobust.drv is present:
%           latex makerobust.drv
%    (b) Without makerobust.drv:
%           latex makerobust.dtx; ...
%    The class ltxdoc loads the configuration file ltxdoc.cfg
%    if available. Here you can specify further options, e.g.
%    use A4 as paper format:
%       \PassOptionsToClass{a4paper}{article}
%
%    Programm calls to get the documentation (example):
%       pdflatex makerobust.dtx
%       makeindex -s gind.ist makerobust.idx
%       pdflatex makerobust.dtx
%       makeindex -s gind.ist makerobust.idx
%       pdflatex makerobust.dtx
%
% Installation:
%    TDS:tex/latex/oberdiek/makerobust.sty
%    TDS:doc/latex/oberdiek/makerobust.pdf
%    TDS:doc/latex/oberdiek/makerobust-example.tex
%    TDS:source/latex/oberdiek/makerobust.dtx
%
%<*ignore>
\begingroup
  \catcode123=1 %
  \catcode125=2 %
  \def\x{LaTeX2e}%
\expandafter\endgroup
\ifcase 0\ifx\install y1\fi\expandafter
         \ifx\csname processbatchFile\endcsname\relax\else1\fi
         \ifx\fmtname\x\else 1\fi\relax
\else\csname fi\endcsname
%</ignore>
%<*install>
\input docstrip.tex
\Msg{************************************************************************}
\Msg{* Installation}
\Msg{* Package: makerobust 2006/03/18 v1.0 Make existing macro robust (HO)}
\Msg{************************************************************************}

\keepsilent
\askforoverwritefalse

\let\MetaPrefix\relax
\preamble

This is a generated file.

Project: makerobust
Version: 2006/03/18 v1.0

Copyright (C) 2006 by
   Heiko Oberdiek <heiko.oberdiek at googlemail.com>

This work may be distributed and/or modified under the
conditions of the LaTeX Project Public License, either
version 1.3c of this license or (at your option) any later
version. This version of this license is in
   http://www.latex-project.org/lppl/lppl-1-3c.txt
and the latest version of this license is in
   http://www.latex-project.org/lppl.txt
and version 1.3 or later is part of all distributions of
LaTeX version 2005/12/01 or later.

This work has the LPPL maintenance status "maintained".

This Current Maintainer of this work is Heiko Oberdiek.

This work consists of the main source file makerobust.dtx
and the derived files
   makerobust.sty, makerobust.pdf, makerobust.ins, makerobust.drv,
   makerobust-example.tex.

\endpreamble
\let\MetaPrefix\DoubleperCent

\generate{%
  \file{makerobust.ins}{\from{makerobust.dtx}{install}}%
  \file{makerobust.drv}{\from{makerobust.dtx}{driver}}%
  \usedir{tex/latex/oberdiek}%
  \file{makerobust.sty}{\from{makerobust.dtx}{package}}%
  \usedir{doc/latex/oberdiek}%
  \file{makerobust-example.tex}{\from{makerobust.dtx}{example}}%
  \nopreamble
  \nopostamble
  \usedir{source/latex/oberdiek/catalogue}%
  \file{makerobust.xml}{\from{makerobust.dtx}{catalogue}}%
}

\catcode32=13\relax% active space
\let =\space%
\Msg{************************************************************************}
\Msg{*}
\Msg{* To finish the installation you have to move the following}
\Msg{* file into a directory searched by TeX:}
\Msg{*}
\Msg{*     makerobust.sty}
\Msg{*}
\Msg{* To produce the documentation run the file `makerobust.drv'}
\Msg{* through LaTeX.}
\Msg{*}
\Msg{* Happy TeXing!}
\Msg{*}
\Msg{************************************************************************}

\endbatchfile
%</install>
%<*ignore>
\fi
%</ignore>
%<*driver>
\NeedsTeXFormat{LaTeX2e}
\ProvidesFile{makerobust.drv}%
  [2006/03/18 v1.0 Make existing macro robust (HO)]%
\documentclass{ltxdoc}
\usepackage{holtxdoc}[2011/11/22]
\begin{document}
  \DocInput{makerobust.dtx}%
\end{document}
%</driver>
% \fi
%
% \CheckSum{59}
%
% \CharacterTable
%  {Upper-case    \A\B\C\D\E\F\G\H\I\J\K\L\M\N\O\P\Q\R\S\T\U\V\W\X\Y\Z
%   Lower-case    \a\b\c\d\e\f\g\h\i\j\k\l\m\n\o\p\q\r\s\t\u\v\w\x\y\z
%   Digits        \0\1\2\3\4\5\6\7\8\9
%   Exclamation   \!     Double quote  \"     Hash (number) \#
%   Dollar        \$     Percent       \%     Ampersand     \&
%   Acute accent  \'     Left paren    \(     Right paren   \)
%   Asterisk      \*     Plus          \+     Comma         \,
%   Minus         \-     Point         \.     Solidus       \/
%   Colon         \:     Semicolon     \;     Less than     \<
%   Equals        \=     Greater than  \>     Question mark \?
%   Commercial at \@     Left bracket  \[     Backslash     \\
%   Right bracket \]     Circumflex    \^     Underscore    \_
%   Grave accent  \`     Left brace    \{     Vertical bar  \|
%   Right brace   \}     Tilde         \~}
%
% \GetFileInfo{makerobust.drv}
%
% \title{The \xpackage{makerobust} package}
% \date{2006/03/18 v1.0}
% \author{Heiko Oberdiek\\\xemail{heiko.oberdiek at googlemail.com}}
%
% \maketitle
%
% \begin{abstract}
% Package \xpackage{makerobust} provides \cs{MakeRobustCommand}
% that converts an existing macro to a robust one.
% \end{abstract}
%
% \tableofcontents
%
% \section{User interface}
%
% \LaTeX\ offers \cs{DeclareRobustCommand} to define a robust macro
% that does not break if it is used in moving arguments.
% Sometimes a macro is already defined, but not robust. For
% example, \cs{(} and \cs{)} are not robust, inside \cs{section}
% the user must use \cs{protect} explicitly. This could be
% avoided by making \cs{(} and \cs{)} robust.
%
% \begin{declcs}{MakeRobustCommand}\M{cmd}
% \end{declcs}
% \cs{MakeRobustCommand} redefines the macro \meta{cmd}
% by using \cs{DeclareRobustCommand} and the existing definition
% of the macro \meta{cmd}.
% \begin{itemize}
% \item It is an error if \meta{cmd} is undefined. If you want to
%   define a robust command, then you can use \cs{DeclareRobustCommand}
%   directly.
% \item If the macro has previously been
%   defined by \cs{DeclareRobustCommand} then the redefinition of
%   \cs{MakeRobustCommand} is omitted, because the macro is already robust.
%   Only an information entry is written to the \xfile{.log} file.
%   Thus you do not get a warning or an error if the macro is already
%   robust because of an updated LaTeX or package that defines the macro.
% \item Two macros are defined for a macro, defined
%   by \cs{DeclareRobustCommand}. Example:
%   \begin{quote}
%   |\DeclareRobustCommand{\foobar}{definition text}|
%   \end{quote}
%   Then the macro ``\cs{foobar}'' contains the protection code
%   and, depending on the protection mode,
%   calls the internal macro ``\cs{foobar }''. Notice the space
%   at the end of the macro name.
%   This internal macro ``\cs{foobar }'' now contains the definition
%   ``|definition text|'', given in \cs{DeclareRobustCommand}.
%
%   Sometimes it can happen, that the internal macro already exists.
%   This can be caused by a previous \cs{DeclareRobustCommand} followed
%   by \cs{renewcommand}. Then the redefinition by \cs{MakeRobustCommand}
%   would be safe.
%
%   However, it can also be possible that the macro is already robust,
%   using the internal macro, but with a different protection code.
%   The redefinition by \cs{MakeRobustCommand} would then generate
%   an infinite loop.
%
%   Therefore \cs{MakeRobustCommand} raises an error message,
%   if the internal macro (with space at the end) already exists.
% \end{itemize}
%
% \subsection{Example}
%
%    \begin{macrocode}
%<*example>
\documentclass{article}
\usepackage{makerobust}
\MakeRobustCommand\(
\MakeRobustCommand\)
\pagestyle{headings}
\begin{document}
\tableofcontents
\section{Einstein: \(E=mc^2\)}
\newpage
Second page.
\end{document}
%</example>
%    \end{macrocode}
%
%
% \StopEventually{
% }
%
% \section{Implementation}
%
%    \begin{macrocode}
%<*package>
\NeedsTeXFormat{LaTeX2e}
\ProvidesPackage{makerobust}%
  [2006/03/18 v1.0 Make existing macro robust (HO)]%
%    \end{macrocode}
%
%    \begin{macrocode}
\def\MakeRobustCommand#1{%
  \begingroup
  \@ifundefined{\expandafter\@gobble\string#1}{%
    \endgroup
    \PackageError{makerobust}{%
      Macro \string`\string#1\string' is not defined%
    }\@ehc
  }{%
    \global\let\MR@gtemp#1%
    \let#1\@undefined
    \expandafter\let\expandafter\MR@temp
        \csname\expandafter\@gobble\string#1 \endcsname
    \DeclareRobustCommand#1{}%
    \ifx#1\MR@gtemp
      \endgroup
      \PackageInfo{makerobust}{%
        \string`\string#1\string' is already robust%
      }%
    \else
      \@ifundefined{MR@temp}{%
        \global\let\MR@gtemp#1%
        \endgroup
        \expandafter\let\csname\expandafter\@gobble\string#1 \endcsname#1%
        \let#1\MR@gtemp
      }{%
        \endgroup
        \PackageError{makerobust}{%
          Internal macro \string`\string#1 \string' already exists%
        }\@ehc
      }%
    \fi
  }%
}
%    \end{macrocode}
%
%    \begin{macrocode}
%</package>
%    \end{macrocode}
%
% \section{Installation}
%
% \subsection{Download}
%
% \paragraph{Package.} This package is available on
% CTAN\footnote{\url{ftp://ftp.ctan.org/tex-archive/}}:
% \begin{description}
% \item[\CTAN{macros/latex/contrib/oberdiek/makerobust.dtx}] The source file.
% \item[\CTAN{macros/latex/contrib/oberdiek/makerobust.pdf}] Documentation.
% \end{description}
%
%
% \paragraph{Bundle.} All the packages of the bundle `oberdiek'
% are also available in a TDS compliant ZIP archive. There
% the packages are already unpacked and the documentation files
% are generated. The files and directories obey the TDS standard.
% \begin{description}
% \item[\CTAN{install/macros/latex/contrib/oberdiek.tds.zip}]
% \end{description}
% \emph{TDS} refers to the standard ``A Directory Structure
% for \TeX\ Files'' (\CTAN{tds/tds.pdf}). Directories
% with \xfile{texmf} in their name are usually organized this way.
%
% \subsection{Bundle installation}
%
% \paragraph{Unpacking.} Unpack the \xfile{oberdiek.tds.zip} in the
% TDS tree (also known as \xfile{texmf} tree) of your choice.
% Example (linux):
% \begin{quote}
%   |unzip oberdiek.tds.zip -d ~/texmf|
% \end{quote}
%
% \paragraph{Script installation.}
% Check the directory \xfile{TDS:scripts/oberdiek/} for
% scripts that need further installation steps.
% Package \xpackage{attachfile2} comes with the Perl script
% \xfile{pdfatfi.pl} that should be installed in such a way
% that it can be called as \texttt{pdfatfi}.
% Example (linux):
% \begin{quote}
%   |chmod +x scripts/oberdiek/pdfatfi.pl|\\
%   |cp scripts/oberdiek/pdfatfi.pl /usr/local/bin/|
% \end{quote}
%
% \subsection{Package installation}
%
% \paragraph{Unpacking.} The \xfile{.dtx} file is a self-extracting
% \docstrip\ archive. The files are extracted by running the
% \xfile{.dtx} through \plainTeX:
% \begin{quote}
%   \verb|tex makerobust.dtx|
% \end{quote}
%
% \paragraph{TDS.} Now the different files must be moved into
% the different directories in your installation TDS tree
% (also known as \xfile{texmf} tree):
% \begin{quote}
% \def\t{^^A
% \begin{tabular}{@{}>{\ttfamily}l@{ $\rightarrow$ }>{\ttfamily}l@{}}
%   makerobust.sty & tex/latex/oberdiek/makerobust.sty\\
%   makerobust.pdf & doc/latex/oberdiek/makerobust.pdf\\
%   makerobust-example.tex & doc/latex/oberdiek/makerobust-example.tex\\
%   makerobust.dtx & source/latex/oberdiek/makerobust.dtx\\
% \end{tabular}^^A
% }^^A
% \sbox0{\t}^^A
% \ifdim\wd0>\linewidth
%   \begingroup
%     \advance\linewidth by\leftmargin
%     \advance\linewidth by\rightmargin
%   \edef\x{\endgroup
%     \def\noexpand\lw{\the\linewidth}^^A
%   }\x
%   \def\lwbox{^^A
%     \leavevmode
%     \hbox to \linewidth{^^A
%       \kern-\leftmargin\relax
%       \hss
%       \usebox0
%       \hss
%       \kern-\rightmargin\relax
%     }^^A
%   }^^A
%   \ifdim\wd0>\lw
%     \sbox0{\small\t}^^A
%     \ifdim\wd0>\linewidth
%       \ifdim\wd0>\lw
%         \sbox0{\footnotesize\t}^^A
%         \ifdim\wd0>\linewidth
%           \ifdim\wd0>\lw
%             \sbox0{\scriptsize\t}^^A
%             \ifdim\wd0>\linewidth
%               \ifdim\wd0>\lw
%                 \sbox0{\tiny\t}^^A
%                 \ifdim\wd0>\linewidth
%                   \lwbox
%                 \else
%                   \usebox0
%                 \fi
%               \else
%                 \lwbox
%               \fi
%             \else
%               \usebox0
%             \fi
%           \else
%             \lwbox
%           \fi
%         \else
%           \usebox0
%         \fi
%       \else
%         \lwbox
%       \fi
%     \else
%       \usebox0
%     \fi
%   \else
%     \lwbox
%   \fi
% \else
%   \usebox0
% \fi
% \end{quote}
% If you have a \xfile{docstrip.cfg} that configures and enables \docstrip's
% TDS installing feature, then some files can already be in the right
% place, see the documentation of \docstrip.
%
% \subsection{Refresh file name databases}
%
% If your \TeX~distribution
% (\teTeX, \mikTeX, \dots) relies on file name databases, you must refresh
% these. For example, \teTeX\ users run \verb|texhash| or
% \verb|mktexlsr|.
%
% \subsection{Some details for the interested}
%
% \paragraph{Attached source.}
%
% The PDF documentation on CTAN also includes the
% \xfile{.dtx} source file. It can be extracted by
% AcrobatReader 6 or higher. Another option is \textsf{pdftk},
% e.g. unpack the file into the current directory:
% \begin{quote}
%   \verb|pdftk makerobust.pdf unpack_files output .|
% \end{quote}
%
% \paragraph{Unpacking with \LaTeX.}
% The \xfile{.dtx} chooses its action depending on the format:
% \begin{description}
% \item[\plainTeX:] Run \docstrip\ and extract the files.
% \item[\LaTeX:] Generate the documentation.
% \end{description}
% If you insist on using \LaTeX\ for \docstrip\ (really,
% \docstrip\ does not need \LaTeX), then inform the autodetect routine
% about your intention:
% \begin{quote}
%   \verb|latex \let\install=y% \iffalse meta-comment
%
% File: makerobust.dtx
% Version: 2006/03/18 v1.0
% Info: Make existing macro robust
%
% Copyright (C) 2006 by
%    Heiko Oberdiek <heiko.oberdiek at googlemail.com>
%
% This work may be distributed and/or modified under the
% conditions of the LaTeX Project Public License, either
% version 1.3c of this license or (at your option) any later
% version. This version of this license is in
%    http://www.latex-project.org/lppl/lppl-1-3c.txt
% and the latest version of this license is in
%    http://www.latex-project.org/lppl.txt
% and version 1.3 or later is part of all distributions of
% LaTeX version 2005/12/01 or later.
%
% This work has the LPPL maintenance status "maintained".
%
% This Current Maintainer of this work is Heiko Oberdiek.
%
% This work consists of the main source file makerobust.dtx
% and the derived files
%    makerobust.sty, makerobust.pdf, makerobust.ins, makerobust.drv,
%    makerobust-example.tex.
%
% Distribution:
%    CTAN:macros/latex/contrib/oberdiek/makerobust.dtx
%    CTAN:macros/latex/contrib/oberdiek/makerobust.pdf
%
% Unpacking:
%    (a) If makerobust.ins is present:
%           tex makerobust.ins
%    (b) Without makerobust.ins:
%           tex makerobust.dtx
%    (c) If you insist on using LaTeX
%           latex \let\install=y% \iffalse meta-comment
%
% File: makerobust.dtx
% Version: 2006/03/18 v1.0
% Info: Make existing macro robust
%
% Copyright (C) 2006 by
%    Heiko Oberdiek <heiko.oberdiek at googlemail.com>
%
% This work may be distributed and/or modified under the
% conditions of the LaTeX Project Public License, either
% version 1.3c of this license or (at your option) any later
% version. This version of this license is in
%    http://www.latex-project.org/lppl/lppl-1-3c.txt
% and the latest version of this license is in
%    http://www.latex-project.org/lppl.txt
% and version 1.3 or later is part of all distributions of
% LaTeX version 2005/12/01 or later.
%
% This work has the LPPL maintenance status "maintained".
%
% This Current Maintainer of this work is Heiko Oberdiek.
%
% This work consists of the main source file makerobust.dtx
% and the derived files
%    makerobust.sty, makerobust.pdf, makerobust.ins, makerobust.drv,
%    makerobust-example.tex.
%
% Distribution:
%    CTAN:macros/latex/contrib/oberdiek/makerobust.dtx
%    CTAN:macros/latex/contrib/oberdiek/makerobust.pdf
%
% Unpacking:
%    (a) If makerobust.ins is present:
%           tex makerobust.ins
%    (b) Without makerobust.ins:
%           tex makerobust.dtx
%    (c) If you insist on using LaTeX
%           latex \let\install=y% \iffalse meta-comment
%
% File: makerobust.dtx
% Version: 2006/03/18 v1.0
% Info: Make existing macro robust
%
% Copyright (C) 2006 by
%    Heiko Oberdiek <heiko.oberdiek at googlemail.com>
%
% This work may be distributed and/or modified under the
% conditions of the LaTeX Project Public License, either
% version 1.3c of this license or (at your option) any later
% version. This version of this license is in
%    http://www.latex-project.org/lppl/lppl-1-3c.txt
% and the latest version of this license is in
%    http://www.latex-project.org/lppl.txt
% and version 1.3 or later is part of all distributions of
% LaTeX version 2005/12/01 or later.
%
% This work has the LPPL maintenance status "maintained".
%
% This Current Maintainer of this work is Heiko Oberdiek.
%
% This work consists of the main source file makerobust.dtx
% and the derived files
%    makerobust.sty, makerobust.pdf, makerobust.ins, makerobust.drv,
%    makerobust-example.tex.
%
% Distribution:
%    CTAN:macros/latex/contrib/oberdiek/makerobust.dtx
%    CTAN:macros/latex/contrib/oberdiek/makerobust.pdf
%
% Unpacking:
%    (a) If makerobust.ins is present:
%           tex makerobust.ins
%    (b) Without makerobust.ins:
%           tex makerobust.dtx
%    (c) If you insist on using LaTeX
%           latex \let\install=y\input{makerobust.dtx}
%        (quote the arguments according to the demands of your shell)
%
% Documentation:
%    (a) If makerobust.drv is present:
%           latex makerobust.drv
%    (b) Without makerobust.drv:
%           latex makerobust.dtx; ...
%    The class ltxdoc loads the configuration file ltxdoc.cfg
%    if available. Here you can specify further options, e.g.
%    use A4 as paper format:
%       \PassOptionsToClass{a4paper}{article}
%
%    Programm calls to get the documentation (example):
%       pdflatex makerobust.dtx
%       makeindex -s gind.ist makerobust.idx
%       pdflatex makerobust.dtx
%       makeindex -s gind.ist makerobust.idx
%       pdflatex makerobust.dtx
%
% Installation:
%    TDS:tex/latex/oberdiek/makerobust.sty
%    TDS:doc/latex/oberdiek/makerobust.pdf
%    TDS:doc/latex/oberdiek/makerobust-example.tex
%    TDS:source/latex/oberdiek/makerobust.dtx
%
%<*ignore>
\begingroup
  \catcode123=1 %
  \catcode125=2 %
  \def\x{LaTeX2e}%
\expandafter\endgroup
\ifcase 0\ifx\install y1\fi\expandafter
         \ifx\csname processbatchFile\endcsname\relax\else1\fi
         \ifx\fmtname\x\else 1\fi\relax
\else\csname fi\endcsname
%</ignore>
%<*install>
\input docstrip.tex
\Msg{************************************************************************}
\Msg{* Installation}
\Msg{* Package: makerobust 2006/03/18 v1.0 Make existing macro robust (HO)}
\Msg{************************************************************************}

\keepsilent
\askforoverwritefalse

\let\MetaPrefix\relax
\preamble

This is a generated file.

Project: makerobust
Version: 2006/03/18 v1.0

Copyright (C) 2006 by
   Heiko Oberdiek <heiko.oberdiek at googlemail.com>

This work may be distributed and/or modified under the
conditions of the LaTeX Project Public License, either
version 1.3c of this license or (at your option) any later
version. This version of this license is in
   http://www.latex-project.org/lppl/lppl-1-3c.txt
and the latest version of this license is in
   http://www.latex-project.org/lppl.txt
and version 1.3 or later is part of all distributions of
LaTeX version 2005/12/01 or later.

This work has the LPPL maintenance status "maintained".

This Current Maintainer of this work is Heiko Oberdiek.

This work consists of the main source file makerobust.dtx
and the derived files
   makerobust.sty, makerobust.pdf, makerobust.ins, makerobust.drv,
   makerobust-example.tex.

\endpreamble
\let\MetaPrefix\DoubleperCent

\generate{%
  \file{makerobust.ins}{\from{makerobust.dtx}{install}}%
  \file{makerobust.drv}{\from{makerobust.dtx}{driver}}%
  \usedir{tex/latex/oberdiek}%
  \file{makerobust.sty}{\from{makerobust.dtx}{package}}%
  \usedir{doc/latex/oberdiek}%
  \file{makerobust-example.tex}{\from{makerobust.dtx}{example}}%
  \nopreamble
  \nopostamble
  \usedir{source/latex/oberdiek/catalogue}%
  \file{makerobust.xml}{\from{makerobust.dtx}{catalogue}}%
}

\catcode32=13\relax% active space
\let =\space%
\Msg{************************************************************************}
\Msg{*}
\Msg{* To finish the installation you have to move the following}
\Msg{* file into a directory searched by TeX:}
\Msg{*}
\Msg{*     makerobust.sty}
\Msg{*}
\Msg{* To produce the documentation run the file `makerobust.drv'}
\Msg{* through LaTeX.}
\Msg{*}
\Msg{* Happy TeXing!}
\Msg{*}
\Msg{************************************************************************}

\endbatchfile
%</install>
%<*ignore>
\fi
%</ignore>
%<*driver>
\NeedsTeXFormat{LaTeX2e}
\ProvidesFile{makerobust.drv}%
  [2006/03/18 v1.0 Make existing macro robust (HO)]%
\documentclass{ltxdoc}
\usepackage{holtxdoc}[2011/11/22]
\begin{document}
  \DocInput{makerobust.dtx}%
\end{document}
%</driver>
% \fi
%
% \CheckSum{59}
%
% \CharacterTable
%  {Upper-case    \A\B\C\D\E\F\G\H\I\J\K\L\M\N\O\P\Q\R\S\T\U\V\W\X\Y\Z
%   Lower-case    \a\b\c\d\e\f\g\h\i\j\k\l\m\n\o\p\q\r\s\t\u\v\w\x\y\z
%   Digits        \0\1\2\3\4\5\6\7\8\9
%   Exclamation   \!     Double quote  \"     Hash (number) \#
%   Dollar        \$     Percent       \%     Ampersand     \&
%   Acute accent  \'     Left paren    \(     Right paren   \)
%   Asterisk      \*     Plus          \+     Comma         \,
%   Minus         \-     Point         \.     Solidus       \/
%   Colon         \:     Semicolon     \;     Less than     \<
%   Equals        \=     Greater than  \>     Question mark \?
%   Commercial at \@     Left bracket  \[     Backslash     \\
%   Right bracket \]     Circumflex    \^     Underscore    \_
%   Grave accent  \`     Left brace    \{     Vertical bar  \|
%   Right brace   \}     Tilde         \~}
%
% \GetFileInfo{makerobust.drv}
%
% \title{The \xpackage{makerobust} package}
% \date{2006/03/18 v1.0}
% \author{Heiko Oberdiek\\\xemail{heiko.oberdiek at googlemail.com}}
%
% \maketitle
%
% \begin{abstract}
% Package \xpackage{makerobust} provides \cs{MakeRobustCommand}
% that converts an existing macro to a robust one.
% \end{abstract}
%
% \tableofcontents
%
% \section{User interface}
%
% \LaTeX\ offers \cs{DeclareRobustCommand} to define a robust macro
% that does not break if it is used in moving arguments.
% Sometimes a macro is already defined, but not robust. For
% example, \cs{(} and \cs{)} are not robust, inside \cs{section}
% the user must use \cs{protect} explicitly. This could be
% avoided by making \cs{(} and \cs{)} robust.
%
% \begin{declcs}{MakeRobustCommand}\M{cmd}
% \end{declcs}
% \cs{MakeRobustCommand} redefines the macro \meta{cmd}
% by using \cs{DeclareRobustCommand} and the existing definition
% of the macro \meta{cmd}.
% \begin{itemize}
% \item It is an error if \meta{cmd} is undefined. If you want to
%   define a robust command, then you can use \cs{DeclareRobustCommand}
%   directly.
% \item If the macro has previously been
%   defined by \cs{DeclareRobustCommand} then the redefinition of
%   \cs{MakeRobustCommand} is omitted, because the macro is already robust.
%   Only an information entry is written to the \xfile{.log} file.
%   Thus you do not get a warning or an error if the macro is already
%   robust because of an updated LaTeX or package that defines the macro.
% \item Two macros are defined for a macro, defined
%   by \cs{DeclareRobustCommand}. Example:
%   \begin{quote}
%   |\DeclareRobustCommand{\foobar}{definition text}|
%   \end{quote}
%   Then the macro ``\cs{foobar}'' contains the protection code
%   and, depending on the protection mode,
%   calls the internal macro ``\cs{foobar }''. Notice the space
%   at the end of the macro name.
%   This internal macro ``\cs{foobar }'' now contains the definition
%   ``|definition text|'', given in \cs{DeclareRobustCommand}.
%
%   Sometimes it can happen, that the internal macro already exists.
%   This can be caused by a previous \cs{DeclareRobustCommand} followed
%   by \cs{renewcommand}. Then the redefinition by \cs{MakeRobustCommand}
%   would be safe.
%
%   However, it can also be possible that the macro is already robust,
%   using the internal macro, but with a different protection code.
%   The redefinition by \cs{MakeRobustCommand} would then generate
%   an infinite loop.
%
%   Therefore \cs{MakeRobustCommand} raises an error message,
%   if the internal macro (with space at the end) already exists.
% \end{itemize}
%
% \subsection{Example}
%
%    \begin{macrocode}
%<*example>
\documentclass{article}
\usepackage{makerobust}
\MakeRobustCommand\(
\MakeRobustCommand\)
\pagestyle{headings}
\begin{document}
\tableofcontents
\section{Einstein: \(E=mc^2\)}
\newpage
Second page.
\end{document}
%</example>
%    \end{macrocode}
%
%
% \StopEventually{
% }
%
% \section{Implementation}
%
%    \begin{macrocode}
%<*package>
\NeedsTeXFormat{LaTeX2e}
\ProvidesPackage{makerobust}%
  [2006/03/18 v1.0 Make existing macro robust (HO)]%
%    \end{macrocode}
%
%    \begin{macrocode}
\def\MakeRobustCommand#1{%
  \begingroup
  \@ifundefined{\expandafter\@gobble\string#1}{%
    \endgroup
    \PackageError{makerobust}{%
      Macro \string`\string#1\string' is not defined%
    }\@ehc
  }{%
    \global\let\MR@gtemp#1%
    \let#1\@undefined
    \expandafter\let\expandafter\MR@temp
        \csname\expandafter\@gobble\string#1 \endcsname
    \DeclareRobustCommand#1{}%
    \ifx#1\MR@gtemp
      \endgroup
      \PackageInfo{makerobust}{%
        \string`\string#1\string' is already robust%
      }%
    \else
      \@ifundefined{MR@temp}{%
        \global\let\MR@gtemp#1%
        \endgroup
        \expandafter\let\csname\expandafter\@gobble\string#1 \endcsname#1%
        \let#1\MR@gtemp
      }{%
        \endgroup
        \PackageError{makerobust}{%
          Internal macro \string`\string#1 \string' already exists%
        }\@ehc
      }%
    \fi
  }%
}
%    \end{macrocode}
%
%    \begin{macrocode}
%</package>
%    \end{macrocode}
%
% \section{Installation}
%
% \subsection{Download}
%
% \paragraph{Package.} This package is available on
% CTAN\footnote{\url{ftp://ftp.ctan.org/tex-archive/}}:
% \begin{description}
% \item[\CTAN{macros/latex/contrib/oberdiek/makerobust.dtx}] The source file.
% \item[\CTAN{macros/latex/contrib/oberdiek/makerobust.pdf}] Documentation.
% \end{description}
%
%
% \paragraph{Bundle.} All the packages of the bundle `oberdiek'
% are also available in a TDS compliant ZIP archive. There
% the packages are already unpacked and the documentation files
% are generated. The files and directories obey the TDS standard.
% \begin{description}
% \item[\CTAN{install/macros/latex/contrib/oberdiek.tds.zip}]
% \end{description}
% \emph{TDS} refers to the standard ``A Directory Structure
% for \TeX\ Files'' (\CTAN{tds/tds.pdf}). Directories
% with \xfile{texmf} in their name are usually organized this way.
%
% \subsection{Bundle installation}
%
% \paragraph{Unpacking.} Unpack the \xfile{oberdiek.tds.zip} in the
% TDS tree (also known as \xfile{texmf} tree) of your choice.
% Example (linux):
% \begin{quote}
%   |unzip oberdiek.tds.zip -d ~/texmf|
% \end{quote}
%
% \paragraph{Script installation.}
% Check the directory \xfile{TDS:scripts/oberdiek/} for
% scripts that need further installation steps.
% Package \xpackage{attachfile2} comes with the Perl script
% \xfile{pdfatfi.pl} that should be installed in such a way
% that it can be called as \texttt{pdfatfi}.
% Example (linux):
% \begin{quote}
%   |chmod +x scripts/oberdiek/pdfatfi.pl|\\
%   |cp scripts/oberdiek/pdfatfi.pl /usr/local/bin/|
% \end{quote}
%
% \subsection{Package installation}
%
% \paragraph{Unpacking.} The \xfile{.dtx} file is a self-extracting
% \docstrip\ archive. The files are extracted by running the
% \xfile{.dtx} through \plainTeX:
% \begin{quote}
%   \verb|tex makerobust.dtx|
% \end{quote}
%
% \paragraph{TDS.} Now the different files must be moved into
% the different directories in your installation TDS tree
% (also known as \xfile{texmf} tree):
% \begin{quote}
% \def\t{^^A
% \begin{tabular}{@{}>{\ttfamily}l@{ $\rightarrow$ }>{\ttfamily}l@{}}
%   makerobust.sty & tex/latex/oberdiek/makerobust.sty\\
%   makerobust.pdf & doc/latex/oberdiek/makerobust.pdf\\
%   makerobust-example.tex & doc/latex/oberdiek/makerobust-example.tex\\
%   makerobust.dtx & source/latex/oberdiek/makerobust.dtx\\
% \end{tabular}^^A
% }^^A
% \sbox0{\t}^^A
% \ifdim\wd0>\linewidth
%   \begingroup
%     \advance\linewidth by\leftmargin
%     \advance\linewidth by\rightmargin
%   \edef\x{\endgroup
%     \def\noexpand\lw{\the\linewidth}^^A
%   }\x
%   \def\lwbox{^^A
%     \leavevmode
%     \hbox to \linewidth{^^A
%       \kern-\leftmargin\relax
%       \hss
%       \usebox0
%       \hss
%       \kern-\rightmargin\relax
%     }^^A
%   }^^A
%   \ifdim\wd0>\lw
%     \sbox0{\small\t}^^A
%     \ifdim\wd0>\linewidth
%       \ifdim\wd0>\lw
%         \sbox0{\footnotesize\t}^^A
%         \ifdim\wd0>\linewidth
%           \ifdim\wd0>\lw
%             \sbox0{\scriptsize\t}^^A
%             \ifdim\wd0>\linewidth
%               \ifdim\wd0>\lw
%                 \sbox0{\tiny\t}^^A
%                 \ifdim\wd0>\linewidth
%                   \lwbox
%                 \else
%                   \usebox0
%                 \fi
%               \else
%                 \lwbox
%               \fi
%             \else
%               \usebox0
%             \fi
%           \else
%             \lwbox
%           \fi
%         \else
%           \usebox0
%         \fi
%       \else
%         \lwbox
%       \fi
%     \else
%       \usebox0
%     \fi
%   \else
%     \lwbox
%   \fi
% \else
%   \usebox0
% \fi
% \end{quote}
% If you have a \xfile{docstrip.cfg} that configures and enables \docstrip's
% TDS installing feature, then some files can already be in the right
% place, see the documentation of \docstrip.
%
% \subsection{Refresh file name databases}
%
% If your \TeX~distribution
% (\teTeX, \mikTeX, \dots) relies on file name databases, you must refresh
% these. For example, \teTeX\ users run \verb|texhash| or
% \verb|mktexlsr|.
%
% \subsection{Some details for the interested}
%
% \paragraph{Attached source.}
%
% The PDF documentation on CTAN also includes the
% \xfile{.dtx} source file. It can be extracted by
% AcrobatReader 6 or higher. Another option is \textsf{pdftk},
% e.g. unpack the file into the current directory:
% \begin{quote}
%   \verb|pdftk makerobust.pdf unpack_files output .|
% \end{quote}
%
% \paragraph{Unpacking with \LaTeX.}
% The \xfile{.dtx} chooses its action depending on the format:
% \begin{description}
% \item[\plainTeX:] Run \docstrip\ and extract the files.
% \item[\LaTeX:] Generate the documentation.
% \end{description}
% If you insist on using \LaTeX\ for \docstrip\ (really,
% \docstrip\ does not need \LaTeX), then inform the autodetect routine
% about your intention:
% \begin{quote}
%   \verb|latex \let\install=y\input{makerobust.dtx}|
% \end{quote}
% Do not forget to quote the argument according to the demands
% of your shell.
%
% \paragraph{Generating the documentation.}
% You can use both the \xfile{.dtx} or the \xfile{.drv} to generate
% the documentation. The process can be configured by the
% configuration file \xfile{ltxdoc.cfg}. For instance, put this
% line into this file, if you want to have A4 as paper format:
% \begin{quote}
%   \verb|\PassOptionsToClass{a4paper}{article}|
% \end{quote}
% An example follows how to generate the
% documentation with pdf\LaTeX:
% \begin{quote}
%\begin{verbatim}
%pdflatex makerobust.dtx
%makeindex -s gind.ist makerobust.idx
%pdflatex makerobust.dtx
%makeindex -s gind.ist makerobust.idx
%pdflatex makerobust.dtx
%\end{verbatim}
% \end{quote}
%
% \section{Catalogue}
%
% The following XML file can be used as source for the
% \href{http://mirror.ctan.org/help/Catalogue/catalogue.html}{\TeX\ Catalogue}.
% The elements \texttt{caption} and \texttt{description} are imported
% from the original XML file from the Catalogue.
% The name of the XML file in the Catalogue is \xfile{makerobust.xml}.
%    \begin{macrocode}
%<*catalogue>
<?xml version='1.0' encoding='us-ascii'?>
<!DOCTYPE entry SYSTEM 'catalogue.dtd'>
<entry datestamp='$Date$' modifier='$Author$' id='makerobust'>
  <name>makerobust</name>
  <caption>Making a macro robust.</caption>
  <authorref id='auth:oberdiek'/>
  <copyright owner='Heiko Oberdiek' year='2006'/>
  <license type='lppl1.3'/>
  <version number='1.0'/>
  <description>
    This package provides the command MakeRobustCommand
    that converts an existing macro to a robust one.
    <p/>
    The package is part of the <xref refid='oberdiek'>oberdiek</xref>
    bundle.
  </description>
  <documentation details='Package documentation'
      href='ctan:/macros/latex/contrib/oberdiek/makerobust.pdf'/>
  <ctan file='true' path='/macros/latex/contrib/oberdiek/makerobust.dtx'/>
  <miktex location='oberdiek'/>
  <texlive location='oberdiek'/>
  <install path='/macros/latex/contrib/oberdiek/oberdiek.tds.zip'/>
</entry>
%</catalogue>
%    \end{macrocode}
%
% \begin{History}
%   \begin{Version}{2006/03/18 v1.0}
%   \item
%     First version.
%   \end{Version}
% \end{History}
%
% \PrintIndex
%
% \Finale
\endinput

%        (quote the arguments according to the demands of your shell)
%
% Documentation:
%    (a) If makerobust.drv is present:
%           latex makerobust.drv
%    (b) Without makerobust.drv:
%           latex makerobust.dtx; ...
%    The class ltxdoc loads the configuration file ltxdoc.cfg
%    if available. Here you can specify further options, e.g.
%    use A4 as paper format:
%       \PassOptionsToClass{a4paper}{article}
%
%    Programm calls to get the documentation (example):
%       pdflatex makerobust.dtx
%       makeindex -s gind.ist makerobust.idx
%       pdflatex makerobust.dtx
%       makeindex -s gind.ist makerobust.idx
%       pdflatex makerobust.dtx
%
% Installation:
%    TDS:tex/latex/oberdiek/makerobust.sty
%    TDS:doc/latex/oberdiek/makerobust.pdf
%    TDS:doc/latex/oberdiek/makerobust-example.tex
%    TDS:source/latex/oberdiek/makerobust.dtx
%
%<*ignore>
\begingroup
  \catcode123=1 %
  \catcode125=2 %
  \def\x{LaTeX2e}%
\expandafter\endgroup
\ifcase 0\ifx\install y1\fi\expandafter
         \ifx\csname processbatchFile\endcsname\relax\else1\fi
         \ifx\fmtname\x\else 1\fi\relax
\else\csname fi\endcsname
%</ignore>
%<*install>
\input docstrip.tex
\Msg{************************************************************************}
\Msg{* Installation}
\Msg{* Package: makerobust 2006/03/18 v1.0 Make existing macro robust (HO)}
\Msg{************************************************************************}

\keepsilent
\askforoverwritefalse

\let\MetaPrefix\relax
\preamble

This is a generated file.

Project: makerobust
Version: 2006/03/18 v1.0

Copyright (C) 2006 by
   Heiko Oberdiek <heiko.oberdiek at googlemail.com>

This work may be distributed and/or modified under the
conditions of the LaTeX Project Public License, either
version 1.3c of this license or (at your option) any later
version. This version of this license is in
   http://www.latex-project.org/lppl/lppl-1-3c.txt
and the latest version of this license is in
   http://www.latex-project.org/lppl.txt
and version 1.3 or later is part of all distributions of
LaTeX version 2005/12/01 or later.

This work has the LPPL maintenance status "maintained".

This Current Maintainer of this work is Heiko Oberdiek.

This work consists of the main source file makerobust.dtx
and the derived files
   makerobust.sty, makerobust.pdf, makerobust.ins, makerobust.drv,
   makerobust-example.tex.

\endpreamble
\let\MetaPrefix\DoubleperCent

\generate{%
  \file{makerobust.ins}{\from{makerobust.dtx}{install}}%
  \file{makerobust.drv}{\from{makerobust.dtx}{driver}}%
  \usedir{tex/latex/oberdiek}%
  \file{makerobust.sty}{\from{makerobust.dtx}{package}}%
  \usedir{doc/latex/oberdiek}%
  \file{makerobust-example.tex}{\from{makerobust.dtx}{example}}%
  \nopreamble
  \nopostamble
  \usedir{source/latex/oberdiek/catalogue}%
  \file{makerobust.xml}{\from{makerobust.dtx}{catalogue}}%
}

\catcode32=13\relax% active space
\let =\space%
\Msg{************************************************************************}
\Msg{*}
\Msg{* To finish the installation you have to move the following}
\Msg{* file into a directory searched by TeX:}
\Msg{*}
\Msg{*     makerobust.sty}
\Msg{*}
\Msg{* To produce the documentation run the file `makerobust.drv'}
\Msg{* through LaTeX.}
\Msg{*}
\Msg{* Happy TeXing!}
\Msg{*}
\Msg{************************************************************************}

\endbatchfile
%</install>
%<*ignore>
\fi
%</ignore>
%<*driver>
\NeedsTeXFormat{LaTeX2e}
\ProvidesFile{makerobust.drv}%
  [2006/03/18 v1.0 Make existing macro robust (HO)]%
\documentclass{ltxdoc}
\usepackage{holtxdoc}[2011/11/22]
\begin{document}
  \DocInput{makerobust.dtx}%
\end{document}
%</driver>
% \fi
%
% \CheckSum{59}
%
% \CharacterTable
%  {Upper-case    \A\B\C\D\E\F\G\H\I\J\K\L\M\N\O\P\Q\R\S\T\U\V\W\X\Y\Z
%   Lower-case    \a\b\c\d\e\f\g\h\i\j\k\l\m\n\o\p\q\r\s\t\u\v\w\x\y\z
%   Digits        \0\1\2\3\4\5\6\7\8\9
%   Exclamation   \!     Double quote  \"     Hash (number) \#
%   Dollar        \$     Percent       \%     Ampersand     \&
%   Acute accent  \'     Left paren    \(     Right paren   \)
%   Asterisk      \*     Plus          \+     Comma         \,
%   Minus         \-     Point         \.     Solidus       \/
%   Colon         \:     Semicolon     \;     Less than     \<
%   Equals        \=     Greater than  \>     Question mark \?
%   Commercial at \@     Left bracket  \[     Backslash     \\
%   Right bracket \]     Circumflex    \^     Underscore    \_
%   Grave accent  \`     Left brace    \{     Vertical bar  \|
%   Right brace   \}     Tilde         \~}
%
% \GetFileInfo{makerobust.drv}
%
% \title{The \xpackage{makerobust} package}
% \date{2006/03/18 v1.0}
% \author{Heiko Oberdiek\\\xemail{heiko.oberdiek at googlemail.com}}
%
% \maketitle
%
% \begin{abstract}
% Package \xpackage{makerobust} provides \cs{MakeRobustCommand}
% that converts an existing macro to a robust one.
% \end{abstract}
%
% \tableofcontents
%
% \section{User interface}
%
% \LaTeX\ offers \cs{DeclareRobustCommand} to define a robust macro
% that does not break if it is used in moving arguments.
% Sometimes a macro is already defined, but not robust. For
% example, \cs{(} and \cs{)} are not robust, inside \cs{section}
% the user must use \cs{protect} explicitly. This could be
% avoided by making \cs{(} and \cs{)} robust.
%
% \begin{declcs}{MakeRobustCommand}\M{cmd}
% \end{declcs}
% \cs{MakeRobustCommand} redefines the macro \meta{cmd}
% by using \cs{DeclareRobustCommand} and the existing definition
% of the macro \meta{cmd}.
% \begin{itemize}
% \item It is an error if \meta{cmd} is undefined. If you want to
%   define a robust command, then you can use \cs{DeclareRobustCommand}
%   directly.
% \item If the macro has previously been
%   defined by \cs{DeclareRobustCommand} then the redefinition of
%   \cs{MakeRobustCommand} is omitted, because the macro is already robust.
%   Only an information entry is written to the \xfile{.log} file.
%   Thus you do not get a warning or an error if the macro is already
%   robust because of an updated LaTeX or package that defines the macro.
% \item Two macros are defined for a macro, defined
%   by \cs{DeclareRobustCommand}. Example:
%   \begin{quote}
%   |\DeclareRobustCommand{\foobar}{definition text}|
%   \end{quote}
%   Then the macro ``\cs{foobar}'' contains the protection code
%   and, depending on the protection mode,
%   calls the internal macro ``\cs{foobar }''. Notice the space
%   at the end of the macro name.
%   This internal macro ``\cs{foobar }'' now contains the definition
%   ``|definition text|'', given in \cs{DeclareRobustCommand}.
%
%   Sometimes it can happen, that the internal macro already exists.
%   This can be caused by a previous \cs{DeclareRobustCommand} followed
%   by \cs{renewcommand}. Then the redefinition by \cs{MakeRobustCommand}
%   would be safe.
%
%   However, it can also be possible that the macro is already robust,
%   using the internal macro, but with a different protection code.
%   The redefinition by \cs{MakeRobustCommand} would then generate
%   an infinite loop.
%
%   Therefore \cs{MakeRobustCommand} raises an error message,
%   if the internal macro (with space at the end) already exists.
% \end{itemize}
%
% \subsection{Example}
%
%    \begin{macrocode}
%<*example>
\documentclass{article}
\usepackage{makerobust}
\MakeRobustCommand\(
\MakeRobustCommand\)
\pagestyle{headings}
\begin{document}
\tableofcontents
\section{Einstein: \(E=mc^2\)}
\newpage
Second page.
\end{document}
%</example>
%    \end{macrocode}
%
%
% \StopEventually{
% }
%
% \section{Implementation}
%
%    \begin{macrocode}
%<*package>
\NeedsTeXFormat{LaTeX2e}
\ProvidesPackage{makerobust}%
  [2006/03/18 v1.0 Make existing macro robust (HO)]%
%    \end{macrocode}
%
%    \begin{macrocode}
\def\MakeRobustCommand#1{%
  \begingroup
  \@ifundefined{\expandafter\@gobble\string#1}{%
    \endgroup
    \PackageError{makerobust}{%
      Macro \string`\string#1\string' is not defined%
    }\@ehc
  }{%
    \global\let\MR@gtemp#1%
    \let#1\@undefined
    \expandafter\let\expandafter\MR@temp
        \csname\expandafter\@gobble\string#1 \endcsname
    \DeclareRobustCommand#1{}%
    \ifx#1\MR@gtemp
      \endgroup
      \PackageInfo{makerobust}{%
        \string`\string#1\string' is already robust%
      }%
    \else
      \@ifundefined{MR@temp}{%
        \global\let\MR@gtemp#1%
        \endgroup
        \expandafter\let\csname\expandafter\@gobble\string#1 \endcsname#1%
        \let#1\MR@gtemp
      }{%
        \endgroup
        \PackageError{makerobust}{%
          Internal macro \string`\string#1 \string' already exists%
        }\@ehc
      }%
    \fi
  }%
}
%    \end{macrocode}
%
%    \begin{macrocode}
%</package>
%    \end{macrocode}
%
% \section{Installation}
%
% \subsection{Download}
%
% \paragraph{Package.} This package is available on
% CTAN\footnote{\url{ftp://ftp.ctan.org/tex-archive/}}:
% \begin{description}
% \item[\CTAN{macros/latex/contrib/oberdiek/makerobust.dtx}] The source file.
% \item[\CTAN{macros/latex/contrib/oberdiek/makerobust.pdf}] Documentation.
% \end{description}
%
%
% \paragraph{Bundle.} All the packages of the bundle `oberdiek'
% are also available in a TDS compliant ZIP archive. There
% the packages are already unpacked and the documentation files
% are generated. The files and directories obey the TDS standard.
% \begin{description}
% \item[\CTAN{install/macros/latex/contrib/oberdiek.tds.zip}]
% \end{description}
% \emph{TDS} refers to the standard ``A Directory Structure
% for \TeX\ Files'' (\CTAN{tds/tds.pdf}). Directories
% with \xfile{texmf} in their name are usually organized this way.
%
% \subsection{Bundle installation}
%
% \paragraph{Unpacking.} Unpack the \xfile{oberdiek.tds.zip} in the
% TDS tree (also known as \xfile{texmf} tree) of your choice.
% Example (linux):
% \begin{quote}
%   |unzip oberdiek.tds.zip -d ~/texmf|
% \end{quote}
%
% \paragraph{Script installation.}
% Check the directory \xfile{TDS:scripts/oberdiek/} for
% scripts that need further installation steps.
% Package \xpackage{attachfile2} comes with the Perl script
% \xfile{pdfatfi.pl} that should be installed in such a way
% that it can be called as \texttt{pdfatfi}.
% Example (linux):
% \begin{quote}
%   |chmod +x scripts/oberdiek/pdfatfi.pl|\\
%   |cp scripts/oberdiek/pdfatfi.pl /usr/local/bin/|
% \end{quote}
%
% \subsection{Package installation}
%
% \paragraph{Unpacking.} The \xfile{.dtx} file is a self-extracting
% \docstrip\ archive. The files are extracted by running the
% \xfile{.dtx} through \plainTeX:
% \begin{quote}
%   \verb|tex makerobust.dtx|
% \end{quote}
%
% \paragraph{TDS.} Now the different files must be moved into
% the different directories in your installation TDS tree
% (also known as \xfile{texmf} tree):
% \begin{quote}
% \def\t{^^A
% \begin{tabular}{@{}>{\ttfamily}l@{ $\rightarrow$ }>{\ttfamily}l@{}}
%   makerobust.sty & tex/latex/oberdiek/makerobust.sty\\
%   makerobust.pdf & doc/latex/oberdiek/makerobust.pdf\\
%   makerobust-example.tex & doc/latex/oberdiek/makerobust-example.tex\\
%   makerobust.dtx & source/latex/oberdiek/makerobust.dtx\\
% \end{tabular}^^A
% }^^A
% \sbox0{\t}^^A
% \ifdim\wd0>\linewidth
%   \begingroup
%     \advance\linewidth by\leftmargin
%     \advance\linewidth by\rightmargin
%   \edef\x{\endgroup
%     \def\noexpand\lw{\the\linewidth}^^A
%   }\x
%   \def\lwbox{^^A
%     \leavevmode
%     \hbox to \linewidth{^^A
%       \kern-\leftmargin\relax
%       \hss
%       \usebox0
%       \hss
%       \kern-\rightmargin\relax
%     }^^A
%   }^^A
%   \ifdim\wd0>\lw
%     \sbox0{\small\t}^^A
%     \ifdim\wd0>\linewidth
%       \ifdim\wd0>\lw
%         \sbox0{\footnotesize\t}^^A
%         \ifdim\wd0>\linewidth
%           \ifdim\wd0>\lw
%             \sbox0{\scriptsize\t}^^A
%             \ifdim\wd0>\linewidth
%               \ifdim\wd0>\lw
%                 \sbox0{\tiny\t}^^A
%                 \ifdim\wd0>\linewidth
%                   \lwbox
%                 \else
%                   \usebox0
%                 \fi
%               \else
%                 \lwbox
%               \fi
%             \else
%               \usebox0
%             \fi
%           \else
%             \lwbox
%           \fi
%         \else
%           \usebox0
%         \fi
%       \else
%         \lwbox
%       \fi
%     \else
%       \usebox0
%     \fi
%   \else
%     \lwbox
%   \fi
% \else
%   \usebox0
% \fi
% \end{quote}
% If you have a \xfile{docstrip.cfg} that configures and enables \docstrip's
% TDS installing feature, then some files can already be in the right
% place, see the documentation of \docstrip.
%
% \subsection{Refresh file name databases}
%
% If your \TeX~distribution
% (\teTeX, \mikTeX, \dots) relies on file name databases, you must refresh
% these. For example, \teTeX\ users run \verb|texhash| or
% \verb|mktexlsr|.
%
% \subsection{Some details for the interested}
%
% \paragraph{Attached source.}
%
% The PDF documentation on CTAN also includes the
% \xfile{.dtx} source file. It can be extracted by
% AcrobatReader 6 or higher. Another option is \textsf{pdftk},
% e.g. unpack the file into the current directory:
% \begin{quote}
%   \verb|pdftk makerobust.pdf unpack_files output .|
% \end{quote}
%
% \paragraph{Unpacking with \LaTeX.}
% The \xfile{.dtx} chooses its action depending on the format:
% \begin{description}
% \item[\plainTeX:] Run \docstrip\ and extract the files.
% \item[\LaTeX:] Generate the documentation.
% \end{description}
% If you insist on using \LaTeX\ for \docstrip\ (really,
% \docstrip\ does not need \LaTeX), then inform the autodetect routine
% about your intention:
% \begin{quote}
%   \verb|latex \let\install=y% \iffalse meta-comment
%
% File: makerobust.dtx
% Version: 2006/03/18 v1.0
% Info: Make existing macro robust
%
% Copyright (C) 2006 by
%    Heiko Oberdiek <heiko.oberdiek at googlemail.com>
%
% This work may be distributed and/or modified under the
% conditions of the LaTeX Project Public License, either
% version 1.3c of this license or (at your option) any later
% version. This version of this license is in
%    http://www.latex-project.org/lppl/lppl-1-3c.txt
% and the latest version of this license is in
%    http://www.latex-project.org/lppl.txt
% and version 1.3 or later is part of all distributions of
% LaTeX version 2005/12/01 or later.
%
% This work has the LPPL maintenance status "maintained".
%
% This Current Maintainer of this work is Heiko Oberdiek.
%
% This work consists of the main source file makerobust.dtx
% and the derived files
%    makerobust.sty, makerobust.pdf, makerobust.ins, makerobust.drv,
%    makerobust-example.tex.
%
% Distribution:
%    CTAN:macros/latex/contrib/oberdiek/makerobust.dtx
%    CTAN:macros/latex/contrib/oberdiek/makerobust.pdf
%
% Unpacking:
%    (a) If makerobust.ins is present:
%           tex makerobust.ins
%    (b) Without makerobust.ins:
%           tex makerobust.dtx
%    (c) If you insist on using LaTeX
%           latex \let\install=y\input{makerobust.dtx}
%        (quote the arguments according to the demands of your shell)
%
% Documentation:
%    (a) If makerobust.drv is present:
%           latex makerobust.drv
%    (b) Without makerobust.drv:
%           latex makerobust.dtx; ...
%    The class ltxdoc loads the configuration file ltxdoc.cfg
%    if available. Here you can specify further options, e.g.
%    use A4 as paper format:
%       \PassOptionsToClass{a4paper}{article}
%
%    Programm calls to get the documentation (example):
%       pdflatex makerobust.dtx
%       makeindex -s gind.ist makerobust.idx
%       pdflatex makerobust.dtx
%       makeindex -s gind.ist makerobust.idx
%       pdflatex makerobust.dtx
%
% Installation:
%    TDS:tex/latex/oberdiek/makerobust.sty
%    TDS:doc/latex/oberdiek/makerobust.pdf
%    TDS:doc/latex/oberdiek/makerobust-example.tex
%    TDS:source/latex/oberdiek/makerobust.dtx
%
%<*ignore>
\begingroup
  \catcode123=1 %
  \catcode125=2 %
  \def\x{LaTeX2e}%
\expandafter\endgroup
\ifcase 0\ifx\install y1\fi\expandafter
         \ifx\csname processbatchFile\endcsname\relax\else1\fi
         \ifx\fmtname\x\else 1\fi\relax
\else\csname fi\endcsname
%</ignore>
%<*install>
\input docstrip.tex
\Msg{************************************************************************}
\Msg{* Installation}
\Msg{* Package: makerobust 2006/03/18 v1.0 Make existing macro robust (HO)}
\Msg{************************************************************************}

\keepsilent
\askforoverwritefalse

\let\MetaPrefix\relax
\preamble

This is a generated file.

Project: makerobust
Version: 2006/03/18 v1.0

Copyright (C) 2006 by
   Heiko Oberdiek <heiko.oberdiek at googlemail.com>

This work may be distributed and/or modified under the
conditions of the LaTeX Project Public License, either
version 1.3c of this license or (at your option) any later
version. This version of this license is in
   http://www.latex-project.org/lppl/lppl-1-3c.txt
and the latest version of this license is in
   http://www.latex-project.org/lppl.txt
and version 1.3 or later is part of all distributions of
LaTeX version 2005/12/01 or later.

This work has the LPPL maintenance status "maintained".

This Current Maintainer of this work is Heiko Oberdiek.

This work consists of the main source file makerobust.dtx
and the derived files
   makerobust.sty, makerobust.pdf, makerobust.ins, makerobust.drv,
   makerobust-example.tex.

\endpreamble
\let\MetaPrefix\DoubleperCent

\generate{%
  \file{makerobust.ins}{\from{makerobust.dtx}{install}}%
  \file{makerobust.drv}{\from{makerobust.dtx}{driver}}%
  \usedir{tex/latex/oberdiek}%
  \file{makerobust.sty}{\from{makerobust.dtx}{package}}%
  \usedir{doc/latex/oberdiek}%
  \file{makerobust-example.tex}{\from{makerobust.dtx}{example}}%
  \nopreamble
  \nopostamble
  \usedir{source/latex/oberdiek/catalogue}%
  \file{makerobust.xml}{\from{makerobust.dtx}{catalogue}}%
}

\catcode32=13\relax% active space
\let =\space%
\Msg{************************************************************************}
\Msg{*}
\Msg{* To finish the installation you have to move the following}
\Msg{* file into a directory searched by TeX:}
\Msg{*}
\Msg{*     makerobust.sty}
\Msg{*}
\Msg{* To produce the documentation run the file `makerobust.drv'}
\Msg{* through LaTeX.}
\Msg{*}
\Msg{* Happy TeXing!}
\Msg{*}
\Msg{************************************************************************}

\endbatchfile
%</install>
%<*ignore>
\fi
%</ignore>
%<*driver>
\NeedsTeXFormat{LaTeX2e}
\ProvidesFile{makerobust.drv}%
  [2006/03/18 v1.0 Make existing macro robust (HO)]%
\documentclass{ltxdoc}
\usepackage{holtxdoc}[2011/11/22]
\begin{document}
  \DocInput{makerobust.dtx}%
\end{document}
%</driver>
% \fi
%
% \CheckSum{59}
%
% \CharacterTable
%  {Upper-case    \A\B\C\D\E\F\G\H\I\J\K\L\M\N\O\P\Q\R\S\T\U\V\W\X\Y\Z
%   Lower-case    \a\b\c\d\e\f\g\h\i\j\k\l\m\n\o\p\q\r\s\t\u\v\w\x\y\z
%   Digits        \0\1\2\3\4\5\6\7\8\9
%   Exclamation   \!     Double quote  \"     Hash (number) \#
%   Dollar        \$     Percent       \%     Ampersand     \&
%   Acute accent  \'     Left paren    \(     Right paren   \)
%   Asterisk      \*     Plus          \+     Comma         \,
%   Minus         \-     Point         \.     Solidus       \/
%   Colon         \:     Semicolon     \;     Less than     \<
%   Equals        \=     Greater than  \>     Question mark \?
%   Commercial at \@     Left bracket  \[     Backslash     \\
%   Right bracket \]     Circumflex    \^     Underscore    \_
%   Grave accent  \`     Left brace    \{     Vertical bar  \|
%   Right brace   \}     Tilde         \~}
%
% \GetFileInfo{makerobust.drv}
%
% \title{The \xpackage{makerobust} package}
% \date{2006/03/18 v1.0}
% \author{Heiko Oberdiek\\\xemail{heiko.oberdiek at googlemail.com}}
%
% \maketitle
%
% \begin{abstract}
% Package \xpackage{makerobust} provides \cs{MakeRobustCommand}
% that converts an existing macro to a robust one.
% \end{abstract}
%
% \tableofcontents
%
% \section{User interface}
%
% \LaTeX\ offers \cs{DeclareRobustCommand} to define a robust macro
% that does not break if it is used in moving arguments.
% Sometimes a macro is already defined, but not robust. For
% example, \cs{(} and \cs{)} are not robust, inside \cs{section}
% the user must use \cs{protect} explicitly. This could be
% avoided by making \cs{(} and \cs{)} robust.
%
% \begin{declcs}{MakeRobustCommand}\M{cmd}
% \end{declcs}
% \cs{MakeRobustCommand} redefines the macro \meta{cmd}
% by using \cs{DeclareRobustCommand} and the existing definition
% of the macro \meta{cmd}.
% \begin{itemize}
% \item It is an error if \meta{cmd} is undefined. If you want to
%   define a robust command, then you can use \cs{DeclareRobustCommand}
%   directly.
% \item If the macro has previously been
%   defined by \cs{DeclareRobustCommand} then the redefinition of
%   \cs{MakeRobustCommand} is omitted, because the macro is already robust.
%   Only an information entry is written to the \xfile{.log} file.
%   Thus you do not get a warning or an error if the macro is already
%   robust because of an updated LaTeX or package that defines the macro.
% \item Two macros are defined for a macro, defined
%   by \cs{DeclareRobustCommand}. Example:
%   \begin{quote}
%   |\DeclareRobustCommand{\foobar}{definition text}|
%   \end{quote}
%   Then the macro ``\cs{foobar}'' contains the protection code
%   and, depending on the protection mode,
%   calls the internal macro ``\cs{foobar }''. Notice the space
%   at the end of the macro name.
%   This internal macro ``\cs{foobar }'' now contains the definition
%   ``|definition text|'', given in \cs{DeclareRobustCommand}.
%
%   Sometimes it can happen, that the internal macro already exists.
%   This can be caused by a previous \cs{DeclareRobustCommand} followed
%   by \cs{renewcommand}. Then the redefinition by \cs{MakeRobustCommand}
%   would be safe.
%
%   However, it can also be possible that the macro is already robust,
%   using the internal macro, but with a different protection code.
%   The redefinition by \cs{MakeRobustCommand} would then generate
%   an infinite loop.
%
%   Therefore \cs{MakeRobustCommand} raises an error message,
%   if the internal macro (with space at the end) already exists.
% \end{itemize}
%
% \subsection{Example}
%
%    \begin{macrocode}
%<*example>
\documentclass{article}
\usepackage{makerobust}
\MakeRobustCommand\(
\MakeRobustCommand\)
\pagestyle{headings}
\begin{document}
\tableofcontents
\section{Einstein: \(E=mc^2\)}
\newpage
Second page.
\end{document}
%</example>
%    \end{macrocode}
%
%
% \StopEventually{
% }
%
% \section{Implementation}
%
%    \begin{macrocode}
%<*package>
\NeedsTeXFormat{LaTeX2e}
\ProvidesPackage{makerobust}%
  [2006/03/18 v1.0 Make existing macro robust (HO)]%
%    \end{macrocode}
%
%    \begin{macrocode}
\def\MakeRobustCommand#1{%
  \begingroup
  \@ifundefined{\expandafter\@gobble\string#1}{%
    \endgroup
    \PackageError{makerobust}{%
      Macro \string`\string#1\string' is not defined%
    }\@ehc
  }{%
    \global\let\MR@gtemp#1%
    \let#1\@undefined
    \expandafter\let\expandafter\MR@temp
        \csname\expandafter\@gobble\string#1 \endcsname
    \DeclareRobustCommand#1{}%
    \ifx#1\MR@gtemp
      \endgroup
      \PackageInfo{makerobust}{%
        \string`\string#1\string' is already robust%
      }%
    \else
      \@ifundefined{MR@temp}{%
        \global\let\MR@gtemp#1%
        \endgroup
        \expandafter\let\csname\expandafter\@gobble\string#1 \endcsname#1%
        \let#1\MR@gtemp
      }{%
        \endgroup
        \PackageError{makerobust}{%
          Internal macro \string`\string#1 \string' already exists%
        }\@ehc
      }%
    \fi
  }%
}
%    \end{macrocode}
%
%    \begin{macrocode}
%</package>
%    \end{macrocode}
%
% \section{Installation}
%
% \subsection{Download}
%
% \paragraph{Package.} This package is available on
% CTAN\footnote{\url{ftp://ftp.ctan.org/tex-archive/}}:
% \begin{description}
% \item[\CTAN{macros/latex/contrib/oberdiek/makerobust.dtx}] The source file.
% \item[\CTAN{macros/latex/contrib/oberdiek/makerobust.pdf}] Documentation.
% \end{description}
%
%
% \paragraph{Bundle.} All the packages of the bundle `oberdiek'
% are also available in a TDS compliant ZIP archive. There
% the packages are already unpacked and the documentation files
% are generated. The files and directories obey the TDS standard.
% \begin{description}
% \item[\CTAN{install/macros/latex/contrib/oberdiek.tds.zip}]
% \end{description}
% \emph{TDS} refers to the standard ``A Directory Structure
% for \TeX\ Files'' (\CTAN{tds/tds.pdf}). Directories
% with \xfile{texmf} in their name are usually organized this way.
%
% \subsection{Bundle installation}
%
% \paragraph{Unpacking.} Unpack the \xfile{oberdiek.tds.zip} in the
% TDS tree (also known as \xfile{texmf} tree) of your choice.
% Example (linux):
% \begin{quote}
%   |unzip oberdiek.tds.zip -d ~/texmf|
% \end{quote}
%
% \paragraph{Script installation.}
% Check the directory \xfile{TDS:scripts/oberdiek/} for
% scripts that need further installation steps.
% Package \xpackage{attachfile2} comes with the Perl script
% \xfile{pdfatfi.pl} that should be installed in such a way
% that it can be called as \texttt{pdfatfi}.
% Example (linux):
% \begin{quote}
%   |chmod +x scripts/oberdiek/pdfatfi.pl|\\
%   |cp scripts/oberdiek/pdfatfi.pl /usr/local/bin/|
% \end{quote}
%
% \subsection{Package installation}
%
% \paragraph{Unpacking.} The \xfile{.dtx} file is a self-extracting
% \docstrip\ archive. The files are extracted by running the
% \xfile{.dtx} through \plainTeX:
% \begin{quote}
%   \verb|tex makerobust.dtx|
% \end{quote}
%
% \paragraph{TDS.} Now the different files must be moved into
% the different directories in your installation TDS tree
% (also known as \xfile{texmf} tree):
% \begin{quote}
% \def\t{^^A
% \begin{tabular}{@{}>{\ttfamily}l@{ $\rightarrow$ }>{\ttfamily}l@{}}
%   makerobust.sty & tex/latex/oberdiek/makerobust.sty\\
%   makerobust.pdf & doc/latex/oberdiek/makerobust.pdf\\
%   makerobust-example.tex & doc/latex/oberdiek/makerobust-example.tex\\
%   makerobust.dtx & source/latex/oberdiek/makerobust.dtx\\
% \end{tabular}^^A
% }^^A
% \sbox0{\t}^^A
% \ifdim\wd0>\linewidth
%   \begingroup
%     \advance\linewidth by\leftmargin
%     \advance\linewidth by\rightmargin
%   \edef\x{\endgroup
%     \def\noexpand\lw{\the\linewidth}^^A
%   }\x
%   \def\lwbox{^^A
%     \leavevmode
%     \hbox to \linewidth{^^A
%       \kern-\leftmargin\relax
%       \hss
%       \usebox0
%       \hss
%       \kern-\rightmargin\relax
%     }^^A
%   }^^A
%   \ifdim\wd0>\lw
%     \sbox0{\small\t}^^A
%     \ifdim\wd0>\linewidth
%       \ifdim\wd0>\lw
%         \sbox0{\footnotesize\t}^^A
%         \ifdim\wd0>\linewidth
%           \ifdim\wd0>\lw
%             \sbox0{\scriptsize\t}^^A
%             \ifdim\wd0>\linewidth
%               \ifdim\wd0>\lw
%                 \sbox0{\tiny\t}^^A
%                 \ifdim\wd0>\linewidth
%                   \lwbox
%                 \else
%                   \usebox0
%                 \fi
%               \else
%                 \lwbox
%               \fi
%             \else
%               \usebox0
%             \fi
%           \else
%             \lwbox
%           \fi
%         \else
%           \usebox0
%         \fi
%       \else
%         \lwbox
%       \fi
%     \else
%       \usebox0
%     \fi
%   \else
%     \lwbox
%   \fi
% \else
%   \usebox0
% \fi
% \end{quote}
% If you have a \xfile{docstrip.cfg} that configures and enables \docstrip's
% TDS installing feature, then some files can already be in the right
% place, see the documentation of \docstrip.
%
% \subsection{Refresh file name databases}
%
% If your \TeX~distribution
% (\teTeX, \mikTeX, \dots) relies on file name databases, you must refresh
% these. For example, \teTeX\ users run \verb|texhash| or
% \verb|mktexlsr|.
%
% \subsection{Some details for the interested}
%
% \paragraph{Attached source.}
%
% The PDF documentation on CTAN also includes the
% \xfile{.dtx} source file. It can be extracted by
% AcrobatReader 6 or higher. Another option is \textsf{pdftk},
% e.g. unpack the file into the current directory:
% \begin{quote}
%   \verb|pdftk makerobust.pdf unpack_files output .|
% \end{quote}
%
% \paragraph{Unpacking with \LaTeX.}
% The \xfile{.dtx} chooses its action depending on the format:
% \begin{description}
% \item[\plainTeX:] Run \docstrip\ and extract the files.
% \item[\LaTeX:] Generate the documentation.
% \end{description}
% If you insist on using \LaTeX\ for \docstrip\ (really,
% \docstrip\ does not need \LaTeX), then inform the autodetect routine
% about your intention:
% \begin{quote}
%   \verb|latex \let\install=y\input{makerobust.dtx}|
% \end{quote}
% Do not forget to quote the argument according to the demands
% of your shell.
%
% \paragraph{Generating the documentation.}
% You can use both the \xfile{.dtx} or the \xfile{.drv} to generate
% the documentation. The process can be configured by the
% configuration file \xfile{ltxdoc.cfg}. For instance, put this
% line into this file, if you want to have A4 as paper format:
% \begin{quote}
%   \verb|\PassOptionsToClass{a4paper}{article}|
% \end{quote}
% An example follows how to generate the
% documentation with pdf\LaTeX:
% \begin{quote}
%\begin{verbatim}
%pdflatex makerobust.dtx
%makeindex -s gind.ist makerobust.idx
%pdflatex makerobust.dtx
%makeindex -s gind.ist makerobust.idx
%pdflatex makerobust.dtx
%\end{verbatim}
% \end{quote}
%
% \section{Catalogue}
%
% The following XML file can be used as source for the
% \href{http://mirror.ctan.org/help/Catalogue/catalogue.html}{\TeX\ Catalogue}.
% The elements \texttt{caption} and \texttt{description} are imported
% from the original XML file from the Catalogue.
% The name of the XML file in the Catalogue is \xfile{makerobust.xml}.
%    \begin{macrocode}
%<*catalogue>
<?xml version='1.0' encoding='us-ascii'?>
<!DOCTYPE entry SYSTEM 'catalogue.dtd'>
<entry datestamp='$Date$' modifier='$Author$' id='makerobust'>
  <name>makerobust</name>
  <caption>Making a macro robust.</caption>
  <authorref id='auth:oberdiek'/>
  <copyright owner='Heiko Oberdiek' year='2006'/>
  <license type='lppl1.3'/>
  <version number='1.0'/>
  <description>
    This package provides the command MakeRobustCommand
    that converts an existing macro to a robust one.
    <p/>
    The package is part of the <xref refid='oberdiek'>oberdiek</xref>
    bundle.
  </description>
  <documentation details='Package documentation'
      href='ctan:/macros/latex/contrib/oberdiek/makerobust.pdf'/>
  <ctan file='true' path='/macros/latex/contrib/oberdiek/makerobust.dtx'/>
  <miktex location='oberdiek'/>
  <texlive location='oberdiek'/>
  <install path='/macros/latex/contrib/oberdiek/oberdiek.tds.zip'/>
</entry>
%</catalogue>
%    \end{macrocode}
%
% \begin{History}
%   \begin{Version}{2006/03/18 v1.0}
%   \item
%     First version.
%   \end{Version}
% \end{History}
%
% \PrintIndex
%
% \Finale
\endinput
|
% \end{quote}
% Do not forget to quote the argument according to the demands
% of your shell.
%
% \paragraph{Generating the documentation.}
% You can use both the \xfile{.dtx} or the \xfile{.drv} to generate
% the documentation. The process can be configured by the
% configuration file \xfile{ltxdoc.cfg}. For instance, put this
% line into this file, if you want to have A4 as paper format:
% \begin{quote}
%   \verb|\PassOptionsToClass{a4paper}{article}|
% \end{quote}
% An example follows how to generate the
% documentation with pdf\LaTeX:
% \begin{quote}
%\begin{verbatim}
%pdflatex makerobust.dtx
%makeindex -s gind.ist makerobust.idx
%pdflatex makerobust.dtx
%makeindex -s gind.ist makerobust.idx
%pdflatex makerobust.dtx
%\end{verbatim}
% \end{quote}
%
% \section{Catalogue}
%
% The following XML file can be used as source for the
% \href{http://mirror.ctan.org/help/Catalogue/catalogue.html}{\TeX\ Catalogue}.
% The elements \texttt{caption} and \texttt{description} are imported
% from the original XML file from the Catalogue.
% The name of the XML file in the Catalogue is \xfile{makerobust.xml}.
%    \begin{macrocode}
%<*catalogue>
<?xml version='1.0' encoding='us-ascii'?>
<!DOCTYPE entry SYSTEM 'catalogue.dtd'>
<entry datestamp='$Date$' modifier='$Author$' id='makerobust'>
  <name>makerobust</name>
  <caption>Making a macro robust.</caption>
  <authorref id='auth:oberdiek'/>
  <copyright owner='Heiko Oberdiek' year='2006'/>
  <license type='lppl1.3'/>
  <version number='1.0'/>
  <description>
    This package provides the command MakeRobustCommand
    that converts an existing macro to a robust one.
    <p/>
    The package is part of the <xref refid='oberdiek'>oberdiek</xref>
    bundle.
  </description>
  <documentation details='Package documentation'
      href='ctan:/macros/latex/contrib/oberdiek/makerobust.pdf'/>
  <ctan file='true' path='/macros/latex/contrib/oberdiek/makerobust.dtx'/>
  <miktex location='oberdiek'/>
  <texlive location='oberdiek'/>
  <install path='/macros/latex/contrib/oberdiek/oberdiek.tds.zip'/>
</entry>
%</catalogue>
%    \end{macrocode}
%
% \begin{History}
%   \begin{Version}{2006/03/18 v1.0}
%   \item
%     First version.
%   \end{Version}
% \end{History}
%
% \PrintIndex
%
% \Finale
\endinput

%        (quote the arguments according to the demands of your shell)
%
% Documentation:
%    (a) If makerobust.drv is present:
%           latex makerobust.drv
%    (b) Without makerobust.drv:
%           latex makerobust.dtx; ...
%    The class ltxdoc loads the configuration file ltxdoc.cfg
%    if available. Here you can specify further options, e.g.
%    use A4 as paper format:
%       \PassOptionsToClass{a4paper}{article}
%
%    Programm calls to get the documentation (example):
%       pdflatex makerobust.dtx
%       makeindex -s gind.ist makerobust.idx
%       pdflatex makerobust.dtx
%       makeindex -s gind.ist makerobust.idx
%       pdflatex makerobust.dtx
%
% Installation:
%    TDS:tex/latex/oberdiek/makerobust.sty
%    TDS:doc/latex/oberdiek/makerobust.pdf
%    TDS:doc/latex/oberdiek/makerobust-example.tex
%    TDS:source/latex/oberdiek/makerobust.dtx
%
%<*ignore>
\begingroup
  \catcode123=1 %
  \catcode125=2 %
  \def\x{LaTeX2e}%
\expandafter\endgroup
\ifcase 0\ifx\install y1\fi\expandafter
         \ifx\csname processbatchFile\endcsname\relax\else1\fi
         \ifx\fmtname\x\else 1\fi\relax
\else\csname fi\endcsname
%</ignore>
%<*install>
\input docstrip.tex
\Msg{************************************************************************}
\Msg{* Installation}
\Msg{* Package: makerobust 2006/03/18 v1.0 Make existing macro robust (HO)}
\Msg{************************************************************************}

\keepsilent
\askforoverwritefalse

\let\MetaPrefix\relax
\preamble

This is a generated file.

Project: makerobust
Version: 2006/03/18 v1.0

Copyright (C) 2006 by
   Heiko Oberdiek <heiko.oberdiek at googlemail.com>

This work may be distributed and/or modified under the
conditions of the LaTeX Project Public License, either
version 1.3c of this license or (at your option) any later
version. This version of this license is in
   http://www.latex-project.org/lppl/lppl-1-3c.txt
and the latest version of this license is in
   http://www.latex-project.org/lppl.txt
and version 1.3 or later is part of all distributions of
LaTeX version 2005/12/01 or later.

This work has the LPPL maintenance status "maintained".

This Current Maintainer of this work is Heiko Oberdiek.

This work consists of the main source file makerobust.dtx
and the derived files
   makerobust.sty, makerobust.pdf, makerobust.ins, makerobust.drv,
   makerobust-example.tex.

\endpreamble
\let\MetaPrefix\DoubleperCent

\generate{%
  \file{makerobust.ins}{\from{makerobust.dtx}{install}}%
  \file{makerobust.drv}{\from{makerobust.dtx}{driver}}%
  \usedir{tex/latex/oberdiek}%
  \file{makerobust.sty}{\from{makerobust.dtx}{package}}%
  \usedir{doc/latex/oberdiek}%
  \file{makerobust-example.tex}{\from{makerobust.dtx}{example}}%
  \nopreamble
  \nopostamble
  \usedir{source/latex/oberdiek/catalogue}%
  \file{makerobust.xml}{\from{makerobust.dtx}{catalogue}}%
}

\catcode32=13\relax% active space
\let =\space%
\Msg{************************************************************************}
\Msg{*}
\Msg{* To finish the installation you have to move the following}
\Msg{* file into a directory searched by TeX:}
\Msg{*}
\Msg{*     makerobust.sty}
\Msg{*}
\Msg{* To produce the documentation run the file `makerobust.drv'}
\Msg{* through LaTeX.}
\Msg{*}
\Msg{* Happy TeXing!}
\Msg{*}
\Msg{************************************************************************}

\endbatchfile
%</install>
%<*ignore>
\fi
%</ignore>
%<*driver>
\NeedsTeXFormat{LaTeX2e}
\ProvidesFile{makerobust.drv}%
  [2006/03/18 v1.0 Make existing macro robust (HO)]%
\documentclass{ltxdoc}
\usepackage{holtxdoc}[2011/11/22]
\begin{document}
  \DocInput{makerobust.dtx}%
\end{document}
%</driver>
% \fi
%
% \CheckSum{59}
%
% \CharacterTable
%  {Upper-case    \A\B\C\D\E\F\G\H\I\J\K\L\M\N\O\P\Q\R\S\T\U\V\W\X\Y\Z
%   Lower-case    \a\b\c\d\e\f\g\h\i\j\k\l\m\n\o\p\q\r\s\t\u\v\w\x\y\z
%   Digits        \0\1\2\3\4\5\6\7\8\9
%   Exclamation   \!     Double quote  \"     Hash (number) \#
%   Dollar        \$     Percent       \%     Ampersand     \&
%   Acute accent  \'     Left paren    \(     Right paren   \)
%   Asterisk      \*     Plus          \+     Comma         \,
%   Minus         \-     Point         \.     Solidus       \/
%   Colon         \:     Semicolon     \;     Less than     \<
%   Equals        \=     Greater than  \>     Question mark \?
%   Commercial at \@     Left bracket  \[     Backslash     \\
%   Right bracket \]     Circumflex    \^     Underscore    \_
%   Grave accent  \`     Left brace    \{     Vertical bar  \|
%   Right brace   \}     Tilde         \~}
%
% \GetFileInfo{makerobust.drv}
%
% \title{The \xpackage{makerobust} package}
% \date{2006/03/18 v1.0}
% \author{Heiko Oberdiek\\\xemail{heiko.oberdiek at googlemail.com}}
%
% \maketitle
%
% \begin{abstract}
% Package \xpackage{makerobust} provides \cs{MakeRobustCommand}
% that converts an existing macro to a robust one.
% \end{abstract}
%
% \tableofcontents
%
% \section{User interface}
%
% \LaTeX\ offers \cs{DeclareRobustCommand} to define a robust macro
% that does not break if it is used in moving arguments.
% Sometimes a macro is already defined, but not robust. For
% example, \cs{(} and \cs{)} are not robust, inside \cs{section}
% the user must use \cs{protect} explicitly. This could be
% avoided by making \cs{(} and \cs{)} robust.
%
% \begin{declcs}{MakeRobustCommand}\M{cmd}
% \end{declcs}
% \cs{MakeRobustCommand} redefines the macro \meta{cmd}
% by using \cs{DeclareRobustCommand} and the existing definition
% of the macro \meta{cmd}.
% \begin{itemize}
% \item It is an error if \meta{cmd} is undefined. If you want to
%   define a robust command, then you can use \cs{DeclareRobustCommand}
%   directly.
% \item If the macro has previously been
%   defined by \cs{DeclareRobustCommand} then the redefinition of
%   \cs{MakeRobustCommand} is omitted, because the macro is already robust.
%   Only an information entry is written to the \xfile{.log} file.
%   Thus you do not get a warning or an error if the macro is already
%   robust because of an updated LaTeX or package that defines the macro.
% \item Two macros are defined for a macro, defined
%   by \cs{DeclareRobustCommand}. Example:
%   \begin{quote}
%   |\DeclareRobustCommand{\foobar}{definition text}|
%   \end{quote}
%   Then the macro ``\cs{foobar}'' contains the protection code
%   and, depending on the protection mode,
%   calls the internal macro ``\cs{foobar }''. Notice the space
%   at the end of the macro name.
%   This internal macro ``\cs{foobar }'' now contains the definition
%   ``|definition text|'', given in \cs{DeclareRobustCommand}.
%
%   Sometimes it can happen, that the internal macro already exists.
%   This can be caused by a previous \cs{DeclareRobustCommand} followed
%   by \cs{renewcommand}. Then the redefinition by \cs{MakeRobustCommand}
%   would be safe.
%
%   However, it can also be possible that the macro is already robust,
%   using the internal macro, but with a different protection code.
%   The redefinition by \cs{MakeRobustCommand} would then generate
%   an infinite loop.
%
%   Therefore \cs{MakeRobustCommand} raises an error message,
%   if the internal macro (with space at the end) already exists.
% \end{itemize}
%
% \subsection{Example}
%
%    \begin{macrocode}
%<*example>
\documentclass{article}
\usepackage{makerobust}
\MakeRobustCommand\(
\MakeRobustCommand\)
\pagestyle{headings}
\begin{document}
\tableofcontents
\section{Einstein: \(E=mc^2\)}
\newpage
Second page.
\end{document}
%</example>
%    \end{macrocode}
%
%
% \StopEventually{
% }
%
% \section{Implementation}
%
%    \begin{macrocode}
%<*package>
\NeedsTeXFormat{LaTeX2e}
\ProvidesPackage{makerobust}%
  [2006/03/18 v1.0 Make existing macro robust (HO)]%
%    \end{macrocode}
%
%    \begin{macrocode}
\def\MakeRobustCommand#1{%
  \begingroup
  \@ifundefined{\expandafter\@gobble\string#1}{%
    \endgroup
    \PackageError{makerobust}{%
      Macro \string`\string#1\string' is not defined%
    }\@ehc
  }{%
    \global\let\MR@gtemp#1%
    \let#1\@undefined
    \expandafter\let\expandafter\MR@temp
        \csname\expandafter\@gobble\string#1 \endcsname
    \DeclareRobustCommand#1{}%
    \ifx#1\MR@gtemp
      \endgroup
      \PackageInfo{makerobust}{%
        \string`\string#1\string' is already robust%
      }%
    \else
      \@ifundefined{MR@temp}{%
        \global\let\MR@gtemp#1%
        \endgroup
        \expandafter\let\csname\expandafter\@gobble\string#1 \endcsname#1%
        \let#1\MR@gtemp
      }{%
        \endgroup
        \PackageError{makerobust}{%
          Internal macro \string`\string#1 \string' already exists%
        }\@ehc
      }%
    \fi
  }%
}
%    \end{macrocode}
%
%    \begin{macrocode}
%</package>
%    \end{macrocode}
%
% \section{Installation}
%
% \subsection{Download}
%
% \paragraph{Package.} This package is available on
% CTAN\footnote{\url{ftp://ftp.ctan.org/tex-archive/}}:
% \begin{description}
% \item[\CTAN{macros/latex/contrib/oberdiek/makerobust.dtx}] The source file.
% \item[\CTAN{macros/latex/contrib/oberdiek/makerobust.pdf}] Documentation.
% \end{description}
%
%
% \paragraph{Bundle.} All the packages of the bundle `oberdiek'
% are also available in a TDS compliant ZIP archive. There
% the packages are already unpacked and the documentation files
% are generated. The files and directories obey the TDS standard.
% \begin{description}
% \item[\CTAN{install/macros/latex/contrib/oberdiek.tds.zip}]
% \end{description}
% \emph{TDS} refers to the standard ``A Directory Structure
% for \TeX\ Files'' (\CTAN{tds/tds.pdf}). Directories
% with \xfile{texmf} in their name are usually organized this way.
%
% \subsection{Bundle installation}
%
% \paragraph{Unpacking.} Unpack the \xfile{oberdiek.tds.zip} in the
% TDS tree (also known as \xfile{texmf} tree) of your choice.
% Example (linux):
% \begin{quote}
%   |unzip oberdiek.tds.zip -d ~/texmf|
% \end{quote}
%
% \paragraph{Script installation.}
% Check the directory \xfile{TDS:scripts/oberdiek/} for
% scripts that need further installation steps.
% Package \xpackage{attachfile2} comes with the Perl script
% \xfile{pdfatfi.pl} that should be installed in such a way
% that it can be called as \texttt{pdfatfi}.
% Example (linux):
% \begin{quote}
%   |chmod +x scripts/oberdiek/pdfatfi.pl|\\
%   |cp scripts/oberdiek/pdfatfi.pl /usr/local/bin/|
% \end{quote}
%
% \subsection{Package installation}
%
% \paragraph{Unpacking.} The \xfile{.dtx} file is a self-extracting
% \docstrip\ archive. The files are extracted by running the
% \xfile{.dtx} through \plainTeX:
% \begin{quote}
%   \verb|tex makerobust.dtx|
% \end{quote}
%
% \paragraph{TDS.} Now the different files must be moved into
% the different directories in your installation TDS tree
% (also known as \xfile{texmf} tree):
% \begin{quote}
% \def\t{^^A
% \begin{tabular}{@{}>{\ttfamily}l@{ $\rightarrow$ }>{\ttfamily}l@{}}
%   makerobust.sty & tex/latex/oberdiek/makerobust.sty\\
%   makerobust.pdf & doc/latex/oberdiek/makerobust.pdf\\
%   makerobust-example.tex & doc/latex/oberdiek/makerobust-example.tex\\
%   makerobust.dtx & source/latex/oberdiek/makerobust.dtx\\
% \end{tabular}^^A
% }^^A
% \sbox0{\t}^^A
% \ifdim\wd0>\linewidth
%   \begingroup
%     \advance\linewidth by\leftmargin
%     \advance\linewidth by\rightmargin
%   \edef\x{\endgroup
%     \def\noexpand\lw{\the\linewidth}^^A
%   }\x
%   \def\lwbox{^^A
%     \leavevmode
%     \hbox to \linewidth{^^A
%       \kern-\leftmargin\relax
%       \hss
%       \usebox0
%       \hss
%       \kern-\rightmargin\relax
%     }^^A
%   }^^A
%   \ifdim\wd0>\lw
%     \sbox0{\small\t}^^A
%     \ifdim\wd0>\linewidth
%       \ifdim\wd0>\lw
%         \sbox0{\footnotesize\t}^^A
%         \ifdim\wd0>\linewidth
%           \ifdim\wd0>\lw
%             \sbox0{\scriptsize\t}^^A
%             \ifdim\wd0>\linewidth
%               \ifdim\wd0>\lw
%                 \sbox0{\tiny\t}^^A
%                 \ifdim\wd0>\linewidth
%                   \lwbox
%                 \else
%                   \usebox0
%                 \fi
%               \else
%                 \lwbox
%               \fi
%             \else
%               \usebox0
%             \fi
%           \else
%             \lwbox
%           \fi
%         \else
%           \usebox0
%         \fi
%       \else
%         \lwbox
%       \fi
%     \else
%       \usebox0
%     \fi
%   \else
%     \lwbox
%   \fi
% \else
%   \usebox0
% \fi
% \end{quote}
% If you have a \xfile{docstrip.cfg} that configures and enables \docstrip's
% TDS installing feature, then some files can already be in the right
% place, see the documentation of \docstrip.
%
% \subsection{Refresh file name databases}
%
% If your \TeX~distribution
% (\teTeX, \mikTeX, \dots) relies on file name databases, you must refresh
% these. For example, \teTeX\ users run \verb|texhash| or
% \verb|mktexlsr|.
%
% \subsection{Some details for the interested}
%
% \paragraph{Attached source.}
%
% The PDF documentation on CTAN also includes the
% \xfile{.dtx} source file. It can be extracted by
% AcrobatReader 6 or higher. Another option is \textsf{pdftk},
% e.g. unpack the file into the current directory:
% \begin{quote}
%   \verb|pdftk makerobust.pdf unpack_files output .|
% \end{quote}
%
% \paragraph{Unpacking with \LaTeX.}
% The \xfile{.dtx} chooses its action depending on the format:
% \begin{description}
% \item[\plainTeX:] Run \docstrip\ and extract the files.
% \item[\LaTeX:] Generate the documentation.
% \end{description}
% If you insist on using \LaTeX\ for \docstrip\ (really,
% \docstrip\ does not need \LaTeX), then inform the autodetect routine
% about your intention:
% \begin{quote}
%   \verb|latex \let\install=y% \iffalse meta-comment
%
% File: makerobust.dtx
% Version: 2006/03/18 v1.0
% Info: Make existing macro robust
%
% Copyright (C) 2006 by
%    Heiko Oberdiek <heiko.oberdiek at googlemail.com>
%
% This work may be distributed and/or modified under the
% conditions of the LaTeX Project Public License, either
% version 1.3c of this license or (at your option) any later
% version. This version of this license is in
%    http://www.latex-project.org/lppl/lppl-1-3c.txt
% and the latest version of this license is in
%    http://www.latex-project.org/lppl.txt
% and version 1.3 or later is part of all distributions of
% LaTeX version 2005/12/01 or later.
%
% This work has the LPPL maintenance status "maintained".
%
% This Current Maintainer of this work is Heiko Oberdiek.
%
% This work consists of the main source file makerobust.dtx
% and the derived files
%    makerobust.sty, makerobust.pdf, makerobust.ins, makerobust.drv,
%    makerobust-example.tex.
%
% Distribution:
%    CTAN:macros/latex/contrib/oberdiek/makerobust.dtx
%    CTAN:macros/latex/contrib/oberdiek/makerobust.pdf
%
% Unpacking:
%    (a) If makerobust.ins is present:
%           tex makerobust.ins
%    (b) Without makerobust.ins:
%           tex makerobust.dtx
%    (c) If you insist on using LaTeX
%           latex \let\install=y% \iffalse meta-comment
%
% File: makerobust.dtx
% Version: 2006/03/18 v1.0
% Info: Make existing macro robust
%
% Copyright (C) 2006 by
%    Heiko Oberdiek <heiko.oberdiek at googlemail.com>
%
% This work may be distributed and/or modified under the
% conditions of the LaTeX Project Public License, either
% version 1.3c of this license or (at your option) any later
% version. This version of this license is in
%    http://www.latex-project.org/lppl/lppl-1-3c.txt
% and the latest version of this license is in
%    http://www.latex-project.org/lppl.txt
% and version 1.3 or later is part of all distributions of
% LaTeX version 2005/12/01 or later.
%
% This work has the LPPL maintenance status "maintained".
%
% This Current Maintainer of this work is Heiko Oberdiek.
%
% This work consists of the main source file makerobust.dtx
% and the derived files
%    makerobust.sty, makerobust.pdf, makerobust.ins, makerobust.drv,
%    makerobust-example.tex.
%
% Distribution:
%    CTAN:macros/latex/contrib/oberdiek/makerobust.dtx
%    CTAN:macros/latex/contrib/oberdiek/makerobust.pdf
%
% Unpacking:
%    (a) If makerobust.ins is present:
%           tex makerobust.ins
%    (b) Without makerobust.ins:
%           tex makerobust.dtx
%    (c) If you insist on using LaTeX
%           latex \let\install=y\input{makerobust.dtx}
%        (quote the arguments according to the demands of your shell)
%
% Documentation:
%    (a) If makerobust.drv is present:
%           latex makerobust.drv
%    (b) Without makerobust.drv:
%           latex makerobust.dtx; ...
%    The class ltxdoc loads the configuration file ltxdoc.cfg
%    if available. Here you can specify further options, e.g.
%    use A4 as paper format:
%       \PassOptionsToClass{a4paper}{article}
%
%    Programm calls to get the documentation (example):
%       pdflatex makerobust.dtx
%       makeindex -s gind.ist makerobust.idx
%       pdflatex makerobust.dtx
%       makeindex -s gind.ist makerobust.idx
%       pdflatex makerobust.dtx
%
% Installation:
%    TDS:tex/latex/oberdiek/makerobust.sty
%    TDS:doc/latex/oberdiek/makerobust.pdf
%    TDS:doc/latex/oberdiek/makerobust-example.tex
%    TDS:source/latex/oberdiek/makerobust.dtx
%
%<*ignore>
\begingroup
  \catcode123=1 %
  \catcode125=2 %
  \def\x{LaTeX2e}%
\expandafter\endgroup
\ifcase 0\ifx\install y1\fi\expandafter
         \ifx\csname processbatchFile\endcsname\relax\else1\fi
         \ifx\fmtname\x\else 1\fi\relax
\else\csname fi\endcsname
%</ignore>
%<*install>
\input docstrip.tex
\Msg{************************************************************************}
\Msg{* Installation}
\Msg{* Package: makerobust 2006/03/18 v1.0 Make existing macro robust (HO)}
\Msg{************************************************************************}

\keepsilent
\askforoverwritefalse

\let\MetaPrefix\relax
\preamble

This is a generated file.

Project: makerobust
Version: 2006/03/18 v1.0

Copyright (C) 2006 by
   Heiko Oberdiek <heiko.oberdiek at googlemail.com>

This work may be distributed and/or modified under the
conditions of the LaTeX Project Public License, either
version 1.3c of this license or (at your option) any later
version. This version of this license is in
   http://www.latex-project.org/lppl/lppl-1-3c.txt
and the latest version of this license is in
   http://www.latex-project.org/lppl.txt
and version 1.3 or later is part of all distributions of
LaTeX version 2005/12/01 or later.

This work has the LPPL maintenance status "maintained".

This Current Maintainer of this work is Heiko Oberdiek.

This work consists of the main source file makerobust.dtx
and the derived files
   makerobust.sty, makerobust.pdf, makerobust.ins, makerobust.drv,
   makerobust-example.tex.

\endpreamble
\let\MetaPrefix\DoubleperCent

\generate{%
  \file{makerobust.ins}{\from{makerobust.dtx}{install}}%
  \file{makerobust.drv}{\from{makerobust.dtx}{driver}}%
  \usedir{tex/latex/oberdiek}%
  \file{makerobust.sty}{\from{makerobust.dtx}{package}}%
  \usedir{doc/latex/oberdiek}%
  \file{makerobust-example.tex}{\from{makerobust.dtx}{example}}%
  \nopreamble
  \nopostamble
  \usedir{source/latex/oberdiek/catalogue}%
  \file{makerobust.xml}{\from{makerobust.dtx}{catalogue}}%
}

\catcode32=13\relax% active space
\let =\space%
\Msg{************************************************************************}
\Msg{*}
\Msg{* To finish the installation you have to move the following}
\Msg{* file into a directory searched by TeX:}
\Msg{*}
\Msg{*     makerobust.sty}
\Msg{*}
\Msg{* To produce the documentation run the file `makerobust.drv'}
\Msg{* through LaTeX.}
\Msg{*}
\Msg{* Happy TeXing!}
\Msg{*}
\Msg{************************************************************************}

\endbatchfile
%</install>
%<*ignore>
\fi
%</ignore>
%<*driver>
\NeedsTeXFormat{LaTeX2e}
\ProvidesFile{makerobust.drv}%
  [2006/03/18 v1.0 Make existing macro robust (HO)]%
\documentclass{ltxdoc}
\usepackage{holtxdoc}[2011/11/22]
\begin{document}
  \DocInput{makerobust.dtx}%
\end{document}
%</driver>
% \fi
%
% \CheckSum{59}
%
% \CharacterTable
%  {Upper-case    \A\B\C\D\E\F\G\H\I\J\K\L\M\N\O\P\Q\R\S\T\U\V\W\X\Y\Z
%   Lower-case    \a\b\c\d\e\f\g\h\i\j\k\l\m\n\o\p\q\r\s\t\u\v\w\x\y\z
%   Digits        \0\1\2\3\4\5\6\7\8\9
%   Exclamation   \!     Double quote  \"     Hash (number) \#
%   Dollar        \$     Percent       \%     Ampersand     \&
%   Acute accent  \'     Left paren    \(     Right paren   \)
%   Asterisk      \*     Plus          \+     Comma         \,
%   Minus         \-     Point         \.     Solidus       \/
%   Colon         \:     Semicolon     \;     Less than     \<
%   Equals        \=     Greater than  \>     Question mark \?
%   Commercial at \@     Left bracket  \[     Backslash     \\
%   Right bracket \]     Circumflex    \^     Underscore    \_
%   Grave accent  \`     Left brace    \{     Vertical bar  \|
%   Right brace   \}     Tilde         \~}
%
% \GetFileInfo{makerobust.drv}
%
% \title{The \xpackage{makerobust} package}
% \date{2006/03/18 v1.0}
% \author{Heiko Oberdiek\\\xemail{heiko.oberdiek at googlemail.com}}
%
% \maketitle
%
% \begin{abstract}
% Package \xpackage{makerobust} provides \cs{MakeRobustCommand}
% that converts an existing macro to a robust one.
% \end{abstract}
%
% \tableofcontents
%
% \section{User interface}
%
% \LaTeX\ offers \cs{DeclareRobustCommand} to define a robust macro
% that does not break if it is used in moving arguments.
% Sometimes a macro is already defined, but not robust. For
% example, \cs{(} and \cs{)} are not robust, inside \cs{section}
% the user must use \cs{protect} explicitly. This could be
% avoided by making \cs{(} and \cs{)} robust.
%
% \begin{declcs}{MakeRobustCommand}\M{cmd}
% \end{declcs}
% \cs{MakeRobustCommand} redefines the macro \meta{cmd}
% by using \cs{DeclareRobustCommand} and the existing definition
% of the macro \meta{cmd}.
% \begin{itemize}
% \item It is an error if \meta{cmd} is undefined. If you want to
%   define a robust command, then you can use \cs{DeclareRobustCommand}
%   directly.
% \item If the macro has previously been
%   defined by \cs{DeclareRobustCommand} then the redefinition of
%   \cs{MakeRobustCommand} is omitted, because the macro is already robust.
%   Only an information entry is written to the \xfile{.log} file.
%   Thus you do not get a warning or an error if the macro is already
%   robust because of an updated LaTeX or package that defines the macro.
% \item Two macros are defined for a macro, defined
%   by \cs{DeclareRobustCommand}. Example:
%   \begin{quote}
%   |\DeclareRobustCommand{\foobar}{definition text}|
%   \end{quote}
%   Then the macro ``\cs{foobar}'' contains the protection code
%   and, depending on the protection mode,
%   calls the internal macro ``\cs{foobar }''. Notice the space
%   at the end of the macro name.
%   This internal macro ``\cs{foobar }'' now contains the definition
%   ``|definition text|'', given in \cs{DeclareRobustCommand}.
%
%   Sometimes it can happen, that the internal macro already exists.
%   This can be caused by a previous \cs{DeclareRobustCommand} followed
%   by \cs{renewcommand}. Then the redefinition by \cs{MakeRobustCommand}
%   would be safe.
%
%   However, it can also be possible that the macro is already robust,
%   using the internal macro, but with a different protection code.
%   The redefinition by \cs{MakeRobustCommand} would then generate
%   an infinite loop.
%
%   Therefore \cs{MakeRobustCommand} raises an error message,
%   if the internal macro (with space at the end) already exists.
% \end{itemize}
%
% \subsection{Example}
%
%    \begin{macrocode}
%<*example>
\documentclass{article}
\usepackage{makerobust}
\MakeRobustCommand\(
\MakeRobustCommand\)
\pagestyle{headings}
\begin{document}
\tableofcontents
\section{Einstein: \(E=mc^2\)}
\newpage
Second page.
\end{document}
%</example>
%    \end{macrocode}
%
%
% \StopEventually{
% }
%
% \section{Implementation}
%
%    \begin{macrocode}
%<*package>
\NeedsTeXFormat{LaTeX2e}
\ProvidesPackage{makerobust}%
  [2006/03/18 v1.0 Make existing macro robust (HO)]%
%    \end{macrocode}
%
%    \begin{macrocode}
\def\MakeRobustCommand#1{%
  \begingroup
  \@ifundefined{\expandafter\@gobble\string#1}{%
    \endgroup
    \PackageError{makerobust}{%
      Macro \string`\string#1\string' is not defined%
    }\@ehc
  }{%
    \global\let\MR@gtemp#1%
    \let#1\@undefined
    \expandafter\let\expandafter\MR@temp
        \csname\expandafter\@gobble\string#1 \endcsname
    \DeclareRobustCommand#1{}%
    \ifx#1\MR@gtemp
      \endgroup
      \PackageInfo{makerobust}{%
        \string`\string#1\string' is already robust%
      }%
    \else
      \@ifundefined{MR@temp}{%
        \global\let\MR@gtemp#1%
        \endgroup
        \expandafter\let\csname\expandafter\@gobble\string#1 \endcsname#1%
        \let#1\MR@gtemp
      }{%
        \endgroup
        \PackageError{makerobust}{%
          Internal macro \string`\string#1 \string' already exists%
        }\@ehc
      }%
    \fi
  }%
}
%    \end{macrocode}
%
%    \begin{macrocode}
%</package>
%    \end{macrocode}
%
% \section{Installation}
%
% \subsection{Download}
%
% \paragraph{Package.} This package is available on
% CTAN\footnote{\url{ftp://ftp.ctan.org/tex-archive/}}:
% \begin{description}
% \item[\CTAN{macros/latex/contrib/oberdiek/makerobust.dtx}] The source file.
% \item[\CTAN{macros/latex/contrib/oberdiek/makerobust.pdf}] Documentation.
% \end{description}
%
%
% \paragraph{Bundle.} All the packages of the bundle `oberdiek'
% are also available in a TDS compliant ZIP archive. There
% the packages are already unpacked and the documentation files
% are generated. The files and directories obey the TDS standard.
% \begin{description}
% \item[\CTAN{install/macros/latex/contrib/oberdiek.tds.zip}]
% \end{description}
% \emph{TDS} refers to the standard ``A Directory Structure
% for \TeX\ Files'' (\CTAN{tds/tds.pdf}). Directories
% with \xfile{texmf} in their name are usually organized this way.
%
% \subsection{Bundle installation}
%
% \paragraph{Unpacking.} Unpack the \xfile{oberdiek.tds.zip} in the
% TDS tree (also known as \xfile{texmf} tree) of your choice.
% Example (linux):
% \begin{quote}
%   |unzip oberdiek.tds.zip -d ~/texmf|
% \end{quote}
%
% \paragraph{Script installation.}
% Check the directory \xfile{TDS:scripts/oberdiek/} for
% scripts that need further installation steps.
% Package \xpackage{attachfile2} comes with the Perl script
% \xfile{pdfatfi.pl} that should be installed in such a way
% that it can be called as \texttt{pdfatfi}.
% Example (linux):
% \begin{quote}
%   |chmod +x scripts/oberdiek/pdfatfi.pl|\\
%   |cp scripts/oberdiek/pdfatfi.pl /usr/local/bin/|
% \end{quote}
%
% \subsection{Package installation}
%
% \paragraph{Unpacking.} The \xfile{.dtx} file is a self-extracting
% \docstrip\ archive. The files are extracted by running the
% \xfile{.dtx} through \plainTeX:
% \begin{quote}
%   \verb|tex makerobust.dtx|
% \end{quote}
%
% \paragraph{TDS.} Now the different files must be moved into
% the different directories in your installation TDS tree
% (also known as \xfile{texmf} tree):
% \begin{quote}
% \def\t{^^A
% \begin{tabular}{@{}>{\ttfamily}l@{ $\rightarrow$ }>{\ttfamily}l@{}}
%   makerobust.sty & tex/latex/oberdiek/makerobust.sty\\
%   makerobust.pdf & doc/latex/oberdiek/makerobust.pdf\\
%   makerobust-example.tex & doc/latex/oberdiek/makerobust-example.tex\\
%   makerobust.dtx & source/latex/oberdiek/makerobust.dtx\\
% \end{tabular}^^A
% }^^A
% \sbox0{\t}^^A
% \ifdim\wd0>\linewidth
%   \begingroup
%     \advance\linewidth by\leftmargin
%     \advance\linewidth by\rightmargin
%   \edef\x{\endgroup
%     \def\noexpand\lw{\the\linewidth}^^A
%   }\x
%   \def\lwbox{^^A
%     \leavevmode
%     \hbox to \linewidth{^^A
%       \kern-\leftmargin\relax
%       \hss
%       \usebox0
%       \hss
%       \kern-\rightmargin\relax
%     }^^A
%   }^^A
%   \ifdim\wd0>\lw
%     \sbox0{\small\t}^^A
%     \ifdim\wd0>\linewidth
%       \ifdim\wd0>\lw
%         \sbox0{\footnotesize\t}^^A
%         \ifdim\wd0>\linewidth
%           \ifdim\wd0>\lw
%             \sbox0{\scriptsize\t}^^A
%             \ifdim\wd0>\linewidth
%               \ifdim\wd0>\lw
%                 \sbox0{\tiny\t}^^A
%                 \ifdim\wd0>\linewidth
%                   \lwbox
%                 \else
%                   \usebox0
%                 \fi
%               \else
%                 \lwbox
%               \fi
%             \else
%               \usebox0
%             \fi
%           \else
%             \lwbox
%           \fi
%         \else
%           \usebox0
%         \fi
%       \else
%         \lwbox
%       \fi
%     \else
%       \usebox0
%     \fi
%   \else
%     \lwbox
%   \fi
% \else
%   \usebox0
% \fi
% \end{quote}
% If you have a \xfile{docstrip.cfg} that configures and enables \docstrip's
% TDS installing feature, then some files can already be in the right
% place, see the documentation of \docstrip.
%
% \subsection{Refresh file name databases}
%
% If your \TeX~distribution
% (\teTeX, \mikTeX, \dots) relies on file name databases, you must refresh
% these. For example, \teTeX\ users run \verb|texhash| or
% \verb|mktexlsr|.
%
% \subsection{Some details for the interested}
%
% \paragraph{Attached source.}
%
% The PDF documentation on CTAN also includes the
% \xfile{.dtx} source file. It can be extracted by
% AcrobatReader 6 or higher. Another option is \textsf{pdftk},
% e.g. unpack the file into the current directory:
% \begin{quote}
%   \verb|pdftk makerobust.pdf unpack_files output .|
% \end{quote}
%
% \paragraph{Unpacking with \LaTeX.}
% The \xfile{.dtx} chooses its action depending on the format:
% \begin{description}
% \item[\plainTeX:] Run \docstrip\ and extract the files.
% \item[\LaTeX:] Generate the documentation.
% \end{description}
% If you insist on using \LaTeX\ for \docstrip\ (really,
% \docstrip\ does not need \LaTeX), then inform the autodetect routine
% about your intention:
% \begin{quote}
%   \verb|latex \let\install=y\input{makerobust.dtx}|
% \end{quote}
% Do not forget to quote the argument according to the demands
% of your shell.
%
% \paragraph{Generating the documentation.}
% You can use both the \xfile{.dtx} or the \xfile{.drv} to generate
% the documentation. The process can be configured by the
% configuration file \xfile{ltxdoc.cfg}. For instance, put this
% line into this file, if you want to have A4 as paper format:
% \begin{quote}
%   \verb|\PassOptionsToClass{a4paper}{article}|
% \end{quote}
% An example follows how to generate the
% documentation with pdf\LaTeX:
% \begin{quote}
%\begin{verbatim}
%pdflatex makerobust.dtx
%makeindex -s gind.ist makerobust.idx
%pdflatex makerobust.dtx
%makeindex -s gind.ist makerobust.idx
%pdflatex makerobust.dtx
%\end{verbatim}
% \end{quote}
%
% \section{Catalogue}
%
% The following XML file can be used as source for the
% \href{http://mirror.ctan.org/help/Catalogue/catalogue.html}{\TeX\ Catalogue}.
% The elements \texttt{caption} and \texttt{description} are imported
% from the original XML file from the Catalogue.
% The name of the XML file in the Catalogue is \xfile{makerobust.xml}.
%    \begin{macrocode}
%<*catalogue>
<?xml version='1.0' encoding='us-ascii'?>
<!DOCTYPE entry SYSTEM 'catalogue.dtd'>
<entry datestamp='$Date$' modifier='$Author$' id='makerobust'>
  <name>makerobust</name>
  <caption>Making a macro robust.</caption>
  <authorref id='auth:oberdiek'/>
  <copyright owner='Heiko Oberdiek' year='2006'/>
  <license type='lppl1.3'/>
  <version number='1.0'/>
  <description>
    This package provides the command MakeRobustCommand
    that converts an existing macro to a robust one.
    <p/>
    The package is part of the <xref refid='oberdiek'>oberdiek</xref>
    bundle.
  </description>
  <documentation details='Package documentation'
      href='ctan:/macros/latex/contrib/oberdiek/makerobust.pdf'/>
  <ctan file='true' path='/macros/latex/contrib/oberdiek/makerobust.dtx'/>
  <miktex location='oberdiek'/>
  <texlive location='oberdiek'/>
  <install path='/macros/latex/contrib/oberdiek/oberdiek.tds.zip'/>
</entry>
%</catalogue>
%    \end{macrocode}
%
% \begin{History}
%   \begin{Version}{2006/03/18 v1.0}
%   \item
%     First version.
%   \end{Version}
% \end{History}
%
% \PrintIndex
%
% \Finale
\endinput

%        (quote the arguments according to the demands of your shell)
%
% Documentation:
%    (a) If makerobust.drv is present:
%           latex makerobust.drv
%    (b) Without makerobust.drv:
%           latex makerobust.dtx; ...
%    The class ltxdoc loads the configuration file ltxdoc.cfg
%    if available. Here you can specify further options, e.g.
%    use A4 as paper format:
%       \PassOptionsToClass{a4paper}{article}
%
%    Programm calls to get the documentation (example):
%       pdflatex makerobust.dtx
%       makeindex -s gind.ist makerobust.idx
%       pdflatex makerobust.dtx
%       makeindex -s gind.ist makerobust.idx
%       pdflatex makerobust.dtx
%
% Installation:
%    TDS:tex/latex/oberdiek/makerobust.sty
%    TDS:doc/latex/oberdiek/makerobust.pdf
%    TDS:doc/latex/oberdiek/makerobust-example.tex
%    TDS:source/latex/oberdiek/makerobust.dtx
%
%<*ignore>
\begingroup
  \catcode123=1 %
  \catcode125=2 %
  \def\x{LaTeX2e}%
\expandafter\endgroup
\ifcase 0\ifx\install y1\fi\expandafter
         \ifx\csname processbatchFile\endcsname\relax\else1\fi
         \ifx\fmtname\x\else 1\fi\relax
\else\csname fi\endcsname
%</ignore>
%<*install>
\input docstrip.tex
\Msg{************************************************************************}
\Msg{* Installation}
\Msg{* Package: makerobust 2006/03/18 v1.0 Make existing macro robust (HO)}
\Msg{************************************************************************}

\keepsilent
\askforoverwritefalse

\let\MetaPrefix\relax
\preamble

This is a generated file.

Project: makerobust
Version: 2006/03/18 v1.0

Copyright (C) 2006 by
   Heiko Oberdiek <heiko.oberdiek at googlemail.com>

This work may be distributed and/or modified under the
conditions of the LaTeX Project Public License, either
version 1.3c of this license or (at your option) any later
version. This version of this license is in
   http://www.latex-project.org/lppl/lppl-1-3c.txt
and the latest version of this license is in
   http://www.latex-project.org/lppl.txt
and version 1.3 or later is part of all distributions of
LaTeX version 2005/12/01 or later.

This work has the LPPL maintenance status "maintained".

This Current Maintainer of this work is Heiko Oberdiek.

This work consists of the main source file makerobust.dtx
and the derived files
   makerobust.sty, makerobust.pdf, makerobust.ins, makerobust.drv,
   makerobust-example.tex.

\endpreamble
\let\MetaPrefix\DoubleperCent

\generate{%
  \file{makerobust.ins}{\from{makerobust.dtx}{install}}%
  \file{makerobust.drv}{\from{makerobust.dtx}{driver}}%
  \usedir{tex/latex/oberdiek}%
  \file{makerobust.sty}{\from{makerobust.dtx}{package}}%
  \usedir{doc/latex/oberdiek}%
  \file{makerobust-example.tex}{\from{makerobust.dtx}{example}}%
  \nopreamble
  \nopostamble
  \usedir{source/latex/oberdiek/catalogue}%
  \file{makerobust.xml}{\from{makerobust.dtx}{catalogue}}%
}

\catcode32=13\relax% active space
\let =\space%
\Msg{************************************************************************}
\Msg{*}
\Msg{* To finish the installation you have to move the following}
\Msg{* file into a directory searched by TeX:}
\Msg{*}
\Msg{*     makerobust.sty}
\Msg{*}
\Msg{* To produce the documentation run the file `makerobust.drv'}
\Msg{* through LaTeX.}
\Msg{*}
\Msg{* Happy TeXing!}
\Msg{*}
\Msg{************************************************************************}

\endbatchfile
%</install>
%<*ignore>
\fi
%</ignore>
%<*driver>
\NeedsTeXFormat{LaTeX2e}
\ProvidesFile{makerobust.drv}%
  [2006/03/18 v1.0 Make existing macro robust (HO)]%
\documentclass{ltxdoc}
\usepackage{holtxdoc}[2011/11/22]
\begin{document}
  \DocInput{makerobust.dtx}%
\end{document}
%</driver>
% \fi
%
% \CheckSum{59}
%
% \CharacterTable
%  {Upper-case    \A\B\C\D\E\F\G\H\I\J\K\L\M\N\O\P\Q\R\S\T\U\V\W\X\Y\Z
%   Lower-case    \a\b\c\d\e\f\g\h\i\j\k\l\m\n\o\p\q\r\s\t\u\v\w\x\y\z
%   Digits        \0\1\2\3\4\5\6\7\8\9
%   Exclamation   \!     Double quote  \"     Hash (number) \#
%   Dollar        \$     Percent       \%     Ampersand     \&
%   Acute accent  \'     Left paren    \(     Right paren   \)
%   Asterisk      \*     Plus          \+     Comma         \,
%   Minus         \-     Point         \.     Solidus       \/
%   Colon         \:     Semicolon     \;     Less than     \<
%   Equals        \=     Greater than  \>     Question mark \?
%   Commercial at \@     Left bracket  \[     Backslash     \\
%   Right bracket \]     Circumflex    \^     Underscore    \_
%   Grave accent  \`     Left brace    \{     Vertical bar  \|
%   Right brace   \}     Tilde         \~}
%
% \GetFileInfo{makerobust.drv}
%
% \title{The \xpackage{makerobust} package}
% \date{2006/03/18 v1.0}
% \author{Heiko Oberdiek\\\xemail{heiko.oberdiek at googlemail.com}}
%
% \maketitle
%
% \begin{abstract}
% Package \xpackage{makerobust} provides \cs{MakeRobustCommand}
% that converts an existing macro to a robust one.
% \end{abstract}
%
% \tableofcontents
%
% \section{User interface}
%
% \LaTeX\ offers \cs{DeclareRobustCommand} to define a robust macro
% that does not break if it is used in moving arguments.
% Sometimes a macro is already defined, but not robust. For
% example, \cs{(} and \cs{)} are not robust, inside \cs{section}
% the user must use \cs{protect} explicitly. This could be
% avoided by making \cs{(} and \cs{)} robust.
%
% \begin{declcs}{MakeRobustCommand}\M{cmd}
% \end{declcs}
% \cs{MakeRobustCommand} redefines the macro \meta{cmd}
% by using \cs{DeclareRobustCommand} and the existing definition
% of the macro \meta{cmd}.
% \begin{itemize}
% \item It is an error if \meta{cmd} is undefined. If you want to
%   define a robust command, then you can use \cs{DeclareRobustCommand}
%   directly.
% \item If the macro has previously been
%   defined by \cs{DeclareRobustCommand} then the redefinition of
%   \cs{MakeRobustCommand} is omitted, because the macro is already robust.
%   Only an information entry is written to the \xfile{.log} file.
%   Thus you do not get a warning or an error if the macro is already
%   robust because of an updated LaTeX or package that defines the macro.
% \item Two macros are defined for a macro, defined
%   by \cs{DeclareRobustCommand}. Example:
%   \begin{quote}
%   |\DeclareRobustCommand{\foobar}{definition text}|
%   \end{quote}
%   Then the macro ``\cs{foobar}'' contains the protection code
%   and, depending on the protection mode,
%   calls the internal macro ``\cs{foobar }''. Notice the space
%   at the end of the macro name.
%   This internal macro ``\cs{foobar }'' now contains the definition
%   ``|definition text|'', given in \cs{DeclareRobustCommand}.
%
%   Sometimes it can happen, that the internal macro already exists.
%   This can be caused by a previous \cs{DeclareRobustCommand} followed
%   by \cs{renewcommand}. Then the redefinition by \cs{MakeRobustCommand}
%   would be safe.
%
%   However, it can also be possible that the macro is already robust,
%   using the internal macro, but with a different protection code.
%   The redefinition by \cs{MakeRobustCommand} would then generate
%   an infinite loop.
%
%   Therefore \cs{MakeRobustCommand} raises an error message,
%   if the internal macro (with space at the end) already exists.
% \end{itemize}
%
% \subsection{Example}
%
%    \begin{macrocode}
%<*example>
\documentclass{article}
\usepackage{makerobust}
\MakeRobustCommand\(
\MakeRobustCommand\)
\pagestyle{headings}
\begin{document}
\tableofcontents
\section{Einstein: \(E=mc^2\)}
\newpage
Second page.
\end{document}
%</example>
%    \end{macrocode}
%
%
% \StopEventually{
% }
%
% \section{Implementation}
%
%    \begin{macrocode}
%<*package>
\NeedsTeXFormat{LaTeX2e}
\ProvidesPackage{makerobust}%
  [2006/03/18 v1.0 Make existing macro robust (HO)]%
%    \end{macrocode}
%
%    \begin{macrocode}
\def\MakeRobustCommand#1{%
  \begingroup
  \@ifundefined{\expandafter\@gobble\string#1}{%
    \endgroup
    \PackageError{makerobust}{%
      Macro \string`\string#1\string' is not defined%
    }\@ehc
  }{%
    \global\let\MR@gtemp#1%
    \let#1\@undefined
    \expandafter\let\expandafter\MR@temp
        \csname\expandafter\@gobble\string#1 \endcsname
    \DeclareRobustCommand#1{}%
    \ifx#1\MR@gtemp
      \endgroup
      \PackageInfo{makerobust}{%
        \string`\string#1\string' is already robust%
      }%
    \else
      \@ifundefined{MR@temp}{%
        \global\let\MR@gtemp#1%
        \endgroup
        \expandafter\let\csname\expandafter\@gobble\string#1 \endcsname#1%
        \let#1\MR@gtemp
      }{%
        \endgroup
        \PackageError{makerobust}{%
          Internal macro \string`\string#1 \string' already exists%
        }\@ehc
      }%
    \fi
  }%
}
%    \end{macrocode}
%
%    \begin{macrocode}
%</package>
%    \end{macrocode}
%
% \section{Installation}
%
% \subsection{Download}
%
% \paragraph{Package.} This package is available on
% CTAN\footnote{\url{ftp://ftp.ctan.org/tex-archive/}}:
% \begin{description}
% \item[\CTAN{macros/latex/contrib/oberdiek/makerobust.dtx}] The source file.
% \item[\CTAN{macros/latex/contrib/oberdiek/makerobust.pdf}] Documentation.
% \end{description}
%
%
% \paragraph{Bundle.} All the packages of the bundle `oberdiek'
% are also available in a TDS compliant ZIP archive. There
% the packages are already unpacked and the documentation files
% are generated. The files and directories obey the TDS standard.
% \begin{description}
% \item[\CTAN{install/macros/latex/contrib/oberdiek.tds.zip}]
% \end{description}
% \emph{TDS} refers to the standard ``A Directory Structure
% for \TeX\ Files'' (\CTAN{tds/tds.pdf}). Directories
% with \xfile{texmf} in their name are usually organized this way.
%
% \subsection{Bundle installation}
%
% \paragraph{Unpacking.} Unpack the \xfile{oberdiek.tds.zip} in the
% TDS tree (also known as \xfile{texmf} tree) of your choice.
% Example (linux):
% \begin{quote}
%   |unzip oberdiek.tds.zip -d ~/texmf|
% \end{quote}
%
% \paragraph{Script installation.}
% Check the directory \xfile{TDS:scripts/oberdiek/} for
% scripts that need further installation steps.
% Package \xpackage{attachfile2} comes with the Perl script
% \xfile{pdfatfi.pl} that should be installed in such a way
% that it can be called as \texttt{pdfatfi}.
% Example (linux):
% \begin{quote}
%   |chmod +x scripts/oberdiek/pdfatfi.pl|\\
%   |cp scripts/oberdiek/pdfatfi.pl /usr/local/bin/|
% \end{quote}
%
% \subsection{Package installation}
%
% \paragraph{Unpacking.} The \xfile{.dtx} file is a self-extracting
% \docstrip\ archive. The files are extracted by running the
% \xfile{.dtx} through \plainTeX:
% \begin{quote}
%   \verb|tex makerobust.dtx|
% \end{quote}
%
% \paragraph{TDS.} Now the different files must be moved into
% the different directories in your installation TDS tree
% (also known as \xfile{texmf} tree):
% \begin{quote}
% \def\t{^^A
% \begin{tabular}{@{}>{\ttfamily}l@{ $\rightarrow$ }>{\ttfamily}l@{}}
%   makerobust.sty & tex/latex/oberdiek/makerobust.sty\\
%   makerobust.pdf & doc/latex/oberdiek/makerobust.pdf\\
%   makerobust-example.tex & doc/latex/oberdiek/makerobust-example.tex\\
%   makerobust.dtx & source/latex/oberdiek/makerobust.dtx\\
% \end{tabular}^^A
% }^^A
% \sbox0{\t}^^A
% \ifdim\wd0>\linewidth
%   \begingroup
%     \advance\linewidth by\leftmargin
%     \advance\linewidth by\rightmargin
%   \edef\x{\endgroup
%     \def\noexpand\lw{\the\linewidth}^^A
%   }\x
%   \def\lwbox{^^A
%     \leavevmode
%     \hbox to \linewidth{^^A
%       \kern-\leftmargin\relax
%       \hss
%       \usebox0
%       \hss
%       \kern-\rightmargin\relax
%     }^^A
%   }^^A
%   \ifdim\wd0>\lw
%     \sbox0{\small\t}^^A
%     \ifdim\wd0>\linewidth
%       \ifdim\wd0>\lw
%         \sbox0{\footnotesize\t}^^A
%         \ifdim\wd0>\linewidth
%           \ifdim\wd0>\lw
%             \sbox0{\scriptsize\t}^^A
%             \ifdim\wd0>\linewidth
%               \ifdim\wd0>\lw
%                 \sbox0{\tiny\t}^^A
%                 \ifdim\wd0>\linewidth
%                   \lwbox
%                 \else
%                   \usebox0
%                 \fi
%               \else
%                 \lwbox
%               \fi
%             \else
%               \usebox0
%             \fi
%           \else
%             \lwbox
%           \fi
%         \else
%           \usebox0
%         \fi
%       \else
%         \lwbox
%       \fi
%     \else
%       \usebox0
%     \fi
%   \else
%     \lwbox
%   \fi
% \else
%   \usebox0
% \fi
% \end{quote}
% If you have a \xfile{docstrip.cfg} that configures and enables \docstrip's
% TDS installing feature, then some files can already be in the right
% place, see the documentation of \docstrip.
%
% \subsection{Refresh file name databases}
%
% If your \TeX~distribution
% (\teTeX, \mikTeX, \dots) relies on file name databases, you must refresh
% these. For example, \teTeX\ users run \verb|texhash| or
% \verb|mktexlsr|.
%
% \subsection{Some details for the interested}
%
% \paragraph{Attached source.}
%
% The PDF documentation on CTAN also includes the
% \xfile{.dtx} source file. It can be extracted by
% AcrobatReader 6 or higher. Another option is \textsf{pdftk},
% e.g. unpack the file into the current directory:
% \begin{quote}
%   \verb|pdftk makerobust.pdf unpack_files output .|
% \end{quote}
%
% \paragraph{Unpacking with \LaTeX.}
% The \xfile{.dtx} chooses its action depending on the format:
% \begin{description}
% \item[\plainTeX:] Run \docstrip\ and extract the files.
% \item[\LaTeX:] Generate the documentation.
% \end{description}
% If you insist on using \LaTeX\ for \docstrip\ (really,
% \docstrip\ does not need \LaTeX), then inform the autodetect routine
% about your intention:
% \begin{quote}
%   \verb|latex \let\install=y% \iffalse meta-comment
%
% File: makerobust.dtx
% Version: 2006/03/18 v1.0
% Info: Make existing macro robust
%
% Copyright (C) 2006 by
%    Heiko Oberdiek <heiko.oberdiek at googlemail.com>
%
% This work may be distributed and/or modified under the
% conditions of the LaTeX Project Public License, either
% version 1.3c of this license or (at your option) any later
% version. This version of this license is in
%    http://www.latex-project.org/lppl/lppl-1-3c.txt
% and the latest version of this license is in
%    http://www.latex-project.org/lppl.txt
% and version 1.3 or later is part of all distributions of
% LaTeX version 2005/12/01 or later.
%
% This work has the LPPL maintenance status "maintained".
%
% This Current Maintainer of this work is Heiko Oberdiek.
%
% This work consists of the main source file makerobust.dtx
% and the derived files
%    makerobust.sty, makerobust.pdf, makerobust.ins, makerobust.drv,
%    makerobust-example.tex.
%
% Distribution:
%    CTAN:macros/latex/contrib/oberdiek/makerobust.dtx
%    CTAN:macros/latex/contrib/oberdiek/makerobust.pdf
%
% Unpacking:
%    (a) If makerobust.ins is present:
%           tex makerobust.ins
%    (b) Without makerobust.ins:
%           tex makerobust.dtx
%    (c) If you insist on using LaTeX
%           latex \let\install=y\input{makerobust.dtx}
%        (quote the arguments according to the demands of your shell)
%
% Documentation:
%    (a) If makerobust.drv is present:
%           latex makerobust.drv
%    (b) Without makerobust.drv:
%           latex makerobust.dtx; ...
%    The class ltxdoc loads the configuration file ltxdoc.cfg
%    if available. Here you can specify further options, e.g.
%    use A4 as paper format:
%       \PassOptionsToClass{a4paper}{article}
%
%    Programm calls to get the documentation (example):
%       pdflatex makerobust.dtx
%       makeindex -s gind.ist makerobust.idx
%       pdflatex makerobust.dtx
%       makeindex -s gind.ist makerobust.idx
%       pdflatex makerobust.dtx
%
% Installation:
%    TDS:tex/latex/oberdiek/makerobust.sty
%    TDS:doc/latex/oberdiek/makerobust.pdf
%    TDS:doc/latex/oberdiek/makerobust-example.tex
%    TDS:source/latex/oberdiek/makerobust.dtx
%
%<*ignore>
\begingroup
  \catcode123=1 %
  \catcode125=2 %
  \def\x{LaTeX2e}%
\expandafter\endgroup
\ifcase 0\ifx\install y1\fi\expandafter
         \ifx\csname processbatchFile\endcsname\relax\else1\fi
         \ifx\fmtname\x\else 1\fi\relax
\else\csname fi\endcsname
%</ignore>
%<*install>
\input docstrip.tex
\Msg{************************************************************************}
\Msg{* Installation}
\Msg{* Package: makerobust 2006/03/18 v1.0 Make existing macro robust (HO)}
\Msg{************************************************************************}

\keepsilent
\askforoverwritefalse

\let\MetaPrefix\relax
\preamble

This is a generated file.

Project: makerobust
Version: 2006/03/18 v1.0

Copyright (C) 2006 by
   Heiko Oberdiek <heiko.oberdiek at googlemail.com>

This work may be distributed and/or modified under the
conditions of the LaTeX Project Public License, either
version 1.3c of this license or (at your option) any later
version. This version of this license is in
   http://www.latex-project.org/lppl/lppl-1-3c.txt
and the latest version of this license is in
   http://www.latex-project.org/lppl.txt
and version 1.3 or later is part of all distributions of
LaTeX version 2005/12/01 or later.

This work has the LPPL maintenance status "maintained".

This Current Maintainer of this work is Heiko Oberdiek.

This work consists of the main source file makerobust.dtx
and the derived files
   makerobust.sty, makerobust.pdf, makerobust.ins, makerobust.drv,
   makerobust-example.tex.

\endpreamble
\let\MetaPrefix\DoubleperCent

\generate{%
  \file{makerobust.ins}{\from{makerobust.dtx}{install}}%
  \file{makerobust.drv}{\from{makerobust.dtx}{driver}}%
  \usedir{tex/latex/oberdiek}%
  \file{makerobust.sty}{\from{makerobust.dtx}{package}}%
  \usedir{doc/latex/oberdiek}%
  \file{makerobust-example.tex}{\from{makerobust.dtx}{example}}%
  \nopreamble
  \nopostamble
  \usedir{source/latex/oberdiek/catalogue}%
  \file{makerobust.xml}{\from{makerobust.dtx}{catalogue}}%
}

\catcode32=13\relax% active space
\let =\space%
\Msg{************************************************************************}
\Msg{*}
\Msg{* To finish the installation you have to move the following}
\Msg{* file into a directory searched by TeX:}
\Msg{*}
\Msg{*     makerobust.sty}
\Msg{*}
\Msg{* To produce the documentation run the file `makerobust.drv'}
\Msg{* through LaTeX.}
\Msg{*}
\Msg{* Happy TeXing!}
\Msg{*}
\Msg{************************************************************************}

\endbatchfile
%</install>
%<*ignore>
\fi
%</ignore>
%<*driver>
\NeedsTeXFormat{LaTeX2e}
\ProvidesFile{makerobust.drv}%
  [2006/03/18 v1.0 Make existing macro robust (HO)]%
\documentclass{ltxdoc}
\usepackage{holtxdoc}[2011/11/22]
\begin{document}
  \DocInput{makerobust.dtx}%
\end{document}
%</driver>
% \fi
%
% \CheckSum{59}
%
% \CharacterTable
%  {Upper-case    \A\B\C\D\E\F\G\H\I\J\K\L\M\N\O\P\Q\R\S\T\U\V\W\X\Y\Z
%   Lower-case    \a\b\c\d\e\f\g\h\i\j\k\l\m\n\o\p\q\r\s\t\u\v\w\x\y\z
%   Digits        \0\1\2\3\4\5\6\7\8\9
%   Exclamation   \!     Double quote  \"     Hash (number) \#
%   Dollar        \$     Percent       \%     Ampersand     \&
%   Acute accent  \'     Left paren    \(     Right paren   \)
%   Asterisk      \*     Plus          \+     Comma         \,
%   Minus         \-     Point         \.     Solidus       \/
%   Colon         \:     Semicolon     \;     Less than     \<
%   Equals        \=     Greater than  \>     Question mark \?
%   Commercial at \@     Left bracket  \[     Backslash     \\
%   Right bracket \]     Circumflex    \^     Underscore    \_
%   Grave accent  \`     Left brace    \{     Vertical bar  \|
%   Right brace   \}     Tilde         \~}
%
% \GetFileInfo{makerobust.drv}
%
% \title{The \xpackage{makerobust} package}
% \date{2006/03/18 v1.0}
% \author{Heiko Oberdiek\\\xemail{heiko.oberdiek at googlemail.com}}
%
% \maketitle
%
% \begin{abstract}
% Package \xpackage{makerobust} provides \cs{MakeRobustCommand}
% that converts an existing macro to a robust one.
% \end{abstract}
%
% \tableofcontents
%
% \section{User interface}
%
% \LaTeX\ offers \cs{DeclareRobustCommand} to define a robust macro
% that does not break if it is used in moving arguments.
% Sometimes a macro is already defined, but not robust. For
% example, \cs{(} and \cs{)} are not robust, inside \cs{section}
% the user must use \cs{protect} explicitly. This could be
% avoided by making \cs{(} and \cs{)} robust.
%
% \begin{declcs}{MakeRobustCommand}\M{cmd}
% \end{declcs}
% \cs{MakeRobustCommand} redefines the macro \meta{cmd}
% by using \cs{DeclareRobustCommand} and the existing definition
% of the macro \meta{cmd}.
% \begin{itemize}
% \item It is an error if \meta{cmd} is undefined. If you want to
%   define a robust command, then you can use \cs{DeclareRobustCommand}
%   directly.
% \item If the macro has previously been
%   defined by \cs{DeclareRobustCommand} then the redefinition of
%   \cs{MakeRobustCommand} is omitted, because the macro is already robust.
%   Only an information entry is written to the \xfile{.log} file.
%   Thus you do not get a warning or an error if the macro is already
%   robust because of an updated LaTeX or package that defines the macro.
% \item Two macros are defined for a macro, defined
%   by \cs{DeclareRobustCommand}. Example:
%   \begin{quote}
%   |\DeclareRobustCommand{\foobar}{definition text}|
%   \end{quote}
%   Then the macro ``\cs{foobar}'' contains the protection code
%   and, depending on the protection mode,
%   calls the internal macro ``\cs{foobar }''. Notice the space
%   at the end of the macro name.
%   This internal macro ``\cs{foobar }'' now contains the definition
%   ``|definition text|'', given in \cs{DeclareRobustCommand}.
%
%   Sometimes it can happen, that the internal macro already exists.
%   This can be caused by a previous \cs{DeclareRobustCommand} followed
%   by \cs{renewcommand}. Then the redefinition by \cs{MakeRobustCommand}
%   would be safe.
%
%   However, it can also be possible that the macro is already robust,
%   using the internal macro, but with a different protection code.
%   The redefinition by \cs{MakeRobustCommand} would then generate
%   an infinite loop.
%
%   Therefore \cs{MakeRobustCommand} raises an error message,
%   if the internal macro (with space at the end) already exists.
% \end{itemize}
%
% \subsection{Example}
%
%    \begin{macrocode}
%<*example>
\documentclass{article}
\usepackage{makerobust}
\MakeRobustCommand\(
\MakeRobustCommand\)
\pagestyle{headings}
\begin{document}
\tableofcontents
\section{Einstein: \(E=mc^2\)}
\newpage
Second page.
\end{document}
%</example>
%    \end{macrocode}
%
%
% \StopEventually{
% }
%
% \section{Implementation}
%
%    \begin{macrocode}
%<*package>
\NeedsTeXFormat{LaTeX2e}
\ProvidesPackage{makerobust}%
  [2006/03/18 v1.0 Make existing macro robust (HO)]%
%    \end{macrocode}
%
%    \begin{macrocode}
\def\MakeRobustCommand#1{%
  \begingroup
  \@ifundefined{\expandafter\@gobble\string#1}{%
    \endgroup
    \PackageError{makerobust}{%
      Macro \string`\string#1\string' is not defined%
    }\@ehc
  }{%
    \global\let\MR@gtemp#1%
    \let#1\@undefined
    \expandafter\let\expandafter\MR@temp
        \csname\expandafter\@gobble\string#1 \endcsname
    \DeclareRobustCommand#1{}%
    \ifx#1\MR@gtemp
      \endgroup
      \PackageInfo{makerobust}{%
        \string`\string#1\string' is already robust%
      }%
    \else
      \@ifundefined{MR@temp}{%
        \global\let\MR@gtemp#1%
        \endgroup
        \expandafter\let\csname\expandafter\@gobble\string#1 \endcsname#1%
        \let#1\MR@gtemp
      }{%
        \endgroup
        \PackageError{makerobust}{%
          Internal macro \string`\string#1 \string' already exists%
        }\@ehc
      }%
    \fi
  }%
}
%    \end{macrocode}
%
%    \begin{macrocode}
%</package>
%    \end{macrocode}
%
% \section{Installation}
%
% \subsection{Download}
%
% \paragraph{Package.} This package is available on
% CTAN\footnote{\url{ftp://ftp.ctan.org/tex-archive/}}:
% \begin{description}
% \item[\CTAN{macros/latex/contrib/oberdiek/makerobust.dtx}] The source file.
% \item[\CTAN{macros/latex/contrib/oberdiek/makerobust.pdf}] Documentation.
% \end{description}
%
%
% \paragraph{Bundle.} All the packages of the bundle `oberdiek'
% are also available in a TDS compliant ZIP archive. There
% the packages are already unpacked and the documentation files
% are generated. The files and directories obey the TDS standard.
% \begin{description}
% \item[\CTAN{install/macros/latex/contrib/oberdiek.tds.zip}]
% \end{description}
% \emph{TDS} refers to the standard ``A Directory Structure
% for \TeX\ Files'' (\CTAN{tds/tds.pdf}). Directories
% with \xfile{texmf} in their name are usually organized this way.
%
% \subsection{Bundle installation}
%
% \paragraph{Unpacking.} Unpack the \xfile{oberdiek.tds.zip} in the
% TDS tree (also known as \xfile{texmf} tree) of your choice.
% Example (linux):
% \begin{quote}
%   |unzip oberdiek.tds.zip -d ~/texmf|
% \end{quote}
%
% \paragraph{Script installation.}
% Check the directory \xfile{TDS:scripts/oberdiek/} for
% scripts that need further installation steps.
% Package \xpackage{attachfile2} comes with the Perl script
% \xfile{pdfatfi.pl} that should be installed in such a way
% that it can be called as \texttt{pdfatfi}.
% Example (linux):
% \begin{quote}
%   |chmod +x scripts/oberdiek/pdfatfi.pl|\\
%   |cp scripts/oberdiek/pdfatfi.pl /usr/local/bin/|
% \end{quote}
%
% \subsection{Package installation}
%
% \paragraph{Unpacking.} The \xfile{.dtx} file is a self-extracting
% \docstrip\ archive. The files are extracted by running the
% \xfile{.dtx} through \plainTeX:
% \begin{quote}
%   \verb|tex makerobust.dtx|
% \end{quote}
%
% \paragraph{TDS.} Now the different files must be moved into
% the different directories in your installation TDS tree
% (also known as \xfile{texmf} tree):
% \begin{quote}
% \def\t{^^A
% \begin{tabular}{@{}>{\ttfamily}l@{ $\rightarrow$ }>{\ttfamily}l@{}}
%   makerobust.sty & tex/latex/oberdiek/makerobust.sty\\
%   makerobust.pdf & doc/latex/oberdiek/makerobust.pdf\\
%   makerobust-example.tex & doc/latex/oberdiek/makerobust-example.tex\\
%   makerobust.dtx & source/latex/oberdiek/makerobust.dtx\\
% \end{tabular}^^A
% }^^A
% \sbox0{\t}^^A
% \ifdim\wd0>\linewidth
%   \begingroup
%     \advance\linewidth by\leftmargin
%     \advance\linewidth by\rightmargin
%   \edef\x{\endgroup
%     \def\noexpand\lw{\the\linewidth}^^A
%   }\x
%   \def\lwbox{^^A
%     \leavevmode
%     \hbox to \linewidth{^^A
%       \kern-\leftmargin\relax
%       \hss
%       \usebox0
%       \hss
%       \kern-\rightmargin\relax
%     }^^A
%   }^^A
%   \ifdim\wd0>\lw
%     \sbox0{\small\t}^^A
%     \ifdim\wd0>\linewidth
%       \ifdim\wd0>\lw
%         \sbox0{\footnotesize\t}^^A
%         \ifdim\wd0>\linewidth
%           \ifdim\wd0>\lw
%             \sbox0{\scriptsize\t}^^A
%             \ifdim\wd0>\linewidth
%               \ifdim\wd0>\lw
%                 \sbox0{\tiny\t}^^A
%                 \ifdim\wd0>\linewidth
%                   \lwbox
%                 \else
%                   \usebox0
%                 \fi
%               \else
%                 \lwbox
%               \fi
%             \else
%               \usebox0
%             \fi
%           \else
%             \lwbox
%           \fi
%         \else
%           \usebox0
%         \fi
%       \else
%         \lwbox
%       \fi
%     \else
%       \usebox0
%     \fi
%   \else
%     \lwbox
%   \fi
% \else
%   \usebox0
% \fi
% \end{quote}
% If you have a \xfile{docstrip.cfg} that configures and enables \docstrip's
% TDS installing feature, then some files can already be in the right
% place, see the documentation of \docstrip.
%
% \subsection{Refresh file name databases}
%
% If your \TeX~distribution
% (\teTeX, \mikTeX, \dots) relies on file name databases, you must refresh
% these. For example, \teTeX\ users run \verb|texhash| or
% \verb|mktexlsr|.
%
% \subsection{Some details for the interested}
%
% \paragraph{Attached source.}
%
% The PDF documentation on CTAN also includes the
% \xfile{.dtx} source file. It can be extracted by
% AcrobatReader 6 or higher. Another option is \textsf{pdftk},
% e.g. unpack the file into the current directory:
% \begin{quote}
%   \verb|pdftk makerobust.pdf unpack_files output .|
% \end{quote}
%
% \paragraph{Unpacking with \LaTeX.}
% The \xfile{.dtx} chooses its action depending on the format:
% \begin{description}
% \item[\plainTeX:] Run \docstrip\ and extract the files.
% \item[\LaTeX:] Generate the documentation.
% \end{description}
% If you insist on using \LaTeX\ for \docstrip\ (really,
% \docstrip\ does not need \LaTeX), then inform the autodetect routine
% about your intention:
% \begin{quote}
%   \verb|latex \let\install=y\input{makerobust.dtx}|
% \end{quote}
% Do not forget to quote the argument according to the demands
% of your shell.
%
% \paragraph{Generating the documentation.}
% You can use both the \xfile{.dtx} or the \xfile{.drv} to generate
% the documentation. The process can be configured by the
% configuration file \xfile{ltxdoc.cfg}. For instance, put this
% line into this file, if you want to have A4 as paper format:
% \begin{quote}
%   \verb|\PassOptionsToClass{a4paper}{article}|
% \end{quote}
% An example follows how to generate the
% documentation with pdf\LaTeX:
% \begin{quote}
%\begin{verbatim}
%pdflatex makerobust.dtx
%makeindex -s gind.ist makerobust.idx
%pdflatex makerobust.dtx
%makeindex -s gind.ist makerobust.idx
%pdflatex makerobust.dtx
%\end{verbatim}
% \end{quote}
%
% \section{Catalogue}
%
% The following XML file can be used as source for the
% \href{http://mirror.ctan.org/help/Catalogue/catalogue.html}{\TeX\ Catalogue}.
% The elements \texttt{caption} and \texttt{description} are imported
% from the original XML file from the Catalogue.
% The name of the XML file in the Catalogue is \xfile{makerobust.xml}.
%    \begin{macrocode}
%<*catalogue>
<?xml version='1.0' encoding='us-ascii'?>
<!DOCTYPE entry SYSTEM 'catalogue.dtd'>
<entry datestamp='$Date$' modifier='$Author$' id='makerobust'>
  <name>makerobust</name>
  <caption>Making a macro robust.</caption>
  <authorref id='auth:oberdiek'/>
  <copyright owner='Heiko Oberdiek' year='2006'/>
  <license type='lppl1.3'/>
  <version number='1.0'/>
  <description>
    This package provides the command MakeRobustCommand
    that converts an existing macro to a robust one.
    <p/>
    The package is part of the <xref refid='oberdiek'>oberdiek</xref>
    bundle.
  </description>
  <documentation details='Package documentation'
      href='ctan:/macros/latex/contrib/oberdiek/makerobust.pdf'/>
  <ctan file='true' path='/macros/latex/contrib/oberdiek/makerobust.dtx'/>
  <miktex location='oberdiek'/>
  <texlive location='oberdiek'/>
  <install path='/macros/latex/contrib/oberdiek/oberdiek.tds.zip'/>
</entry>
%</catalogue>
%    \end{macrocode}
%
% \begin{History}
%   \begin{Version}{2006/03/18 v1.0}
%   \item
%     First version.
%   \end{Version}
% \end{History}
%
% \PrintIndex
%
% \Finale
\endinput
|
% \end{quote}
% Do not forget to quote the argument according to the demands
% of your shell.
%
% \paragraph{Generating the documentation.}
% You can use both the \xfile{.dtx} or the \xfile{.drv} to generate
% the documentation. The process can be configured by the
% configuration file \xfile{ltxdoc.cfg}. For instance, put this
% line into this file, if you want to have A4 as paper format:
% \begin{quote}
%   \verb|\PassOptionsToClass{a4paper}{article}|
% \end{quote}
% An example follows how to generate the
% documentation with pdf\LaTeX:
% \begin{quote}
%\begin{verbatim}
%pdflatex makerobust.dtx
%makeindex -s gind.ist makerobust.idx
%pdflatex makerobust.dtx
%makeindex -s gind.ist makerobust.idx
%pdflatex makerobust.dtx
%\end{verbatim}
% \end{quote}
%
% \section{Catalogue}
%
% The following XML file can be used as source for the
% \href{http://mirror.ctan.org/help/Catalogue/catalogue.html}{\TeX\ Catalogue}.
% The elements \texttt{caption} and \texttt{description} are imported
% from the original XML file from the Catalogue.
% The name of the XML file in the Catalogue is \xfile{makerobust.xml}.
%    \begin{macrocode}
%<*catalogue>
<?xml version='1.0' encoding='us-ascii'?>
<!DOCTYPE entry SYSTEM 'catalogue.dtd'>
<entry datestamp='$Date$' modifier='$Author$' id='makerobust'>
  <name>makerobust</name>
  <caption>Making a macro robust.</caption>
  <authorref id='auth:oberdiek'/>
  <copyright owner='Heiko Oberdiek' year='2006'/>
  <license type='lppl1.3'/>
  <version number='1.0'/>
  <description>
    This package provides the command MakeRobustCommand
    that converts an existing macro to a robust one.
    <p/>
    The package is part of the <xref refid='oberdiek'>oberdiek</xref>
    bundle.
  </description>
  <documentation details='Package documentation'
      href='ctan:/macros/latex/contrib/oberdiek/makerobust.pdf'/>
  <ctan file='true' path='/macros/latex/contrib/oberdiek/makerobust.dtx'/>
  <miktex location='oberdiek'/>
  <texlive location='oberdiek'/>
  <install path='/macros/latex/contrib/oberdiek/oberdiek.tds.zip'/>
</entry>
%</catalogue>
%    \end{macrocode}
%
% \begin{History}
%   \begin{Version}{2006/03/18 v1.0}
%   \item
%     First version.
%   \end{Version}
% \end{History}
%
% \PrintIndex
%
% \Finale
\endinput
|
% \end{quote}
% Do not forget to quote the argument according to the demands
% of your shell.
%
% \paragraph{Generating the documentation.}
% You can use both the \xfile{.dtx} or the \xfile{.drv} to generate
% the documentation. The process can be configured by the
% configuration file \xfile{ltxdoc.cfg}. For instance, put this
% line into this file, if you want to have A4 as paper format:
% \begin{quote}
%   \verb|\PassOptionsToClass{a4paper}{article}|
% \end{quote}
% An example follows how to generate the
% documentation with pdf\LaTeX:
% \begin{quote}
%\begin{verbatim}
%pdflatex makerobust.dtx
%makeindex -s gind.ist makerobust.idx
%pdflatex makerobust.dtx
%makeindex -s gind.ist makerobust.idx
%pdflatex makerobust.dtx
%\end{verbatim}
% \end{quote}
%
% \section{Catalogue}
%
% The following XML file can be used as source for the
% \href{http://mirror.ctan.org/help/Catalogue/catalogue.html}{\TeX\ Catalogue}.
% The elements \texttt{caption} and \texttt{description} are imported
% from the original XML file from the Catalogue.
% The name of the XML file in the Catalogue is \xfile{makerobust.xml}.
%    \begin{macrocode}
%<*catalogue>
<?xml version='1.0' encoding='us-ascii'?>
<!DOCTYPE entry SYSTEM 'catalogue.dtd'>
<entry datestamp='$Date$' modifier='$Author$' id='makerobust'>
  <name>makerobust</name>
  <caption>Making a macro robust.</caption>
  <authorref id='auth:oberdiek'/>
  <copyright owner='Heiko Oberdiek' year='2006'/>
  <license type='lppl1.3'/>
  <version number='1.0'/>
  <description>
    This package provides the command MakeRobustCommand
    that converts an existing macro to a robust one.
    <p/>
    The package is part of the <xref refid='oberdiek'>oberdiek</xref>
    bundle.
  </description>
  <documentation details='Package documentation'
      href='ctan:/macros/latex/contrib/oberdiek/makerobust.pdf'/>
  <ctan file='true' path='/macros/latex/contrib/oberdiek/makerobust.dtx'/>
  <miktex location='oberdiek'/>
  <texlive location='oberdiek'/>
  <install path='/macros/latex/contrib/oberdiek/oberdiek.tds.zip'/>
</entry>
%</catalogue>
%    \end{macrocode}
%
% \begin{History}
%   \begin{Version}{2006/03/18 v1.0}
%   \item
%     First version.
%   \end{Version}
% \end{History}
%
% \PrintIndex
%
% \Finale
\endinput
|
% \end{quote}
% Do not forget to quote the argument according to the demands
% of your shell.
%
% \paragraph{Generating the documentation.}
% You can use both the \xfile{.dtx} or the \xfile{.drv} to generate
% the documentation. The process can be configured by the
% configuration file \xfile{ltxdoc.cfg}. For instance, put this
% line into this file, if you want to have A4 as paper format:
% \begin{quote}
%   \verb|\PassOptionsToClass{a4paper}{article}|
% \end{quote}
% An example follows how to generate the
% documentation with pdf\LaTeX:
% \begin{quote}
%\begin{verbatim}
%pdflatex makerobust.dtx
%makeindex -s gind.ist makerobust.idx
%pdflatex makerobust.dtx
%makeindex -s gind.ist makerobust.idx
%pdflatex makerobust.dtx
%\end{verbatim}
% \end{quote}
%
% \section{Catalogue}
%
% The following XML file can be used as source for the
% \href{http://mirror.ctan.org/help/Catalogue/catalogue.html}{\TeX\ Catalogue}.
% The elements \texttt{caption} and \texttt{description} are imported
% from the original XML file from the Catalogue.
% The name of the XML file in the Catalogue is \xfile{makerobust.xml}.
%    \begin{macrocode}
%<*catalogue>
<?xml version='1.0' encoding='us-ascii'?>
<!DOCTYPE entry SYSTEM 'catalogue.dtd'>
<entry datestamp='$Date$' modifier='$Author$' id='makerobust'>
  <name>makerobust</name>
  <caption>Making a macro robust.</caption>
  <authorref id='auth:oberdiek'/>
  <copyright owner='Heiko Oberdiek' year='2006'/>
  <license type='lppl1.3'/>
  <version number='1.0'/>
  <description>
    This package provides the command MakeRobustCommand
    that converts an existing macro to a robust one.
    <p/>
    The package is part of the <xref refid='oberdiek'>oberdiek</xref>
    bundle.
  </description>
  <documentation details='Package documentation'
      href='ctan:/macros/latex/contrib/oberdiek/makerobust.pdf'/>
  <ctan file='true' path='/macros/latex/contrib/oberdiek/makerobust.dtx'/>
  <miktex location='oberdiek'/>
  <texlive location='oberdiek'/>
  <install path='/macros/latex/contrib/oberdiek/oberdiek.tds.zip'/>
</entry>
%</catalogue>
%    \end{macrocode}
%
% \begin{History}
%   \begin{Version}{2006/03/18 v1.0}
%   \item
%     First version.
%   \end{Version}
% \end{History}
%
% \PrintIndex
%
% \Finale
\endinput
